Un ulteriore metodo per la misurazione del fuoco della lente è quello che prevede l'utilizzo della legge dei punti coniugati.
 A partire dall'equazione di Gauss per le lenti sottili, apportando alcune correzioni legate al modello delle lenti
 spesse, si può trovare una serie di equazioni che descrivono il comportamento delle lenti attraverso semplici misure di
 lunghezze. Tali equazioni sono:
\[p=P_L - P_O + \frac {dr} {2} - \frac {\overline{PP'}}{2}\]
\[q=P_S - P_L -\frac {dr} {2} + \frac {\overline{PP'}} {2}\]
\begin{equation} \label{eq:punticoniugati}
\frac{1}{f} = \frac {1}{p} + \frac {1}{q}
\end{equation}

% con  p\textsubscript{o} distanza oggetto, p\textsubscript{L} distanza lente e p\textsubscript{S} distanza schermo.
 Sono stati prelevati due campioni diversi, dato che il primo campione possedeva delle imprecisioni. I campioni si possono trovare
 nella \autoref{tab:02tab1}.
%**02_tab_1.fdat**	**02_tab_2.txt**
\begin{tabella}
	\centering
	\begin{tabulary}{\textwidth}{CCCC}
\toprule
%primo campione
\multicolumn{2}{c}{Campione I} & \multicolumn{2}{c}{Campione II} \\
\cmidrule(r){1-2} \cmidrule(l){3-4}
$P_L$ & $P_S$ & $P_L$ & $P_S$ \\ \midrule
23.70 & 30.00 & 21.00 & 62.25 \\ \midrule
24.40 & 32.00 & 22.00 & 47.50 \\ \midrule
24.30 & 34.50 & 23.00 & 43.30 \\ \midrule
24.70 & 37.00 & 24.00 & 41.10 \\ \midrule
27.90 & 40.00 & 25.00 & 40.15 \\ \midrule
33.00 & 43.50 & 26.00 & 39.90 \\ \midrule
37.95 & 47.50 & 27.00 & 39.70 \\ \midrule
43.35 & 52.50 & 28.00 & 40.10 \\ \midrule
57.00 & 65.50 & 29.00 & 40.75 \\ \midrule
68.90 & 77.00 & 30.00 & 41.10 \\ \midrule
83.00 & 91.00 & 35.00 & 44.70 \\ \midrule
103.15 & 111.00 & 40.00 & 49.10 \\ \midrule
127.35 & 135.00 & 50.00 & 58.30 \\ \midrule
- & - & 62.00 & 69.90 \\ \midrule
- & - & 75.00 & 82.70 \\ \midrule
- & - & 90.00 & 97.50 \\
\bottomrule%
% \\ \midrule
%secondo campione
\end{tabulary}


	\caption{Campioni (udm in $[\cm\,]$)}
	\label{tab:02tab1}
\end{tabella}
%\begin{tabella}
	%\centering
	%\input{./tabelle/02_tab_2.tex}
	%\caption{Campione II $[\cm\,]$}
	%\label{tab:02tab2}
%\end{tabella}
%
I valori mantenuti costanti durante tutta l'esperienza e i paramentri fisici della lente sono:
\[P_O = 13.15 \cm , \nicefrac{dr}{2} = 0.115 \cm ,\frac {\overline{PP'}}{2} = \frac{\overline{VV'}}{6} = 0.133 \mm\]
Tali campioni sono stati presi in momenti diversi, in particolare sono stati riposizionati tutti i cavalieri tra un campione e l'
 altro. Si faccia un grafico per cercare di comprendere come i dati si ditribuiscono in un piano cartesiano: nel  \autoref{fig:02graph1.tex}, è rappresentato $\usuq$ in funzione di $\usup$, dalla formula
 \eqref{eq:punticoniugati} risulta che il grafico dovrebbe essere una retta con
 pendenza $-1$.
\begin{grafico} \centering \begin{tikzpicture}[gnuplot]
%% generated with GNUPLOT 4.6p4 (Lua 5.1; terminal rev. 99, script rev. 100)
%% lun 22 dic 2014 14:04:42 CET
\path (0.000,0.000) rectangle (12.500,8.750);
\gpcolor{color=gp lt color border}
\gpsetlinetype{gp lt border}
\gpsetlinewidth{1.00}
\draw[gp path] (1.688,0.985)--(1.868,0.985);
\draw[gp path] (11.947,0.985)--(11.767,0.985);
\node[gp node right] at (1.504,0.985) { 0.02};
\draw[gp path] (1.688,1.910)--(1.868,1.910);
\draw[gp path] (11.947,1.910)--(11.767,1.910);
\node[gp node right] at (1.504,1.910) { 0.04};
\draw[gp path] (1.688,2.834)--(1.868,2.834);
\draw[gp path] (11.947,2.834)--(11.767,2.834);
\node[gp node right] at (1.504,2.834) { 0.06};
\draw[gp path] (1.688,3.759)--(1.868,3.759);
\draw[gp path] (11.947,3.759)--(11.767,3.759);
\node[gp node right] at (1.504,3.759) { 0.08};
\draw[gp path] (1.688,4.683)--(1.868,4.683);
\draw[gp path] (11.947,4.683)--(11.767,4.683);
\node[gp node right] at (1.504,4.683) { 0.1};
\draw[gp path] (1.688,5.608)--(1.868,5.608);
\draw[gp path] (11.947,5.608)--(11.767,5.608);
\node[gp node right] at (1.504,5.608) { 0.12};
\draw[gp path] (1.688,6.532)--(1.868,6.532);
\draw[gp path] (11.947,6.532)--(11.767,6.532);
\node[gp node right] at (1.504,6.532) { 0.14};
\draw[gp path] (1.688,7.457)--(1.868,7.457);
\draw[gp path] (11.947,7.457)--(11.767,7.457);
\node[gp node right] at (1.504,7.457) { 0.16};
\draw[gp path] (1.688,8.381)--(1.868,8.381);
\draw[gp path] (11.947,8.381)--(11.767,8.381);
\node[gp node right] at (1.504,8.381) { 0.18};
\draw[gp path] (1.688,0.985)--(1.688,1.165);
\draw[gp path] (1.688,8.381)--(1.688,8.201);
\node[gp node center] at (1.688,0.677) { 0};
\draw[gp path] (3.154,0.985)--(3.154,1.165);
\draw[gp path] (3.154,8.381)--(3.154,8.201);
\node[gp node center] at (3.154,0.677) { 0.02};
\draw[gp path] (4.619,0.985)--(4.619,1.165);
\draw[gp path] (4.619,8.381)--(4.619,8.201);
\node[gp node center] at (4.619,0.677) { 0.04};
\draw[gp path] (6.085,0.985)--(6.085,1.165);
\draw[gp path] (6.085,8.381)--(6.085,8.201);
\node[gp node center] at (6.085,0.677) { 0.06};
\draw[gp path] (7.550,0.985)--(7.550,1.165);
\draw[gp path] (7.550,8.381)--(7.550,8.201);
\node[gp node center] at (7.550,0.677) { 0.08};
\draw[gp path] (9.016,0.985)--(9.016,1.165);
\draw[gp path] (9.016,8.381)--(9.016,8.201);
\node[gp node center] at (9.016,0.677) { 0.1};
\draw[gp path] (10.481,0.985)--(10.481,1.165);
\draw[gp path] (10.481,8.381)--(10.481,8.201);
\node[gp node center] at (10.481,0.677) { 0.12};
\draw[gp path] (11.947,0.985)--(11.947,1.165);
\draw[gp path] (11.947,8.381)--(11.947,8.201);
\node[gp node center] at (11.947,0.677) { 0.14};
\draw[gp path] (1.688,8.381)--(1.688,0.985)--(11.947,0.985)--(11.947,8.381)--cycle;
\node[gp node center,rotate=-270] at (0.246,4.683) {$\nicefrac{1}{q} [\cm^{-1}]$};
\node[gp node center] at (6.817,0.215) {$\nicefrac{1}{p} [\cm^{-1}]$};
\node[gp node right] at (10.479,8.047) {Campione I};
\gpcolor{color=gp lt color 0}
\gpsetpointsize{4.00}
\gppoint{gp mark 1}{(8.646,7.699)}
\gppoint{gp mark 1}{(8.212,6.348)}
\gppoint{gp mark 1}{(8.271,4.705)}
\gppoint{gp mark 1}{(8.043,3.896)}
\gppoint{gp mark 1}{(6.662,3.961)}
\gppoint{gp mark 1}{(5.383,4.570)}
\gppoint{gp mark 1}{(4.645,5.030)}
\gppoint{gp mark 1}{(4.116,5.253)}
\gppoint{gp mark 1}{(3.360,5.662)}
\gppoint{gp mark 1}{(3.003,5.948)}
\gppoint{gp mark 1}{(2.737,6.024)}
\gppoint{gp mark 1}{(2.502,6.141)}
\gppoint{gp mark 1}{(2.330,6.306)}
\gppoint{gp mark 1}{(11.121,8.047)}
\gpcolor{color=gp lt color border}
\node[gp node right] at (10.479,7.739) {Campione II};
\gpcolor{rgb color={0.000,0.502,0.000}}
\gppoint{gp mark 2}{(6.042,4.320)}
\gppoint{gp mark 2}{(6.317,4.079)}
\gppoint{gp mark 2}{(6.629,3.961)}
\gppoint{gp mark 2}{(6.986,3.773)}
\gppoint{gp mark 2}{(7.399,3.447)}
\gppoint{gp mark 2}{(7.881,3.163)}
\gppoint{gp mark 2}{(8.453,2.804)}
\gppoint{gp mark 2}{(9.141,2.366)}
\gppoint{gp mark 2}{(9.985,1.891)}
\gppoint{gp mark 2}{(11.045,1.188)}
\gppoint{gp mark 2}{(5.045,4.951)}
\gppoint{gp mark 2}{(4.419,5.283)}
\gppoint{gp mark 2}{(3.678,5.802)}
\gppoint{gp mark 2}{(3.189,6.102)}
\gppoint{gp mark 2}{(2.873,6.264)}
\gppoint{gp mark 2}{(2.642,6.435)}
\gppoint{gp mark 2}{(11.121,7.739)}
\gpcolor{color=gp lt color border}
\draw[gp path] (1.688,8.381)--(1.688,0.985)--(11.947,0.985)--(11.947,8.381)--cycle;
%% coordinates of the plot area
\gpdefrectangularnode{gp plot 1}{\pgfpoint{1.688cm}{0.985cm}}{\pgfpoint{11.947cm}{8.381cm}}
\end{tikzpicture}
%% gnuplot variables
 \caption{Punti coniugati} \label{fig:02graph1.tex} \end{grafico}

Si evince facilmente come nel "Campione I" ci siano dei punti visibilmente fuori scala, probabilmente a causa di
 inesattezze da parte degli sperimentatori nel prendere quei dati (quando la lente era particolarmente vicina all'oggetto risultava
 meno semplice andare ad identificare in maniera precisa a quale distanza sarebbe dovuto essere lo schermo affinché l'immagine
 fosse a fuoco). Alcune accortezze, prese proprio in risposta al primo campione poco convincente, hanno permesso di prendere
 un altro set di misure che fosse più preciso del precedente. Ma i due campioni appartengono alla stessa popolazione? La risposta
 risulta evidentemente no, infatti anche nella zona in cui il "Campione I" non è evidentemente sbagliato le due rette non coincidono,
 come si può vedere dal \autoref{fig:02graph2.tex} in cui sono stati riportati i dati non evidentemente sbagliati assieme alla rette
 che li
 interpolano $y = ax + b$, che si possono trovare nella \autoref{tab:02tab3}.
%**02_graph_2.txt**	**02_tab_3.txt**
\begin{grafico} \centering \begin{tikzpicture}[gnuplot]
%% generated with GNUPLOT 4.6p4 (Lua 5.1; terminal rev. 99, script rev. 100)
%% lun 22 dic 2014 16:08:38 CET
\path (0.000,0.000) rectangle (12.500,8.750);
\gpcolor{color=gp lt color border}
\gpsetlinetype{gp lt border}
\gpsetlinewidth{1.00}
\draw[gp path] (1.868,1.165)--(1.688,1.165);
\node[gp node right] at (1.504,1.165) { 0};
\draw[gp path] (1.868,1.616)--(1.778,1.616);
\draw[gp path] (1.868,2.067)--(1.688,2.067);
\node[gp node right] at (1.504,2.067) { 0.02};
\draw[gp path] (1.868,2.518)--(1.778,2.518);
\draw[gp path] (1.868,2.969)--(1.688,2.969);
\node[gp node right] at (1.504,2.969) { 0.04};
\draw[gp path] (1.868,3.420)--(1.778,3.420);
\draw[gp path] (1.868,3.871)--(1.688,3.871);
\node[gp node right] at (1.504,3.871) { 0.06};
\draw[gp path] (1.868,4.322)--(1.778,4.322);
\draw[gp path] (1.868,4.773)--(1.688,4.773);
\node[gp node right] at (1.504,4.773) { 0.08};
\draw[gp path] (1.868,5.224)--(1.778,5.224);
\draw[gp path] (1.868,5.675)--(1.688,5.675);
\node[gp node right] at (1.504,5.675) { 0.1};
\draw[gp path] (1.868,6.126)--(1.778,6.126);
\draw[gp path] (1.868,6.577)--(1.688,6.577);
\node[gp node right] at (1.504,6.577) { 0.12};
\draw[gp path] (1.868,7.028)--(1.778,7.028);
\draw[gp path] (1.868,7.479)--(1.688,7.479);
\node[gp node right] at (1.504,7.479) { 0.14};
\draw[gp path] (1.868,7.930)--(1.778,7.930);
\draw[gp path] (1.868,8.381)--(1.688,8.381);
\node[gp node right] at (1.504,8.381) { 0.16};
\draw[gp path] (1.868,1.165)--(1.868,0.985);
\node[gp node center] at (1.868,0.677) { 0};
\draw[gp path] (2.588,1.165)--(2.588,1.075);
\draw[gp path] (3.308,1.165)--(3.308,0.985);
\node[gp node center] at (3.308,0.677) { 0.02};
\draw[gp path] (4.028,1.165)--(4.028,1.075);
\draw[gp path] (4.748,1.165)--(4.748,0.985);
\node[gp node center] at (4.748,0.677) { 0.04};
\draw[gp path] (5.468,1.165)--(5.468,1.075);
\draw[gp path] (6.188,1.165)--(6.188,0.985);
\node[gp node center] at (6.188,0.677) { 0.06};
\draw[gp path] (6.908,1.165)--(6.908,1.075);
\draw[gp path] (7.627,1.165)--(7.627,0.985);
\node[gp node center] at (7.627,0.677) { 0.08};
\draw[gp path] (8.347,1.165)--(8.347,1.075);
\draw[gp path] (9.067,1.165)--(9.067,0.985);
\node[gp node center] at (9.067,0.677) { 0.1};
\draw[gp path] (9.787,1.165)--(9.787,1.075);
\draw[gp path] (10.507,1.165)--(10.507,0.985);
\node[gp node center] at (10.507,0.677) { 0.12};
\draw[gp path] (11.227,1.165)--(11.227,1.075);
\draw[gp path] (11.947,1.165)--(11.947,0.985);
\node[gp node center] at (11.947,0.677) { 0.14};
\draw[gp path] (1.868,8.381)--(1.868,1.165)--(11.947,1.165);
\node[gp node center,rotate=-270] at (0.246,4.773) {$\nicefrac{1}{q} [\cm^{-1}]$};
\node[gp node center] at (6.907,0.215) {$\nicefrac{1}{p} [\cm^{-1}]$};
\node[gp node right] at (10.479,8.047) {Campione I};
\gpcolor{rgb color={0.698,0.000,0.000}}
\gpsetpointsize{4.00}
\gppoint{gp mark 1}{(6.145,5.321)}
\gppoint{gp mark 1}{(6.415,5.086)}
\gppoint{gp mark 1}{(6.722,4.970)}
\gppoint{gp mark 1}{(7.073,4.787)}
\gppoint{gp mark 1}{(7.479,4.469)}
\gppoint{gp mark 1}{(7.953,4.192)}
\gppoint{gp mark 1}{(8.515,3.841)}
\gppoint{gp mark 1}{(9.191,3.414)}
\gppoint{gp mark 1}{(10.020,2.951)}
\gppoint{gp mark 1}{(11.061,2.265)}
\gppoint{gp mark 1}{(5.166,5.937)}
\gppoint{gp mark 1}{(4.551,6.260)}
\gppoint{gp mark 1}{(3.823,6.766)}
\gppoint{gp mark 1}{(3.342,7.059)}
\gppoint{gp mark 1}{(3.032,7.217)}
\gppoint{gp mark 1}{(2.805,7.384)}
\gppoint{gp mark 1}{(11.121,8.047)}
\gpcolor{color=gp lt color border}
\node[gp node right] at (10.479,7.739) {Campione II};
\gpcolor{rgb color={0.000,0.000,0.800}}
\gppoint{gp mark 2}{(5.498,5.564)}
\gppoint{gp mark 2}{(4.773,6.014)}
\gppoint{gp mark 2}{(4.253,6.231)}
\gppoint{gp mark 2}{(3.510,6.631)}
\gppoint{gp mark 2}{(3.160,6.909)}
\gppoint{gp mark 2}{(2.899,6.983)}
\gppoint{gp mark 2}{(2.668,7.098)}
\gppoint{gp mark 2}{(2.499,7.258)}
\gppoint{gp mark 2}{(11.121,7.739)}
\gpcolor{color=gp lt color border}
\node[gp node right] at (10.479,7.431) {f(x)};
\gpcolor{rgb color={1.000,0.000,0.000}}
\gpsetlinetype{gp lt plot 2}
\draw[gp path] (10.663,7.431)--(11.579,7.431);
\draw[gp path] (1.868,7.955)--(1.961,7.897)--(2.054,7.840)--(2.147,7.783)--(2.239,7.725)%
  --(2.332,7.668)--(2.425,7.610)--(2.518,7.553)--(2.611,7.496)--(2.704,7.438)--(2.797,7.381)%
  --(2.889,7.324)--(2.982,7.266)--(3.075,7.209)--(3.168,7.151)--(3.261,7.094)--(3.354,7.037)%
  --(3.447,6.979)--(3.539,6.922)--(3.632,6.865)--(3.725,6.807)--(3.818,6.750)--(3.911,6.692)%
  --(4.004,6.635)--(4.096,6.578)--(4.189,6.520)--(4.282,6.463)--(4.375,6.406)--(4.468,6.348)%
  --(4.561,6.291)--(4.654,6.233)--(4.746,6.176)--(4.839,6.119)--(4.932,6.061)--(5.025,6.004)%
  --(5.118,5.947)--(5.211,5.889)--(5.304,5.832)--(5.396,5.774)--(5.489,5.717)--(5.582,5.660)%
  --(5.675,5.602)--(5.768,5.545)--(5.861,5.488)--(5.954,5.430)--(6.046,5.373)--(6.139,5.316)%
  --(6.232,5.258)--(6.325,5.201)--(6.418,5.143)--(6.511,5.086)--(6.604,5.029)--(6.696,4.971)%
  --(6.789,4.914)--(6.882,4.857)--(6.975,4.799)--(7.068,4.742)--(7.161,4.684)--(7.254,4.627)%
  --(7.346,4.570)--(7.439,4.512)--(7.532,4.455)--(7.625,4.398)--(7.718,4.340)--(7.811,4.283)%
  --(7.904,4.225)--(7.996,4.168)--(8.089,4.111)--(8.182,4.053)--(8.275,3.996)--(8.368,3.939)%
  --(8.461,3.881)--(8.553,3.824)--(8.646,3.766)--(8.739,3.709)--(8.832,3.652)--(8.925,3.594)%
  --(9.018,3.537)--(9.111,3.480)--(9.203,3.422)--(9.296,3.365)--(9.389,3.307)--(9.482,3.250)%
  --(9.575,3.193)--(9.668,3.135)--(9.761,3.078)--(9.853,3.021)--(9.946,2.963)--(10.039,2.906)%
  --(10.132,2.848)--(10.225,2.791)--(10.318,2.734)--(10.411,2.676)--(10.503,2.619)--(10.596,2.562)%
  --(10.689,2.504)--(10.782,2.447)--(10.875,2.389)--(10.968,2.332)--(11.061,2.275);
\gpcolor{color=gp lt color border}
\node[gp node right] at (10.479,7.123) {g(x)};
\gpcolor{rgb color={0.102,0.000,1.000}}
\gpsetlinetype{gp lt plot 3}
\draw[gp path] (10.663,7.123)--(11.579,7.123);
\draw[gp path] (1.868,7.570)--(1.961,7.519)--(2.054,7.468)--(2.147,7.417)--(2.239,7.366)%
  --(2.332,7.315)--(2.425,7.264)--(2.518,7.213)--(2.611,7.162)--(2.704,7.111)--(2.797,7.060)%
  --(2.889,7.008)--(2.982,6.957)--(3.075,6.906)--(3.168,6.855)--(3.261,6.804)--(3.354,6.753)%
  --(3.447,6.702)--(3.539,6.651)--(3.632,6.600)--(3.725,6.549)--(3.818,6.498)--(3.911,6.447)%
  --(4.004,6.396)--(4.096,6.345)--(4.189,6.294)--(4.282,6.242)--(4.375,6.191)--(4.468,6.140)%
  --(4.561,6.089)--(4.654,6.038)--(4.746,5.987)--(4.839,5.936)--(4.932,5.885)--(5.025,5.834)%
  --(5.118,5.783)--(5.211,5.732)--(5.304,5.681)--(5.396,5.630)--(5.489,5.579)--(5.582,5.528)%
  --(5.675,5.477)--(5.768,5.425)--(5.861,5.374)--(5.954,5.323)--(6.046,5.272)--(6.139,5.221)%
  --(6.232,5.170)--(6.325,5.119)--(6.418,5.068)--(6.511,5.017)--(6.604,4.966)--(6.696,4.915)%
  --(6.789,4.864)--(6.882,4.813)--(6.975,4.762)--(7.068,4.711)--(7.161,4.659)--(7.254,4.608)%
  --(7.346,4.557)--(7.439,4.506)--(7.532,4.455)--(7.625,4.404)--(7.718,4.353)--(7.811,4.302)%
  --(7.904,4.251)--(7.996,4.200)--(8.089,4.149)--(8.182,4.098)--(8.275,4.047)--(8.368,3.996)%
  --(8.461,3.945)--(8.553,3.894)--(8.646,3.842)--(8.739,3.791)--(8.832,3.740)--(8.925,3.689)%
  --(9.018,3.638)--(9.111,3.587)--(9.203,3.536)--(9.296,3.485)--(9.389,3.434)--(9.482,3.383)%
  --(9.575,3.332)--(9.668,3.281)--(9.761,3.230)--(9.853,3.179)--(9.946,3.128)--(10.039,3.076)%
  --(10.132,3.025)--(10.225,2.974)--(10.318,2.923)--(10.411,2.872)--(10.503,2.821)--(10.596,2.770)%
  --(10.689,2.719)--(10.782,2.668)--(10.875,2.617)--(10.968,2.566)--(11.061,2.515);
\gpcolor{color=gp lt color border}
\gpsetlinetype{gp lt border}
\draw[gp path] (1.868,8.381)--(1.868,1.165)--(11.947,1.165);
%% coordinates of the plot area
\gpdefrectangularnode{gp plot 1}{\pgfpoint{1.868cm}{1.165cm}}{\pgfpoint{11.947cm}{8.381cm}}
\end{tikzpicture}
%% gnuplot variables
 \caption{Le interpolazioni dei due campioni} \label{fig:02graph2.tex} \end{grafico}
\begin{tabella}
	\centering
	\begin{tabulary}{\textwidth}{CCCCC}
\toprule
Campione & $a$  & $\sigma_a$ & $b [\cm^{-1}]$ & $\sigma_b [\cm^{-1}]$\\ \midrule
I & -0.88 & 0.02 & 0.1420 & 0.0006\\ %\midrule
II & -0.986 & 0.004 & 0.1505 & 0.0003\\
\bottomrule
\end{tabulary}


	\caption{Rette interpolanti}
	\label{tab:02tab3}
\end{tabella}

Come è stato già detto, la teoria ci dice che la retta che rappresentiamo sul piano cartesiano ha pendenza $-1$. Il nostro approccio
 sperimentale, però, inserisce errori di tipo statistico e sisetmatico che fanno in modo che la retta che effettivamente si trova
 non ha coefficiente angolare esattamente $-1$ ma ci si avvicina. A questo punto sorge spontaneo chiedersi se convenga di più fare
 un'analisi dati basata su un'interpolazione a doppio parametro (pendenza, intercetta), o fissare la pendenza a $-1$ come vuole la
 teoria andando a cercare quale sia l'intercetta in qusto caso. Le due diverse interpolazioni danno origine al
 \autoref{fig:02_graph_3.tex}.
%**02_graph_3.txt**	**02_tab_4.txt**
\begin{grafico} \centering \begin{tikzpicture}[gnuplot]
%% generated with GNUPLOT 4.6p4 (Lua 5.1; terminal rev. 99, script rev. 100)
%% lun 22 dic 2014 16:08:38 CET
\path (0.000,0.000) rectangle (12.500,8.750);
\gpcolor{color=gp lt color border}
\gpsetlinetype{gp lt border}
\gpsetlinewidth{1.00}
\draw[gp path] (1.868,1.165)--(1.688,1.165);
\node[gp node right] at (1.504,1.165) { 0};
\draw[gp path] (1.868,1.616)--(1.778,1.616);
\draw[gp path] (1.868,2.067)--(1.688,2.067);
\node[gp node right] at (1.504,2.067) { 0.02};
\draw[gp path] (1.868,2.518)--(1.778,2.518);
\draw[gp path] (1.868,2.969)--(1.688,2.969);
\node[gp node right] at (1.504,2.969) { 0.04};
\draw[gp path] (1.868,3.420)--(1.778,3.420);
\draw[gp path] (1.868,3.871)--(1.688,3.871);
\node[gp node right] at (1.504,3.871) { 0.06};
\draw[gp path] (1.868,4.322)--(1.778,4.322);
\draw[gp path] (1.868,4.773)--(1.688,4.773);
\node[gp node right] at (1.504,4.773) { 0.08};
\draw[gp path] (1.868,5.224)--(1.778,5.224);
\draw[gp path] (1.868,5.675)--(1.688,5.675);
\node[gp node right] at (1.504,5.675) { 0.1};
\draw[gp path] (1.868,6.126)--(1.778,6.126);
\draw[gp path] (1.868,6.577)--(1.688,6.577);
\node[gp node right] at (1.504,6.577) { 0.12};
\draw[gp path] (1.868,7.028)--(1.778,7.028);
\draw[gp path] (1.868,7.479)--(1.688,7.479);
\node[gp node right] at (1.504,7.479) { 0.14};
\draw[gp path] (1.868,7.930)--(1.778,7.930);
\draw[gp path] (1.868,8.381)--(1.688,8.381);
\node[gp node right] at (1.504,8.381) { 0.16};
\draw[gp path] (1.868,1.165)--(1.868,0.985);
\node[gp node center] at (1.868,0.677) { 0};
\draw[gp path] (2.588,1.165)--(2.588,1.075);
\draw[gp path] (3.308,1.165)--(3.308,0.985);
\node[gp node center] at (3.308,0.677) { 0.02};
\draw[gp path] (4.028,1.165)--(4.028,1.075);
\draw[gp path] (4.748,1.165)--(4.748,0.985);
\node[gp node center] at (4.748,0.677) { 0.04};
\draw[gp path] (5.468,1.165)--(5.468,1.075);
\draw[gp path] (6.188,1.165)--(6.188,0.985);
\node[gp node center] at (6.188,0.677) { 0.06};
\draw[gp path] (6.908,1.165)--(6.908,1.075);
\draw[gp path] (7.627,1.165)--(7.627,0.985);
\node[gp node center] at (7.627,0.677) { 0.08};
\draw[gp path] (8.347,1.165)--(8.347,1.075);
\draw[gp path] (9.067,1.165)--(9.067,0.985);
\node[gp node center] at (9.067,0.677) { 0.1};
\draw[gp path] (9.787,1.165)--(9.787,1.075);
\draw[gp path] (10.507,1.165)--(10.507,0.985);
\node[gp node center] at (10.507,0.677) { 0.12};
\draw[gp path] (11.227,1.165)--(11.227,1.075);
\draw[gp path] (11.947,1.165)--(11.947,0.985);
\node[gp node center] at (11.947,0.677) { 0.14};
\draw[gp path] (1.868,8.381)--(1.868,1.165)--(11.947,1.165);
\node[gp node center,rotate=-270] at (0.246,4.773) {$\nicefrac{1}{q} [\cm^{-1}]$};
\node[gp node center] at (6.907,0.215) {$\nicefrac{1}{p} [\cm^{-1}]$};
\node[gp node right] at (10.479,8.047) {Dati Sperimentali};
\gpcolor{color=gp lt color 0}
\gpsetpointsize{4.00}
\gppoint{gp mark 1}{(6.145,5.321)}
\gppoint{gp mark 1}{(6.415,5.086)}
\gppoint{gp mark 1}{(6.722,4.970)}
\gppoint{gp mark 1}{(7.073,4.787)}
\gppoint{gp mark 1}{(7.479,4.469)}
\gppoint{gp mark 1}{(7.953,4.192)}
\gppoint{gp mark 1}{(8.515,3.841)}
\gppoint{gp mark 1}{(9.191,3.414)}
\gppoint{gp mark 1}{(10.020,2.951)}
\gppoint{gp mark 1}{(11.061,2.265)}
\gppoint{gp mark 1}{(5.166,5.937)}
\gppoint{gp mark 1}{(4.551,6.260)}
\gppoint{gp mark 1}{(3.823,6.766)}
\gppoint{gp mark 1}{(3.342,7.059)}
\gppoint{gp mark 1}{(3.032,7.217)}
\gppoint{gp mark 1}{(2.805,7.384)}
\gppoint{gp mark 1}{(11.121,8.047)}
\gpcolor{color=gp lt color border}
\node[gp node right] at (10.479,7.739) {Interpolazione con due parametri};
\gpcolor{rgb color={0.000,0.502,0.000}}
\gpsetlinetype{gp lt plot 1}
\draw[gp path] (10.663,7.739)--(11.579,7.739);
\draw[gp path] (1.868,7.955)--(1.961,7.897)--(2.054,7.840)--(2.147,7.783)--(2.239,7.725)%
  --(2.332,7.668)--(2.425,7.610)--(2.518,7.553)--(2.611,7.496)--(2.704,7.438)--(2.797,7.381)%
  --(2.889,7.324)--(2.982,7.266)--(3.075,7.209)--(3.168,7.151)--(3.261,7.094)--(3.354,7.037)%
  --(3.447,6.979)--(3.539,6.922)--(3.632,6.865)--(3.725,6.807)--(3.818,6.750)--(3.911,6.692)%
  --(4.004,6.635)--(4.096,6.578)--(4.189,6.520)--(4.282,6.463)--(4.375,6.406)--(4.468,6.348)%
  --(4.561,6.291)--(4.654,6.233)--(4.746,6.176)--(4.839,6.119)--(4.932,6.061)--(5.025,6.004)%
  --(5.118,5.947)--(5.211,5.889)--(5.304,5.832)--(5.396,5.774)--(5.489,5.717)--(5.582,5.660)%
  --(5.675,5.602)--(5.768,5.545)--(5.861,5.488)--(5.954,5.430)--(6.046,5.373)--(6.139,5.316)%
  --(6.232,5.258)--(6.325,5.201)--(6.418,5.143)--(6.511,5.086)--(6.604,5.029)--(6.696,4.971)%
  --(6.789,4.914)--(6.882,4.857)--(6.975,4.799)--(7.068,4.742)--(7.161,4.684)--(7.254,4.627)%
  --(7.346,4.570)--(7.439,4.512)--(7.532,4.455)--(7.625,4.398)--(7.718,4.340)--(7.811,4.283)%
  --(7.904,4.225)--(7.996,4.168)--(8.089,4.111)--(8.182,4.053)--(8.275,3.996)--(8.368,3.939)%
  --(8.461,3.881)--(8.553,3.824)--(8.646,3.766)--(8.739,3.709)--(8.832,3.652)--(8.925,3.594)%
  --(9.018,3.537)--(9.111,3.480)--(9.203,3.422)--(9.296,3.365)--(9.389,3.307)--(9.482,3.250)%
  --(9.575,3.193)--(9.668,3.135)--(9.761,3.078)--(9.853,3.021)--(9.946,2.963)--(10.039,2.906)%
  --(10.132,2.848)--(10.225,2.791)--(10.318,2.734)--(10.411,2.676)--(10.503,2.619)--(10.596,2.562)%
  --(10.689,2.504)--(10.782,2.447)--(10.875,2.389)--(10.968,2.332)--(11.061,2.275);
\gpcolor{color=gp lt color border}
\node[gp node right] at (10.479,7.431) {Interpolazione con un parametro};
\gpcolor{color=gp lt color 2}
\gpsetlinetype{gp lt plot 2}
\draw[gp path] (10.663,7.431)--(11.579,7.431);
\draw[gp path] (1.868,7.994)--(1.961,7.936)--(2.054,7.878)--(2.147,7.819)--(2.239,7.761)%
  --(2.332,7.703)--(2.425,7.645)--(2.518,7.587)--(2.611,7.529)--(2.704,7.470)--(2.797,7.412)%
  --(2.889,7.354)--(2.982,7.296)--(3.075,7.238)--(3.168,7.180)--(3.261,7.121)--(3.354,7.063)%
  --(3.447,7.005)--(3.539,6.947)--(3.632,6.889)--(3.725,6.831)--(3.818,6.772)--(3.911,6.714)%
  --(4.004,6.656)--(4.096,6.598)--(4.189,6.540)--(4.282,6.482)--(4.375,6.423)--(4.468,6.365)%
  --(4.561,6.307)--(4.654,6.249)--(4.746,6.191)--(4.839,6.133)--(4.932,6.074)--(5.025,6.016)%
  --(5.118,5.958)--(5.211,5.900)--(5.304,5.842)--(5.396,5.784)--(5.489,5.725)--(5.582,5.667)%
  --(5.675,5.609)--(5.768,5.551)--(5.861,5.493)--(5.954,5.435)--(6.046,5.376)--(6.139,5.318)%
  --(6.232,5.260)--(6.325,5.202)--(6.418,5.144)--(6.511,5.086)--(6.604,5.027)--(6.696,4.969)%
  --(6.789,4.911)--(6.882,4.853)--(6.975,4.795)--(7.068,4.737)--(7.161,4.678)--(7.254,4.620)%
  --(7.346,4.562)--(7.439,4.504)--(7.532,4.446)--(7.625,4.388)--(7.718,4.329)--(7.811,4.271)%
  --(7.904,4.213)--(7.996,4.155)--(8.089,4.097)--(8.182,4.039)--(8.275,3.980)--(8.368,3.922)%
  --(8.461,3.864)--(8.553,3.806)--(8.646,3.748)--(8.739,3.690)--(8.832,3.631)--(8.925,3.573)%
  --(9.018,3.515)--(9.111,3.457)--(9.203,3.399)--(9.296,3.340)--(9.389,3.282)--(9.482,3.224)%
  --(9.575,3.166)--(9.668,3.108)--(9.761,3.050)--(9.853,2.991)--(9.946,2.933)--(10.039,2.875)%
  --(10.132,2.817)--(10.225,2.759)--(10.318,2.701)--(10.411,2.642)--(10.503,2.584)--(10.596,2.526)%
  --(10.689,2.468)--(10.782,2.410)--(10.875,2.352)--(10.968,2.293)--(11.061,2.235);
\gpcolor{color=gp lt color border}
\gpsetlinetype{gp lt border}
\draw[gp path] (1.868,8.381)--(1.868,1.165)--(11.947,1.165);
%% coordinates of the plot area
\gpdefrectangularnode{gp plot 1}{\pgfpoint{1.868cm}{1.165cm}}{\pgfpoint{11.947cm}{8.381cm}}
\end{tikzpicture}
%% gnuplot variables
 \caption{Le due diverse interpolazioni} \label{fig:02_graph_3.tex} \end{grafico}
\begin{tabella}
	\centering
	\begin{tabulary}{\textwidth}{CCCCC}
\toprule
Numero parametri & $a$  & $\sigma_a$ & $b [\cm^{-1}]$ & $\sigma_b [\cm^{-1}]$\\ \midrule
I & -1 & - & 0.1514 & 0.0002\\
II & -0.986 & 0.004 & 0.1505 & 0.0003\\
\bottomrule
\end{tabulary}


	\caption{Numero parametri d'interpolazione}
	\label{tab:02tab4}
\end{tabella}

Per comprendere quale dei due modi di interpretare il fenomeno sia più opportuno usare si utilizzi il metodo dell'$\mathcal{F}$-test:
 sono stati presi gli scarti quadratici medi delle due interpolazioni diverse e sono stati normalizzati ai gradi di libertà
 (sono state raccolte 16 coppie di dati, ciò vuol dire che la retta con un solo parametro libero ha 15 gradi di libertà,
 siano $n_1$ nelle formule, e la retta con due parametri liberi ha 14 gradi di libertà, $n_2$).
 La formula utilizzata è:
\[\mathcal{F}=\frac{\dfrac{{\sum_i \big( y_i-f_1 (c_i) \big) ^2} - {\sum_i \big( y_i-f_2 (c_i) \big) ^2}}{n_1-n_2}}{\dfrac{\sum_i \big( y_i-f_2 
(c_i) \big) ^2}{n_2}}.\]

Andando a calcolare la $\mathcal{F}$ da questa formula risulta $\mathcal{F} = 0.1740$. Ora, per capire quale delle due interpolazioni meglio si addice alla
 casistica trovata, va cercato nelle tabelle di riferimento relative alla funzione di Von Mises-Fisher, andando a leggere il valore
 in corrispondenza dei gradi di liberà del numeratore (in questo caso 1) per trovare la colonna e i gradi di libertà del denominatore
 (in questo caso $14$) per trovare la riga, si legge che vale la pena introdurre un nuovo parametro alla teoria 
 se \footnote{Si prenda una significanza del $90\%$, riferimento tabella: \url{http://www.socr.ucla.edu/applets.dir/f_table.html}} $\mathcal{F} > 3.10$, 
 da cui evinciamo come non sia necessaria la stima del secondo parametro d'interpolazione della retta, che può essere fissato senza perdere
 troppe informazioni a $-1$. Per limitare le perdite di dati si effettuino le analisi sia considerando la pendenza della retta fissa,
 sia considerandola come una variabile casuale dipendente dal campione.

Considerando la pendenza dipendente dal campione, abbiamo una retta che interseca gli assi in due punti di coordinate diverse,
 coordinate che indicano proprio due stime diverse del reciproco del fuoco della lente, che chiameremo rispettivamente
$f_x$ se l'intersezione è con l'asse delle $x$ e $f_y$ se l'intersezione è con l'asse delle $y$.
 Per quanto riguarda $f_y$, la sua stima è piuttosto semplice: prendendo l'equazione della retta interpolante $y=ax + b$
 con $a$ e $b$ letti dalla tabella di cui sopra, basta cercare il punto con la $x$ nulla per trovare l'intersezione con gli assi.
 Con una banale sostituzione otteniamo $f_y= b$ il che ci dà immediatamente un valore reciproco del fuoco che sia anche
 accompagnato dalla sua incertezza: con una semplice propagazione si trova che l'incertezza su $f_y$ è uguale a quella su $b$,
 stimata a partire dalle formule di interpolazione che massimizzano la verosimiglianza (anche questo riportato nella 
\autoref{tab:02tab4}).
 Il primo valore del reciproco del fuoco quindi risulta:
\[f_y = (0.1505 \pm 0.0003) \cm^{-1}.\]

Per quanto riguarda l'intersezione con l'asse delle $x$, stavolta il calcolo è leggermente più complesso, infatti sostituendo $y = 0$
 alla formula della retta interpolata si ottiene $f_x = -\frac{q}{m}$.  Per trovare l'incertezza a questo punto risulta necessario
 propagare gli errori su $m$ e su $q$. Applicando la solita formula di propagazione risulta
\[\sigma_{f_x} = \sqrt {\frac{1} {m^2} \sigma_q^2 + \frac{q}{m^2} \sigma_m^2 + 2 \cov (q,m) }.\] Sostituendo,
 il reciproco del fuoco ha un valore di:
\[f_x = (0.153 \pm 0.002) \cm^{-1}.\]

Per trovare il fuoco è necessario calcolare i reciproci di $f_x$ ed $f_y$, andando a propagare gli errori secondo la formula di 
propagazione: $f'_x = \frac{1}{f_x}$ e $\sigma_{f'x} = \frac{\sigma_{fx}}{f_{x}^2}$ ed analogo per $f_y$. Da cui i risultati:
\[f'_x = (6.55 \pm 0.09) \cm \]
\[f'_y = (6.64 \pm 0.01) \cm .\]

Per dare un risultato finale, si possono includere tutti gli errori nel fatto che questi due numeri sono diversi: si può andare a
 stimare effettivamente il fuoco $f$ facendo una media aritmetica dei valori ottenuti e si può stimare l'incertezza
 andando a vedere la
 semidifferenza tra i due valori casuali ottenuti. Il risultato finale è:
\[f = (6.60 \pm 0.04) \cm .\]

Si veda ora come si sarebbero potuti analizzare i dati considerando la retta con l'unico parametro libero l'intercetta. In questo
 caso il risultato del fuoco, come del resto un'indicazione del suo errore statistico, ci vengono dati dai parametri interpolati e
 riportati nella \autoref{tab:02tab4}, per cui la lunghezza focale, calcolata come nel caso precedente trovando il reciproco
 dell'intercetta e propagando gli errori, sembrerebbe valere:
\[f' = (6.604 \pm 0.009) \cm .\]
Ricordando però che i campioni raccolti per la stima della lunghezza focale attraverso il metodo dei punti coniugati erano due ed
 erano visibilmente poco compatibili tra loro, sorge spontaneo chiedersi quali siano gli errori sistematici collegati a tale metodo.
 Dalle formule della legge dei punti coniugati risulta che p dipende dalla posizione iniziale dell'oggetto, che tra
 l'altro rimane fisso per tutta la durata dell'esperimento (probabilmente è da ricondurre ad una non perfetta lettura della posizione
 dell'oggetto il fatto che i due campioni non siano sottoinsiemi della stessa popolazione), il che vuol dire che un'errata lettura
 nella posizione iniziale dell'oggetto può sistematicamente creare un bias nel campione, spostando la retta interpolante dalla
 posizione che realmente dovrebbe occupare. Per stimare tale errore sistematico sono stati creati dei campioni fittizi di $\usup$ e
 di $\usuq$ considerando l'oggetto non nella posizione registrata al momento dell'esperimento, ma spostata di una sigma (0.5 mm) da
 uno dei due lati. I valori ottenuti sono riassunti nella \autoref{tab:02tab5}.
%**02_tab_5.txt**
\begin{tabella}
	\centering
	\begin{tabulary}{\textwidth}{XXXX}
\toprule
$\usup$ con $p_o$ aumentato & $\usup$ campione & $\usup$ con $p_o$ diminuito & $\usuq$\\ \midrule
0.0596 & 0.0594 & 0.0592 & 0.0922\\ \midrule
0.0634 & 0.0632 & 0.0630 & 0.0869\\ \midrule
0.0677 & 0.0674 & 0.0672 & 0.0844\\ \midrule
0.0726 & 0.0723 & 0.0720 & 0.0803\\ \midrule
0.0782 & 0.0779 & 0.0776 & 0.0733\\ \midrule
0.0849 & 0.0845 & 0.0842 & 0.0671\\ \midrule
0.0927 & 0.0923 & 0.0919 & 0.0593\\ \midrule
0.1022 & 0.1017 & 0.1012 & 0.0499\\ \midrule
0.1139 & 0.1132 & 0.1126 & 0.0396\\ \midrule
0.1285 & 0.1277 & 0.1269 & 0.0244\\ \midrule
0.0459 & 0.0458 & 0.0457 & 0.1058\\ \midrule
0.0373 & 0.0373 & 0.0372 & 0.1130\\ \midrule
0.0272 & 0.0272 & 0.0271 & 0.1242\\ \midrule
0.0205 & 0.0205 & 0.0205 & 0.1307\\ \midrule
0.0162 & 0.0162 & 0.0162 & 0.1342\\ \midrule
0.0130 & 0.0130 & 0.0130 & 0.1379\\
\bottomrule
\end{tabulary}

	\caption{Campioni con errori sistematici $[\cm^{-1}]$}
	\label{tab:02tab5}
\end{tabella}

Andando a rappresentare sia il campione reale che i campioni fittizi con le relative rette interpolate si vede il 
\autoref{fig:02_graph_4.tex}.
%**02_graf_4.txt**	**02_tab_6.txt**
\begin{grafico} \centering \begin{tikzpicture}[gnuplot]
%% generated with GNUPLOT 4.6p4 (Lua 5.1; terminal rev. 99, script rev. 100)
%% lun 22 dic 2014 16:08:38 CET
\path (0.000,0.000) rectangle (12.500,8.750);
\gpcolor{color=gp lt color border}
\gpsetlinetype{gp lt border}
\gpsetlinewidth{1.00}
\draw[gp path] (1.868,1.165)--(1.688,1.165);
\node[gp node right] at (1.504,1.165) { 0};
\draw[gp path] (1.868,1.616)--(1.778,1.616);
\draw[gp path] (1.868,2.067)--(1.688,2.067);
\node[gp node right] at (1.504,2.067) { 0.02};
\draw[gp path] (1.868,2.518)--(1.778,2.518);
\draw[gp path] (1.868,2.969)--(1.688,2.969);
\node[gp node right] at (1.504,2.969) { 0.04};
\draw[gp path] (1.868,3.420)--(1.778,3.420);
\draw[gp path] (1.868,3.871)--(1.688,3.871);
\node[gp node right] at (1.504,3.871) { 0.06};
\draw[gp path] (1.868,4.322)--(1.778,4.322);
\draw[gp path] (1.868,4.773)--(1.688,4.773);
\node[gp node right] at (1.504,4.773) { 0.08};
\draw[gp path] (1.868,5.224)--(1.778,5.224);
\draw[gp path] (1.868,5.675)--(1.688,5.675);
\node[gp node right] at (1.504,5.675) { 0.1};
\draw[gp path] (1.868,6.126)--(1.778,6.126);
\draw[gp path] (1.868,6.577)--(1.688,6.577);
\node[gp node right] at (1.504,6.577) { 0.12};
\draw[gp path] (1.868,7.028)--(1.778,7.028);
\draw[gp path] (1.868,7.479)--(1.688,7.479);
\node[gp node right] at (1.504,7.479) { 0.14};
\draw[gp path] (1.868,7.930)--(1.778,7.930);
\draw[gp path] (1.868,8.381)--(1.688,8.381);
\node[gp node right] at (1.504,8.381) { 0.16};
\draw[gp path] (1.868,1.165)--(1.868,0.985);
\node[gp node center] at (1.868,0.677) { 0};
\draw[gp path] (2.588,1.165)--(2.588,1.075);
\draw[gp path] (3.308,1.165)--(3.308,0.985);
\node[gp node center] at (3.308,0.677) { 0.02};
\draw[gp path] (4.028,1.165)--(4.028,1.075);
\draw[gp path] (4.748,1.165)--(4.748,0.985);
\node[gp node center] at (4.748,0.677) { 0.04};
\draw[gp path] (5.468,1.165)--(5.468,1.075);
\draw[gp path] (6.188,1.165)--(6.188,0.985);
\node[gp node center] at (6.188,0.677) { 0.06};
\draw[gp path] (6.908,1.165)--(6.908,1.075);
\draw[gp path] (7.627,1.165)--(7.627,0.985);
\node[gp node center] at (7.627,0.677) { 0.08};
\draw[gp path] (8.347,1.165)--(8.347,1.075);
\draw[gp path] (9.067,1.165)--(9.067,0.985);
\node[gp node center] at (9.067,0.677) { 0.1};
\draw[gp path] (9.787,1.165)--(9.787,1.075);
\draw[gp path] (10.507,1.165)--(10.507,0.985);
\node[gp node center] at (10.507,0.677) { 0.12};
\draw[gp path] (11.227,1.165)--(11.227,1.075);
\draw[gp path] (11.947,1.165)--(11.947,0.985);
\node[gp node center] at (11.947,0.677) { 0.14};
\draw[gp path] (1.868,8.381)--(1.868,1.165)--(11.947,1.165);
\node[gp node center,rotate=-270] at (0.246,4.773) {$\nicefrac{1}{q} [\cm^{-1}]$};
\node[gp node center] at (6.907,0.215) {$\nicefrac{1}{p} [\cm^{-1}]$};
\node[gp node right] at (10.479,8.047) {Campione originario};
\gpcolor{rgb color={0.000,0.471,0.000}}
\gpsetpointsize{4.00}
\gppoint{gp mark 1}{(6.145,5.321)}
\gppoint{gp mark 1}{(6.415,5.086)}
\gppoint{gp mark 1}{(6.722,4.970)}
\gppoint{gp mark 1}{(7.073,4.787)}
\gppoint{gp mark 1}{(7.479,4.469)}
\gppoint{gp mark 1}{(7.953,4.192)}
\gppoint{gp mark 1}{(8.515,3.841)}
\gppoint{gp mark 1}{(9.191,3.414)}
\gppoint{gp mark 1}{(10.020,2.951)}
\gppoint{gp mark 1}{(11.061,2.265)}
\gppoint{gp mark 1}{(5.166,5.937)}
\gppoint{gp mark 1}{(4.551,6.260)}
\gppoint{gp mark 1}{(3.823,6.766)}
\gppoint{gp mark 1}{(3.342,7.059)}
\gppoint{gp mark 1}{(3.032,7.217)}
\gppoint{gp mark 1}{(2.805,7.384)}
\gppoint{gp mark 1}{(11.121,8.047)}
\gpcolor{color=gp lt color border}
\node[gp node right] at (10.479,7.739) {Campione per difetto};
\gpcolor{rgb color={0.000,0.000,0.859}}
\gppoint{gp mark 2}{(6.133,5.321)}
\gppoint{gp mark 2}{(6.401,5.086)}
\gppoint{gp mark 2}{(6.706,4.970)}
\gppoint{gp mark 2}{(7.054,4.787)}
\gppoint{gp mark 2}{(7.457,4.469)}
\gppoint{gp mark 2}{(7.927,4.192)}
\gppoint{gp mark 2}{(8.484,3.841)}
\gppoint{gp mark 2}{(9.154,3.414)}
\gppoint{gp mark 2}{(9.974,2.951)}
\gppoint{gp mark 2}{(11.002,2.265)}
\gppoint{gp mark 2}{(5.158,5.937)}
\gppoint{gp mark 2}{(4.546,6.260)}
\gppoint{gp mark 2}{(3.820,6.766)}
\gppoint{gp mark 2}{(3.341,7.059)}
\gppoint{gp mark 2}{(3.031,7.217)}
\gppoint{gp mark 2}{(2.804,7.384)}
\gppoint{gp mark 2}{(11.121,7.739)}
\gpcolor{color=gp lt color border}
\node[gp node right] at (10.479,7.431) {Campione per eccesso};
\gpcolor{rgb color={0.976,0.443,0.004}}
\gppoint{gp mark 3}{(6.159,5.321)}
\gppoint{gp mark 3}{(6.432,5.086)}
\gppoint{gp mark 3}{(6.742,4.970)}
\gppoint{gp mark 3}{(7.095,4.787)}
\gppoint{gp mark 3}{(7.498,4.469)}
\gppoint{gp mark 3}{(7.980,4.192)}
\gppoint{gp mark 3}{(8.542,3.841)}
\gppoint{gp mark 3}{(9.226,3.414)}
\gppoint{gp mark 3}{(10.068,2.951)}
\gppoint{gp mark 3}{(11.119,2.265)}
\gppoint{gp mark 3}{(5.172,5.937)}
\gppoint{gp mark 3}{(4.553,6.260)}
\gppoint{gp mark 3}{(3.826,6.766)}
\gppoint{gp mark 3}{(3.344,7.059)}
\gppoint{gp mark 3}{(3.034,7.217)}
\gppoint{gp mark 3}{(2.804,7.384)}
\gppoint{gp mark 3}{(11.121,7.431)}
\gpcolor{color=gp lt color border}
\node[gp node right] at (10.479,7.123) {Fit};
\gpcolor{rgb color={0.412,0.765,0.000}}
\gpsetlinetype{gp lt plot 0}
\draw[gp path] (10.663,7.123)--(11.579,7.123);
\draw[gp path] (1.868,7.994)--(1.961,7.935)--(2.055,7.877)--(2.148,7.818)--(2.242,7.760)%
  --(2.335,7.701)--(2.429,7.643)--(2.522,7.584)--(2.616,7.526)--(2.709,7.467)--(2.802,7.409)%
  --(2.896,7.350)--(2.989,7.291)--(3.083,7.233)--(3.176,7.174)--(3.270,7.116)--(3.363,7.057)%
  --(3.457,6.999)--(3.550,6.940)--(3.643,6.882)--(3.737,6.823)--(3.830,6.765)--(3.924,6.706)%
  --(4.017,6.648)--(4.111,6.589)--(4.204,6.530)--(4.298,6.472)--(4.391,6.413)--(4.484,6.355)%
  --(4.578,6.296)--(4.671,6.238)--(4.765,6.179)--(4.858,6.121)--(4.952,6.062)--(5.045,6.004)%
  --(5.139,5.945)--(5.232,5.887)--(5.325,5.828)--(5.419,5.769)--(5.512,5.711)--(5.606,5.652)%
  --(5.699,5.594)--(5.793,5.535)--(5.886,5.477)--(5.980,5.418)--(6.073,5.360)--(6.166,5.301)%
  --(6.260,5.243)--(6.353,5.184)--(6.447,5.126)--(6.540,5.067)--(6.634,5.008)--(6.727,4.950)%
  --(6.821,4.891)--(6.914,4.833)--(7.007,4.774)--(7.101,4.716)--(7.194,4.657)--(7.288,4.599)%
  --(7.381,4.540)--(7.475,4.482)--(7.568,4.423)--(7.662,4.365)--(7.755,4.306)--(7.848,4.247)%
  --(7.942,4.189)--(8.035,4.130)--(8.129,4.072)--(8.222,4.013)--(8.316,3.955)--(8.409,3.896)%
  --(8.503,3.838)--(8.596,3.779)--(8.690,3.721)--(8.783,3.662)--(8.876,3.604)--(8.970,3.545)%
  --(9.063,3.486)--(9.157,3.428)--(9.250,3.369)--(9.344,3.311)--(9.437,3.252)--(9.531,3.194)%
  --(9.624,3.135)--(9.717,3.077)--(9.811,3.018)--(9.904,2.960)--(9.998,2.901)--(10.091,2.843)%
  --(10.185,2.784)--(10.278,2.725)--(10.372,2.667)--(10.465,2.608)--(10.558,2.550)--(10.652,2.491)%
  --(10.745,2.433)--(10.839,2.374)--(10.932,2.316)--(11.026,2.257)--(11.119,2.199);
\gpcolor{color=gp lt color border}
\node[gp node right] at (10.479,6.815) {Fit per difetto};
\gpcolor{rgb color={0.506,0.506,1.000}}
\gpsetlinetype{gp lt plot 4}
\draw[gp path] (10.663,6.815)--(11.579,6.815);
\draw[gp path] (1.868,7.982)--(1.961,7.924)--(2.055,7.865)--(2.148,7.807)--(2.242,7.748)%
  --(2.335,7.690)--(2.429,7.631)--(2.522,7.572)--(2.616,7.514)--(2.709,7.455)--(2.802,7.397)%
  --(2.896,7.338)--(2.989,7.280)--(3.083,7.221)--(3.176,7.163)--(3.270,7.104)--(3.363,7.046)%
  --(3.457,6.987)--(3.550,6.929)--(3.643,6.870)--(3.737,6.811)--(3.830,6.753)--(3.924,6.694)%
  --(4.017,6.636)--(4.111,6.577)--(4.204,6.519)--(4.298,6.460)--(4.391,6.402)--(4.484,6.343)%
  --(4.578,6.285)--(4.671,6.226)--(4.765,6.168)--(4.858,6.109)--(4.952,6.050)--(5.045,5.992)%
  --(5.139,5.933)--(5.232,5.875)--(5.325,5.816)--(5.419,5.758)--(5.512,5.699)--(5.606,5.641)%
  --(5.699,5.582)--(5.793,5.524)--(5.886,5.465)--(5.980,5.407)--(6.073,5.348)--(6.166,5.289)%
  --(6.260,5.231)--(6.353,5.172)--(6.447,5.114)--(6.540,5.055)--(6.634,4.997)--(6.727,4.938)%
  --(6.821,4.880)--(6.914,4.821)--(7.007,4.763)--(7.101,4.704)--(7.194,4.646)--(7.288,4.587)%
  --(7.381,4.528)--(7.475,4.470)--(7.568,4.411)--(7.662,4.353)--(7.755,4.294)--(7.848,4.236)%
  --(7.942,4.177)--(8.035,4.119)--(8.129,4.060)--(8.222,4.002)--(8.316,3.943)--(8.409,3.885)%
  --(8.503,3.826)--(8.596,3.767)--(8.690,3.709)--(8.783,3.650)--(8.876,3.592)--(8.970,3.533)%
  --(9.063,3.475)--(9.157,3.416)--(9.250,3.358)--(9.344,3.299)--(9.437,3.241)--(9.531,3.182)%
  --(9.624,3.124)--(9.717,3.065)--(9.811,3.006)--(9.904,2.948)--(9.998,2.889)--(10.091,2.831)%
  --(10.185,2.772)--(10.278,2.714)--(10.372,2.655)--(10.465,2.597)--(10.558,2.538)--(10.652,2.480)%
  --(10.745,2.421)--(10.839,2.363)--(10.932,2.304)--(11.026,2.245)--(11.119,2.187);
\gpcolor{color=gp lt color border}
\node[gp node right] at (10.479,6.507) {Fit per eccesso};
\gpcolor{rgb color={1.000,1.000,0.004}}
\gpsetlinetype{gp lt plot 5}
\draw[gp path] (10.663,6.507)--(11.579,6.507);
\draw[gp path] (1.868,8.006)--(1.961,7.947)--(2.055,7.889)--(2.148,7.830)--(2.242,7.772)%
  --(2.335,7.713)--(2.429,7.655)--(2.522,7.596)--(2.616,7.538)--(2.709,7.479)--(2.802,7.420)%
  --(2.896,7.362)--(2.989,7.303)--(3.083,7.245)--(3.176,7.186)--(3.270,7.128)--(3.363,7.069)%
  --(3.457,7.011)--(3.550,6.952)--(3.643,6.894)--(3.737,6.835)--(3.830,6.777)--(3.924,6.718)%
  --(4.017,6.659)--(4.111,6.601)--(4.204,6.542)--(4.298,6.484)--(4.391,6.425)--(4.484,6.367)%
  --(4.578,6.308)--(4.671,6.250)--(4.765,6.191)--(4.858,6.133)--(4.952,6.074)--(5.045,6.016)%
  --(5.139,5.957)--(5.232,5.898)--(5.325,5.840)--(5.419,5.781)--(5.512,5.723)--(5.606,5.664)%
  --(5.699,5.606)--(5.793,5.547)--(5.886,5.489)--(5.980,5.430)--(6.073,5.372)--(6.166,5.313)%
  --(6.260,5.255)--(6.353,5.196)--(6.447,5.137)--(6.540,5.079)--(6.634,5.020)--(6.727,4.962)%
  --(6.821,4.903)--(6.914,4.845)--(7.007,4.786)--(7.101,4.728)--(7.194,4.669)--(7.288,4.611)%
  --(7.381,4.552)--(7.475,4.493)--(7.568,4.435)--(7.662,4.376)--(7.755,4.318)--(7.848,4.259)%
  --(7.942,4.201)--(8.035,4.142)--(8.129,4.084)--(8.222,4.025)--(8.316,3.967)--(8.409,3.908)%
  --(8.503,3.850)--(8.596,3.791)--(8.690,3.732)--(8.783,3.674)--(8.876,3.615)--(8.970,3.557)%
  --(9.063,3.498)--(9.157,3.440)--(9.250,3.381)--(9.344,3.323)--(9.437,3.264)--(9.531,3.206)%
  --(9.624,3.147)--(9.717,3.089)--(9.811,3.030)--(9.904,2.971)--(9.998,2.913)--(10.091,2.854)%
  --(10.185,2.796)--(10.278,2.737)--(10.372,2.679)--(10.465,2.620)--(10.558,2.562)--(10.652,2.503)%
  --(10.745,2.445)--(10.839,2.386)--(10.932,2.328)--(11.026,2.269)--(11.119,2.210);
\gpcolor{color=gp lt color border}
\gpsetlinetype{gp lt border}
\draw[gp path] (1.868,8.381)--(1.868,1.165)--(11.947,1.165);
%% coordinates of the plot area
\gpdefrectangularnode{gp plot 1}{\pgfpoint{1.868cm}{1.165cm}}{\pgfpoint{11.947cm}{8.381cm}}
\end{tikzpicture}
%% gnuplot variables
 \caption{Errori su $P_O$} \label{fig:02_graph_4.tex} \end{grafico}
\begin{tabella}
	\centering
	\begin{tabulary}{\textwidth}{LCC}
\toprule
Campione & Intercetta $[\cm^{-1}]$  & $\sigma [\cm^{-1}]$\\ \midrule
Campione originario & 0.1514 & 0.0002\\ \midrule
Con $p_{o}'=(p_o+0.5)\mm$ & 0.1517 & 0.0008\\ \midrule
Con $p_{o}'=(p_o-0.5)\mm$ & 0.1512 & 0.0001\\
\bottomrule
\end{tabulary}

	\caption{Rette interpolanti errori sistematici}
	\label{tab:02tab6}
\end{tabella}
Da un confronto delle rette interpolanti nella \autoref{tab:02tab6} si può stimare l'incertezza che
 può essere collegata all'imprecisione nella lettura della
 posizione del cavaliere portalampada. In particolare si può leggere l'effettivo valore del fuoco come quello collegato al
 campione realmente raccolto, e ad esso si può associare un errore sistematico legato alla massima distanza tra i fuochi fittizi
 ricavati dalle interpolazioni dei campioni fittizi, che risulta di 0.01~cm. Il valore finale risulta, quindi:
\[f^{\star}_2 = (6.60 \pm 0.01) \cm.\]

Dove è stata considerata la sommma tra gli errori statistici e gli errori sistematici su $P_O$, rispetto ai quali gli
 errori statistici precedentemente considerati risultano trascurabili.
 Il valore più giusto tra i due, alla luce dell'$\mathcal{F}$-test effettuato e delle considerazioni sugli errori sistematici, è il secondo:
 infatti la semplicità della retta interpolante a signolo parametro libero permette un più semplice studio degli errori collegati
 al posizionamento dell'oggetto. Un approccio simile per la ricerca dell'errore sistematico nel caso del doppio parametro libero
 renderebbe impossibile l'approssimazione dell'errore attraverso le formule sopra citate, rendendo difficoltosa una pratica stima
 dello stesso. Il valore ottenuto va comunque corretto tramite i coefficienti di aberrazione sferica, come verrà discusso più avanti.
