L'ultimo metodo per la misura della lunghezza focale è il cosiddetto 
metodo di Bessel: questo metodo, usando una differenza delle 
posizioni, consente di ridurre l'errore sistematico dovuto alle 
imprecisioni nel posizionamento dei cavalieri. Il metodo consiste 
nel fissare lo schermo e l'oggetto, e muovere la lente fino a 
trovare i fuochi. 

Si troveranno due posizioni della lente, come spiegato sotto. Sia 
$L$ la distanza tra l'oggetto e lo schermo (sarà quindi fissata, e 
dovrà essere $L \geq 4f$, e $S=p_2-p_1$ la differenza di posizioni $p_1$ e $p_2$ della 
lente dove l'immagine sia a fuoco).  Dalla formula \eqref{eq:punticoniugati} si ricava, 
sostituendo $L$ e $S$,
\begin{equation}
f = \frac{L^2-S^2}{4L}
\end{equation}
e per trovare l'errore $\sigma_f$ deriviamo
\begin{equation*} \label{eq:dfdl}
\frac{\partial f}{\partial L} = \frac{L^2+S^2}{4L^2} \approx \frac{1}{4}
\end{equation*}
\begin{equation*}
\frac{\partial f}{\partial S} = -\frac{S}{2L}
\end{equation*}

Questi errori sono minimizzati per $L \approx 4f$, d'altra parte la 
profondità di campo rende difficile una stima precisa della 
posizione di coincidenza. Si \`e scelto quindi di usare valori di $L$ di poco 
superiori a $4f$ (calcolato con le stime precedenti). 
Ricordando che
\begin{align}
	 P_O &= (13.15 \pm 0.05)\cm \\
	 P_S &= (55.00 \pm 0.05)\cm \\
	 %\overline{VV'} &= 0.8 \cm \\
	 %\overline{PP'} &\approx \overline{VV'}/3 = 0.267 \cm \\
	 L &= P_S -P_0 -\overline{PP'}
\end{align}
si sono trovate una decina di misure di $p_1$ e 
$p_2$, e si \`e calcolato per ciascuna il fuoco col suo errore 
(tutto ciò è illustrato nella \autoref{tab:bessel}) effettuando 
infine una media pesata per ottenere la stima del fuoco col metodo 
di Bessel $f^{\star}_3$ e il suo errore $\sigma_{f^{\star}_3}$:
\begin{equation}
f^{\star}_3 = (6.588 \pm 0.007) \cm
\end{equation}
\begin{tabella}
	\centering
	\begin{tabulary}{\textwidth}{CCCCC}
\toprule
$p_1$ & $p_2$ & $S$ & $f$ & $\sigma_f$\\ \midrule
21.35 & 46.55 & 25.20 & 6.58 & 0.27 \\ \midrule
21.25 & 46.45 & 25.20 & 6.58 & 0.27 \\ \midrule
21.35 & 46.60 & 25.25 & 6.56 & 0.28 \\ \midrule
21.30 & 46.45 & 25.15 & 6.59 & 0.27 \\ \midrule
21.35 & 46.45 & 25.10 & 6.61 & 0.27 \\ \midrule
21.35 & 46.45 & 25.10 & 6.61 & 0.27 \\ \midrule
21.35 & 46.50 & 25.15 & 6.59 & 0.27 \\ \midrule
21.30 & 46.45 & 25.15 & 6.59 & 0.27 \\ \midrule
21.35 & 46.55 & 25.20 & 6.58 & 0.27 \\ \midrule
21.30 & 46.45 & 25.15 & 6.59 & 0.27 \\
\bottomrule
\end{tabulary}

	\caption{Metodo di Bessel $[\cm\,]$}
	\label{tab:bessel}
\end{tabella}
% 03_tab_1.tex
