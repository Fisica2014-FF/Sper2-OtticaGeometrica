%**01_tab_1.tx**

Per la stima della posizione del fuoco e il calcolo degli errori, si considera la seguente formula:
\[ f = P_L - P_0 + (\mu _0 - \mu ^{\star}) + \frac{\mathrm{d}r}{2} - \frac{PP'}{2}.\]
\( \mu ^{\star} \) \`e individuato dall'intersezione delle due rette interpolanti precedentemente calcolate:
\[ \mu ^{\star} = x_{intersezione} = \frac{a - a'}{b' - b} = F(a, a', b, b'), \]
da cui, grazie alla formula di propagazione degli errori, si ottiene, considerando $(a, b)$, $(a', b')$ rispettivamente correlati e le rette tra loro indipendenti,
\[ Var[F(a, a', b, b')] = (\frac{\partial F}{\partial a}|_{x_{int}})^2 \cdot Var(a) + 
(\frac{\partial F}{\partial a'}|_{x_{int}})^2 \cdot Var(a') + \hfill
(\frac{\partial F}{\partial b}|_{x_{int}})^2 \cdot Var(b) + 
(\frac{\partial F}{\partial b'}|_{x_{int}})^2 \cdot Var(b) + \hfill
2(\frac{\partial F}{\partial a}|_{x_{int}})(\frac{\partial F}{\partial b}|_{x_{int}}) \cdot Cov(a, b) +
2(\frac{\partial F}{\partial a'}|_{x_{int}})(\frac{\partial F}{\partial b'}|_{x_{int}})^2 \cdot Cov (a', b'), \]
che, sotto radice quadrata, d\`a l'incertezza per \( \mu ^{\star} \), considerandolo distribuito normalmente.
Svolgendo i calcoli, si trova
\[ Var[F(a, a', b, b')] = (\frac{1}{b'- b}|_{x_{int}})^2 \cdot (Var(a) + Var(a')) + (\frac{a - a'}{(b' - b)^2}|_{x_{int}})^2 (Var(b) + Var(b')) + 
2 (\frac{1}{b'- b}|_{x_{int}}) (\frac{a - a'}{(b' - b)^2}|_{x_{int}}) (Cov(a, b) + Cov(a', b') . \]
Calcoliamo le covarianze:
\[ Cov(a, b) = -\frac{\sum_{i} x_i}{\Delta}\sigma_y^2, \]
dove $\Delta$ \`e il parametro di interpolazione lineare; vale lo stesso per a' e b', con le adeguate (x, y).

Si trova per il fuoco il valore 
\[ f=(7.14437 \pm 0.7) \cm ; \] 
si \`e considerata trascurabile l'incertezza su $\mu*$ rispetto a quello fornito dal laboratorio su $P_L$ e $P_0$, pari a $\sigma _P = 0.05 \cm$: ci\`o porta propagando a un'incertezza di $\sqrt{2} \cdot \sigma_P = 0.07 \cm$. Tale stima va considerata a meno della correzione di aberrazione sferica, che verr\`a compiuta sulla miglior stima del valore di \( f \).

%1)Io mi dilungherei personalmente un pochino di più sulla questione dell'errore, ho capito che è calato dall'alto, però ne parlerei con un po' più precisione: intendo, io direi qualcosa del tipo "L'incertezza è presa di 0.7 perché a causa dei ritardi delle componenti meccaniche e degli errori di lettura è stato precedentemente verificato che il valore della posizione di ogni cavaliere ha un'incertezza di 0.05 cm -tra l'altro Gabriele, c'è uno zero che mi manca qui-, per cui l'incertezza -Non usare errore, nella prima lezione che abbiamo fatto con lui ha fatto un discorso su come gli erorri siano una cosa, mentre quello che stima un fisico è un'incertezza- è data dalla somma quadratica di tutte le incertezze presenti nella formula, ma da un'occhiata rapida risulta evidente come tutte le incertezze sono effettivamente trascurabili rispetto a quelle su $P_L$ e $P_o$, per cui l'incertezza si riduce a $\radq{2} \cdot \sigma_{posizionamento} = 0.07 \cm$". Tra l'altro tu ti sei calcolato mustar e hai suato muzero senza scriverli da nessuna parte. almeno una righetta dopo il grafico con scritto il valore ce la metterei sinceramente---Davide

