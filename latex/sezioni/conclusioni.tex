Le grandezze fisiche della lente cercate possono essere presentate in una lista che ne contenga i valori finali:
\begin{itemize}
\item $f^*_1 = (6.76 \pm 0.08) \cm$
\item $f^*_2 = (6.65 \pm 0.05) \cm$
\item $f^*_3 = (6.63 \pm 0.05) \cm$
\item $f = (6.66 \pm 0.03) \cm$
\item $c = (1.7 \pm 0.1)$
\item $t = (0.194 \pm 0.002) \cm$
\item $A = (0.113 \pm 0.005) \cm$
\item $V = (56 \pm 1)$  
\end{itemize}
(dove il fuoco \`e inteso nel giallo).

Per il calcolo delle grandezze fisiche non dirette (per esempio per il calcolo della media del fuoco o per il calcolo della costante di aberrazione sferica $c$) si \`e operato con i valori ottenuti nell'analisi dati non approssimati, l'approssimazione \`e stata fatta solo in fase di presentazione dei dati stessi.

Le stime dei fuochi sono sufficientemente compatibili tra loro, con la deviazione massima della prima stima, giustificabile con la maggiore incertezza legata all'esperienza: la maggior parte dell'incertezza della seconda e terza stima \`e dovuta alla correzione per aberrazione sferica (vedi \autoref{subsec:aberrazione_sferica}).

Le stime dei coefficienti di aberrazione sono in linea con le aspettative teoriche.
