%**01_tab_1.tx**


I risultati delle misure sono nella \autoref{tab:01_tab_1.tex}.
\begin{tabella}
	\centering
	\begin{center}
\begin{tabulary}{\textwidth}{CCC}
\toprule
Posizione relativa della lente & Diametro fascio a $30 \cm$ & Diametro fascio a $130 \cm$ \\ \midrule
0.400 & 1.187 & 1.000 \\ \midrule
0.450 & 1.193 & 1.105 \\ \midrule
0.460 & 1.171 & 1.172 \\ \midrule
0.475 & 1.168 & 1.190 \\ \midrule
0.500 & 1.186 & 1.254 \\ \midrule
0.550 & 1.190 & 1.367 \\
\bottomrule
\end{tabulary}
\end{center}

	\caption{Risultati autocollimazione $[\cm\,]$}
	\label{tab:01_tab_1.tex}
\end{tabella}
Per la stima della posizione del fuoco e il calcolo dell'incertezza, si considera la seguente formula:
\begin{equation} \label{eq:autocollimazione}
f = P_L - P_O + \left(\mu _0 - \mu ^{\star}\right) + \frac{\mathrm{d}r}{2} - \frac{\overline{PP'}}{2}.
\end{equation}
Per la stima di $ \mu ^{\star} $, valore del micrometro per cui il fascio \`e parallelo, si sono interpolati i diametri del fascio in ordinata con i valori indicati dal micrometro: $ \mu ^{\star}$ \`e individuato dall'ascissa dell'intersezione delle due rette interpolanti ($y=ax+b$ per la prima, con notazione simile per la seconda) nel \autoref{fig:01_graph_1.tex}, i cui parametri sono nella \autoref{tab:01tab2}.
\begin{tabella}
	\centering
	\begin{tabulary}{\textwidth}{CCCCC}
\toprule
Da vicino & $a$ & $\sigma_a$ & $b [\cm]$ & $\sigma_b [\cm]$ \\ \cmidrule{2-5}
 & 11.7 & 0.5 & 0.2 & 1.0\\ \midrule
Da lontano & $a'$ & $\sigma_{a'}$ & $b' [\cm]$ & $\sigma_{b'} [\cm]$ \\ \cmidrule{2-5}
 & 0.1 & 0.7 & 25 & 1  \\
\bottomrule
\end{tabulary}

	\caption{Coefficienti delle rette interpolanti}
	\label{tab:01tab2}
\end{tabella}

\begin{grafico} \centering \begin{tikzpicture}[gnuplot]
%% generated with GNUPLOT 4.6p5 (Lua 5.1; terminal rev. 99, script rev. 100)
%% mar 23 dic 2014 16:48:59 CET
\path (0.000,0.000) rectangle (12.500,8.750);
\gpcolor{color=gp lt color border}
\gpsetlinetype{gp lt border}
\gpsetlinewidth{1.00}
\draw[gp path] (1.688,1.807)--(1.868,1.807);
\node[gp node right] at (1.504,1.807) { 10};
\draw[gp path] (1.688,2.218)--(1.778,2.218);
\draw[gp path] (1.688,2.629)--(1.868,2.629);
\node[gp node right] at (1.504,2.629) { 10.5};
\draw[gp path] (1.688,3.039)--(1.778,3.039);
\draw[gp path] (1.688,3.450)--(1.868,3.450);
\node[gp node right] at (1.504,3.450) { 11};
\draw[gp path] (1.688,3.861)--(1.778,3.861);
\draw[gp path] (1.688,4.272)--(1.868,4.272);
\node[gp node right] at (1.504,4.272) { 11.5};
\draw[gp path] (1.688,4.683)--(1.778,4.683);
\draw[gp path] (1.688,5.094)--(1.868,5.094);
\node[gp node right] at (1.504,5.094) { 12};
\draw[gp path] (1.688,5.505)--(1.778,5.505);
\draw[gp path] (1.688,5.916)--(1.868,5.916);
\node[gp node right] at (1.504,5.916) { 12.5};
\draw[gp path] (1.688,6.327)--(1.778,6.327);
\draw[gp path] (1.688,6.737)--(1.868,6.737);
\node[gp node right] at (1.504,6.737) { 13};
\draw[gp path] (1.688,7.148)--(1.778,7.148);
\draw[gp path] (1.688,7.559)--(1.868,7.559);
\node[gp node right] at (1.504,7.559) { 13.5};
\draw[gp path] (1.688,7.970)--(1.778,7.970);
\draw[gp path] (2.714,0.985)--(2.714,1.165);
\node[gp node center] at (2.714,0.677) { 0.4};
\draw[gp path] (3.227,0.985)--(3.227,1.075);
\draw[gp path] (3.740,0.985)--(3.740,1.165);
\node[gp node center] at (3.740,0.677) { 0.42};
\draw[gp path] (4.253,0.985)--(4.253,1.075);
\draw[gp path] (4.766,0.985)--(4.766,1.165);
\node[gp node center] at (4.766,0.677) { 0.44};
\draw[gp path] (5.279,0.985)--(5.279,1.075);
\draw[gp path] (5.792,0.985)--(5.792,1.165);
\node[gp node center] at (5.792,0.677) { 0.46};
\draw[gp path] (6.305,0.985)--(6.305,1.075);
\draw[gp path] (6.818,0.985)--(6.818,1.165);
\node[gp node center] at (6.818,0.677) { 0.48};
\draw[gp path] (7.330,0.985)--(7.330,1.075);
\draw[gp path] (7.843,0.985)--(7.843,1.165);
\node[gp node center] at (7.843,0.677) { 0.5};
\draw[gp path] (8.356,0.985)--(8.356,1.075);
\draw[gp path] (8.869,0.985)--(8.869,1.165);
\node[gp node center] at (8.869,0.677) { 0.52};
\draw[gp path] (9.382,0.985)--(9.382,1.075);
\draw[gp path] (9.895,0.985)--(9.895,1.165);
\node[gp node center] at (9.895,0.677) { 0.54};
\draw[gp path] (10.408,0.985)--(10.408,1.075);
\draw[gp path] (1.688,7.938)--(1.688,1.807);
\draw[gp path] (2.714,0.985)--(10.408,0.985);
\node[gp node center,rotate=-270] at (0.246,4.683) {Diametro $[\mm\,]$};
\node[gp node center] at (6.817,0.215) {$\mu [\cm\,]$};
\node[gp node right] at (10.479,8.047) {Dati da vicino};
\gpcolor{rgb color={0.698,0.000,0.000}}
\gpsetpointsize{4.00}
\gppoint{gp mark 1}{(2.714,4.880)}
\gppoint{gp mark 1}{(5.279,4.979)}
\gppoint{gp mark 1}{(5.792,4.617)}
\gppoint{gp mark 1}{(6.561,4.568)}
\gppoint{gp mark 1}{(7.843,4.864)}
\gppoint{gp mark 1}{(10.408,4.930)}
\gppoint{gp mark 1}{(11.121,8.047)}
\gpcolor{color=gp lt color border}
\node[gp node right] at (10.479,7.739) {Interpolazione da vicino};
\gpcolor{rgb color={1.000,0.000,0.000}}
\gpsetlinetype{gp lt plot 0}
\draw[gp path] (10.663,7.739)--(11.579,7.739);
\draw[gp path] (2.714,4.785)--(2.792,4.786)--(2.869,4.786)--(2.947,4.787)--(3.025,4.787)%
  --(3.102,4.787)--(3.180,4.788)--(3.258,4.788)--(3.336,4.789)--(3.413,4.789)--(3.491,4.790)%
  --(3.569,4.790)--(3.647,4.791)--(3.724,4.791)--(3.802,4.791)--(3.880,4.792)--(3.957,4.792)%
  --(4.035,4.793)--(4.113,4.793)--(4.191,4.794)--(4.268,4.794)--(4.346,4.794)--(4.424,4.795)%
  --(4.501,4.795)--(4.579,4.796)--(4.657,4.796)--(4.735,4.797)--(4.812,4.797)--(4.890,4.798)%
  --(4.968,4.798)--(5.045,4.798)--(5.123,4.799)--(5.201,4.799)--(5.279,4.800)--(5.356,4.800)%
  --(5.434,4.801)--(5.512,4.801)--(5.590,4.801)--(5.667,4.802)--(5.745,4.802)--(5.823,4.803)%
  --(5.900,4.803)--(5.978,4.804)--(6.056,4.804)--(6.134,4.805)--(6.211,4.805)--(6.289,4.805)%
  --(6.367,4.806)--(6.444,4.806)--(6.522,4.807)--(6.600,4.807)--(6.678,4.808)--(6.755,4.808)%
  --(6.833,4.808)--(6.911,4.809)--(6.988,4.809)--(7.066,4.810)--(7.144,4.810)--(7.222,4.811)%
  --(7.299,4.811)--(7.377,4.812)--(7.455,4.812)--(7.533,4.812)--(7.610,4.813)--(7.688,4.813)%
  --(7.766,4.814)--(7.843,4.814)--(7.921,4.815)--(7.999,4.815)--(8.077,4.815)--(8.154,4.816)%
  --(8.232,4.816)--(8.310,4.817)--(8.387,4.817)--(8.465,4.818)--(8.543,4.818)--(8.621,4.818)%
  --(8.698,4.819)--(8.776,4.819)--(8.854,4.820)--(8.931,4.820)--(9.009,4.821)--(9.087,4.821)%
  --(9.165,4.822)--(9.242,4.822)--(9.320,4.822)--(9.398,4.823)--(9.476,4.823)--(9.553,4.824)%
  --(9.631,4.824)--(9.709,4.825)--(9.786,4.825)--(9.864,4.825)--(9.942,4.826)--(10.020,4.826)%
  --(10.097,4.827)--(10.175,4.827)--(10.253,4.828)--(10.330,4.828)--(10.408,4.829);
\gpcolor{color=gp lt color border}
\node[gp node right] at (10.479,7.431) {Dati da lontano};
\gpcolor{rgb color={0.000,0.000,0.800}}
\gppoint{gp mark 3}{(2.714,1.807)}
\gppoint{gp mark 3}{(5.279,3.533)}
\gppoint{gp mark 3}{(5.792,4.634)}
\gppoint{gp mark 3}{(6.561,4.930)}
\gppoint{gp mark 3}{(7.843,5.981)}
\gppoint{gp mark 3}{(10.408,7.839)}
\gppoint{gp mark 3}{(11.121,7.431)}
\gpcolor{color=gp lt color border}
\node[gp node right] at (10.479,7.123) {Interpolazione da lontano};
\gpcolor{rgb color={0.102,0.000,1.000}}
\draw[gp path] (10.663,7.123)--(11.579,7.123);
\draw[gp path] (2.714,1.839)--(2.792,1.901)--(2.869,1.962)--(2.947,2.024)--(3.025,2.085)%
  --(3.102,2.147)--(3.180,2.209)--(3.258,2.270)--(3.336,2.332)--(3.413,2.394)--(3.491,2.455)%
  --(3.569,2.517)--(3.647,2.578)--(3.724,2.640)--(3.802,2.702)--(3.880,2.763)--(3.957,2.825)%
  --(4.035,2.886)--(4.113,2.948)--(4.191,3.010)--(4.268,3.071)--(4.346,3.133)--(4.424,3.194)%
  --(4.501,3.256)--(4.579,3.318)--(4.657,3.379)--(4.735,3.441)--(4.812,3.503)--(4.890,3.564)%
  --(4.968,3.626)--(5.045,3.687)--(5.123,3.749)--(5.201,3.811)--(5.279,3.872)--(5.356,3.934)%
  --(5.434,3.995)--(5.512,4.057)--(5.590,4.119)--(5.667,4.180)--(5.745,4.242)--(5.823,4.303)%
  --(5.900,4.365)--(5.978,4.427)--(6.056,4.488)--(6.134,4.550)--(6.211,4.612)--(6.289,4.673)%
  --(6.367,4.735)--(6.444,4.796)--(6.522,4.858)--(6.600,4.920)--(6.678,4.981)--(6.755,5.043)%
  --(6.833,5.104)--(6.911,5.166)--(6.988,5.228)--(7.066,5.289)--(7.144,5.351)--(7.222,5.412)%
  --(7.299,5.474)--(7.377,5.536)--(7.455,5.597)--(7.533,5.659)--(7.610,5.721)--(7.688,5.782)%
  --(7.766,5.844)--(7.843,5.905)--(7.921,5.967)--(7.999,6.029)--(8.077,6.090)--(8.154,6.152)%
  --(8.232,6.213)--(8.310,6.275)--(8.387,6.337)--(8.465,6.398)--(8.543,6.460)--(8.621,6.521)%
  --(8.698,6.583)--(8.776,6.645)--(8.854,6.706)--(8.931,6.768)--(9.009,6.829)--(9.087,6.891)%
  --(9.165,6.953)--(9.242,7.014)--(9.320,7.076)--(9.398,7.138)--(9.476,7.199)--(9.553,7.261)%
  --(9.631,7.322)--(9.709,7.384)--(9.786,7.446)--(9.864,7.507)--(9.942,7.569)--(10.020,7.630)%
  --(10.097,7.692)--(10.175,7.754)--(10.253,7.815)--(10.330,7.877)--(10.408,7.938);
\gpcolor{color=gp lt color border}
\gpsetlinetype{gp lt border}
\draw[gp path] (1.688,7.938)--(1.688,1.807);
\draw[gp path] (2.714,0.985)--(10.408,0.985);
%% coordinates of the plot area
\gpdefrectangularnode{gp plot 1}{\pgfpoint{1.688cm}{0.985cm}}{\pgfpoint{11.947cm}{8.381cm}}
\end{tikzpicture}
%% gnuplot variables
 \caption{Interpolazione lineare} \label{fig:01_graph_1.tex} \end{grafico}

\[ \mu ^{\star} = x_{\mathrm{intersezione}} = \frac{b - b'}{a' - a}  = 4.73 \mm , \]
da cui, dall'equazione \eqref{eq:autocollimazione} grazie alla formula di propagazione quadratica, si ottiene, considerando $\left(a, b\right)$, $\left(a', b'\right)$ rispettivamente correlati e le rette tra loro indipendenti,
% split va dentro a un'equazione
\begin{equation*}
\begin{split} % align numera tutte le righe e se devo toglierne una devo usare \nonumber alla fine della riga, align* nessuna. Meglio split che numera solo una volta
	\sigma^2_{\mu ^{\star}\left(a, a', b, b'\right)}  &= \left(\left.\frac{\partial F}{\partial a}\right|_{x_{\mathrm{int}}}\right)^2   \sigma^2_{a} + \left(\left.\frac{\partial F}{\partial a'}\right|_{x_{\mathrm{int}}}\right)^2   \sigma^2_{a'} + \left(\left.\frac{\partial F}{\partial b}\right|_{x_{\mathrm{int}}}\right)^2   \sigma^2_{b} +\\
								&+ \left(\left.\frac{\partial F}{\partial b'}\right|_{x_{\mathrm{int}}}\right)^2   \sigma^2_{b'} + 2\left(\left.\frac{\partial F}{\partial a}\right|_{x_{\mathrm{int}}}\right)\left(\left.\frac{\partial F}{\partial b}\right|_{x_{\mathrm{int}}}\right)   \cov\left(a, b\right) + \\
								&+ 2\left(\left.\frac{\partial F}{\partial a'}\right|_{x_{\mathrm{int}}}\right)\left(\left.\frac{\partial F}{\partial b'}\right|_{x_{\mathrm{int}}}\right)^2   \cov \left(a', b'\right)
\end{split}
\end{equation*}
che, sotto radice quadrata, d\`a l'incertezza per $ \mu ^{\star} $, considerandolo distribuito normalmente.
Svolgendo i calcoli, si trova
\begin{equation}
\begin{split}
\sigma^2_{\mu ^{\star}}  &=  \left(\left.\frac{b - b'}{\left(a' - a\right)^2}\right|_{x_{\mathrm{int}}}\right)^2 \left(\sigma^2_{a} + \sigma^2_{a'}\right) + \left(\left.\frac{1}{a'- a}\right|_{x_{\mathrm{int}}}\right)^2   \left(\sigma^2_{b} + \sigma^2_{b'}\right) \\
						 &+ 2 \left(\left.\frac{1}{a'- a}\right|_{x_{\mathrm{int}}}\right) \left(\left.\frac{b - b'}{\left(a' - a\right)^2}\right|_{x_{\mathrm{int}}}\right) \big(\cov\left(a, b\right) + \cov(a', b') \big).
\end{split}
\end{equation}
Calcoliamo le covarianze:
\[ \cov\left(a, b\right) = -\frac{\sum_{i} x_i}{\Delta}\sigma_y^2 \] 
dove $\Delta$ \`e il parametro di interpolazione lineare\footnote{M. Loreti, \textit{Teoria degli Errori e Fondamenti di Statistica}, p. 266}; vale lo stesso per a' e b', con le adeguate (x, y).
\[\cov(a, b) = -0.501 \cm^2 \] 
\[\cov(a', b') = -0.895 \cm^2. \]
L'incertezza cos\`i calcolata risulta di $0.0003\cm$, molto bassa a causa della precisione micrometrica, migliorata grazie al fit lineare. Tuttavia si ritiene pi\`u corretto considerarla non pi\`u bassa dell'incertezza strumentale. L'intersezione \`e quindi stimata come
\[ \mu ^{\star} =  \left(0.473 \pm 0.001\right) \cm. \] 

Per rendere visivamente apprezzabile l'incertezza sulle rette, \`e stato creato un grafico (\autoref{fig:01_graph_2.tex}) contenente le rette tracciate per i valori estremali della quota e del coefficiente angolare.
\begin{grafico} \centering \begin{tikzpicture}[gnuplot]
%% generated with GNUPLOT 4.6p4 (Lua 5.1; terminal rev. 99, script rev. 100)
%% mer 24 dic 2014 14:30:09 CET
\path (0.000,0.000) rectangle (12.500,8.750);
\gpcolor{color=gp lt color border}
\gpsetlinetype{gp lt border}
\gpsetlinewidth{1.00}
\draw[gp path] (4.235,0.970)--(4.055,0.970);
\draw[gp path] (4.235,0.970)--(4.415,0.970);
\node[gp node right] at (4.051,0.970) {-5};
\draw[gp path] (4.235,1.971)--(4.055,1.971);
\draw[gp path] (4.235,1.971)--(4.415,1.971);
\node[gp node right] at (4.051,1.971) { 0};
\draw[gp path] (4.235,2.973)--(4.055,2.973);
\draw[gp path] (4.235,2.973)--(4.415,2.973);
\node[gp node right] at (4.051,2.973) { 5};
\draw[gp path] (4.235,3.974)--(4.055,3.974);
\draw[gp path] (4.235,3.974)--(4.415,3.974);
\node[gp node right] at (4.051,3.974) { 10};
\draw[gp path] (4.235,4.976)--(4.055,4.976);
\draw[gp path] (4.235,4.976)--(4.415,4.976);
\node[gp node right] at (4.051,4.976) { 15};
\draw[gp path] (4.235,5.977)--(4.055,5.977);
\draw[gp path] (4.235,5.977)--(4.415,5.977);
\node[gp node right] at (4.051,5.977) { 20};
\draw[gp path] (4.235,6.979)--(4.055,6.979);
\draw[gp path] (4.235,6.979)--(4.415,6.979);
\node[gp node right] at (4.051,6.979) { 25};
\draw[gp path] (4.235,7.980)--(4.055,7.980);
\draw[gp path] (4.235,7.980)--(4.415,7.980);
\node[gp node right] at (4.051,7.980) { 30};
\draw[gp path] (2.173,1.971)--(2.173,1.791);
\draw[gp path] (2.173,1.971)--(2.173,2.151);
\node[gp node center] at (2.173,1.483) {-0.5};
\draw[gp path] (4.235,1.971)--(4.235,1.791);
\draw[gp path] (4.235,1.971)--(4.235,2.151);
\node[gp node center] at (4.235,1.483) { 0};
\draw[gp path] (6.297,1.971)--(6.297,1.791);
\draw[gp path] (6.297,1.971)--(6.297,2.151);
\node[gp node center] at (6.297,1.483) { 0.5};
\draw[gp path] (8.359,1.971)--(8.359,1.791);
\draw[gp path] (8.359,1.971)--(8.359,2.151);
\node[gp node center] at (8.359,1.483) { 1};
\draw[gp path] (10.421,1.971)--(10.421,1.791);
\draw[gp path] (10.421,1.971)--(10.421,2.151);
\node[gp node center] at (10.421,1.483) { 1.5};
\gpcolor{color=gp lt color axes}
\gpsetlinetype{gp lt axes}
\draw[gp path] (0.400,1.971)--(11.947,1.971);
\draw[gp path] (4.235,0.369)--(4.235,8.381);
\gpcolor{color=gp lt color border}
\node[gp node center,rotate=-270] at (0.246,4.375) {Diametro $[\mm\,]$};
\node[gp node center] at (6.173,0.215) {$\mu [\cm\,]$};
\node[gp node right] at (10.479,1.011) {Interpolazione da vicino};
\gpcolor{rgb color={1.000,0.000,0.000}}
\gpsetlinetype{gp lt plot 0}
\draw[gp path] (10.663,1.011)--(11.579,1.011);
\draw[gp path] (0.400,4.291)--(0.517,4.292)--(0.633,4.293)--(0.750,4.294)--(0.867,4.295)%
  --(0.983,4.296)--(1.100,4.297)--(1.216,4.298)--(1.333,4.299)--(1.450,4.300)--(1.566,4.301)%
  --(1.683,4.302)--(1.800,4.303)--(1.916,4.304)--(2.033,4.305)--(2.150,4.306)--(2.266,4.307)%
  --(2.383,4.308)--(2.499,4.309)--(2.616,4.310)--(2.733,4.311)--(2.849,4.312)--(2.966,4.313)%
  --(3.083,4.314)--(3.199,4.315)--(3.316,4.316)--(3.433,4.317)--(3.549,4.317)--(3.666,4.318)%
  --(3.782,4.319)--(3.899,4.320)--(4.016,4.321)--(4.132,4.322)--(4.249,4.323)--(4.366,4.324)%
  --(4.482,4.325)--(4.599,4.326)--(4.716,4.327)--(4.832,4.328)--(4.949,4.329)--(5.065,4.330)%
  --(5.182,4.331)--(5.299,4.332)--(5.415,4.333)--(5.532,4.334)--(5.649,4.335)--(5.765,4.336)%
  --(5.882,4.337)--(5.999,4.338)--(6.115,4.339)--(6.232,4.340)--(6.348,4.341)--(6.465,4.342)%
  --(6.582,4.343)--(6.698,4.344)--(6.815,4.345)--(6.932,4.346)--(7.048,4.347)--(7.165,4.348)%
  --(7.282,4.349)--(7.398,4.350)--(7.515,4.351)--(7.631,4.352)--(7.748,4.353)--(7.865,4.354)%
  --(7.981,4.355)--(8.098,4.356)--(8.215,4.357)--(8.331,4.358)--(8.448,4.359)--(8.565,4.360)%
  --(8.681,4.361)--(8.798,4.362)--(8.914,4.363)--(9.031,4.364)--(9.148,4.365)--(9.264,4.366)%
  --(9.381,4.367)--(9.498,4.368)--(9.614,4.369)--(9.731,4.370)--(9.848,4.371)--(9.964,4.372)%
  --(10.081,4.373)--(10.197,4.374)--(10.314,4.375)--(10.431,4.376)--(10.547,4.377)--(10.664,4.378)%
  --(10.781,4.379)--(10.897,4.380)--(11.014,4.381)--(11.131,4.382)--(11.247,4.383)--(11.364,4.384)%
  --(11.480,4.385)--(11.597,4.386)--(11.714,4.387)--(11.830,4.388)--(11.947,4.389);
\gpcolor{rgb color={1.000,0.537,0.000}}
\gpsetlinetype{gp lt plot 1}
\draw[gp path] (0.400,4.197)--(0.517,4.204)--(0.633,4.210)--(0.750,4.217)--(0.867,4.224)%
  --(0.983,4.231)--(1.100,4.238)--(1.216,4.245)--(1.333,4.251)--(1.450,4.258)--(1.566,4.265)%
  --(1.683,4.272)--(1.800,4.279)--(1.916,4.286)--(2.033,4.292)--(2.150,4.299)--(2.266,4.306)%
  --(2.383,4.313)--(2.499,4.320)--(2.616,4.326)--(2.733,4.333)--(2.849,4.340)--(2.966,4.347)%
  --(3.083,4.354)--(3.199,4.361)--(3.316,4.367)--(3.433,4.374)--(3.549,4.381)--(3.666,4.388)%
  --(3.782,4.395)--(3.899,4.402)--(4.016,4.408)--(4.132,4.415)--(4.249,4.422)--(4.366,4.429)%
  --(4.482,4.436)--(4.599,4.442)--(4.716,4.449)--(4.832,4.456)--(4.949,4.463)--(5.065,4.470)%
  --(5.182,4.477)--(5.299,4.483)--(5.415,4.490)--(5.532,4.497)--(5.649,4.504)--(5.765,4.511)%
  --(5.882,4.518)--(5.999,4.524)--(6.115,4.531)--(6.232,4.538)--(6.348,4.545)--(6.465,4.552)%
  --(6.582,4.558)--(6.698,4.565)--(6.815,4.572)--(6.932,4.579)--(7.048,4.586)--(7.165,4.593)%
  --(7.282,4.599)--(7.398,4.606)--(7.515,4.613)--(7.631,4.620)--(7.748,4.627)--(7.865,4.634)%
  --(7.981,4.640)--(8.098,4.647)--(8.215,4.654)--(8.331,4.661)--(8.448,4.668)--(8.565,4.675)%
  --(8.681,4.681)--(8.798,4.688)--(8.914,4.695)--(9.031,4.702)--(9.148,4.709)--(9.264,4.715)%
  --(9.381,4.722)--(9.498,4.729)--(9.614,4.736)--(9.731,4.743)--(9.848,4.750)--(9.964,4.756)%
  --(10.081,4.763)--(10.197,4.770)--(10.314,4.777)--(10.431,4.784)--(10.547,4.791)--(10.664,4.797)%
  --(10.781,4.804)--(10.897,4.811)--(11.014,4.818)--(11.131,4.825)--(11.247,4.831)--(11.364,4.838)%
  --(11.480,4.845)--(11.597,4.852)--(11.714,4.859)--(11.830,4.866)--(11.947,4.872);
\draw[gp path] (0.400,4.001)--(0.517,4.008)--(0.633,4.015)--(0.750,4.022)--(0.867,4.028)%
  --(0.983,4.035)--(1.100,4.042)--(1.216,4.049)--(1.333,4.056)--(1.450,4.063)--(1.566,4.069)%
  --(1.683,4.076)--(1.800,4.083)--(1.916,4.090)--(2.033,4.097)--(2.150,4.103)--(2.266,4.110)%
  --(2.383,4.117)--(2.499,4.124)--(2.616,4.131)--(2.733,4.138)--(2.849,4.144)--(2.966,4.151)%
  --(3.083,4.158)--(3.199,4.165)--(3.316,4.172)--(3.433,4.179)--(3.549,4.185)--(3.666,4.192)%
  --(3.782,4.199)--(3.899,4.206)--(4.016,4.213)--(4.132,4.219)--(4.249,4.226)--(4.366,4.233)%
  --(4.482,4.240)--(4.599,4.247)--(4.716,4.254)--(4.832,4.260)--(4.949,4.267)--(5.065,4.274)%
  --(5.182,4.281)--(5.299,4.288)--(5.415,4.295)--(5.532,4.301)--(5.649,4.308)--(5.765,4.315)%
  --(5.882,4.322)--(5.999,4.329)--(6.115,4.335)--(6.232,4.342)--(6.348,4.349)--(6.465,4.356)%
  --(6.582,4.363)--(6.698,4.370)--(6.815,4.376)--(6.932,4.383)--(7.048,4.390)--(7.165,4.397)%
  --(7.282,4.404)--(7.398,4.411)--(7.515,4.417)--(7.631,4.424)--(7.748,4.431)--(7.865,4.438)%
  --(7.981,4.445)--(8.098,4.451)--(8.215,4.458)--(8.331,4.465)--(8.448,4.472)--(8.565,4.479)%
  --(8.681,4.486)--(8.798,4.492)--(8.914,4.499)--(9.031,4.506)--(9.148,4.513)--(9.264,4.520)%
  --(9.381,4.527)--(9.498,4.533)--(9.614,4.540)--(9.731,4.547)--(9.848,4.554)--(9.964,4.561)%
  --(10.081,4.567)--(10.197,4.574)--(10.314,4.581)--(10.431,4.588)--(10.547,4.595)--(10.664,4.602)%
  --(10.781,4.608)--(10.897,4.615)--(11.014,4.622)--(11.131,4.629)--(11.247,4.636)--(11.364,4.643)%
  --(11.480,4.649)--(11.597,4.656)--(11.714,4.663)--(11.830,4.670)--(11.947,4.677);
\draw[gp path] (0.400,4.580)--(0.517,4.575)--(0.633,4.571)--(0.750,4.566)--(0.867,4.561)%
  --(0.983,4.556)--(1.100,4.551)--(1.216,4.546)--(1.333,4.542)--(1.450,4.537)--(1.566,4.532)%
  --(1.683,4.527)--(1.800,4.522)--(1.916,4.517)--(2.033,4.513)--(2.150,4.508)--(2.266,4.503)%
  --(2.383,4.498)--(2.499,4.493)--(2.616,4.488)--(2.733,4.484)--(2.849,4.479)--(2.966,4.474)%
  --(3.083,4.469)--(3.199,4.464)--(3.316,4.459)--(3.433,4.454)--(3.549,4.450)--(3.666,4.445)%
  --(3.782,4.440)--(3.899,4.435)--(4.016,4.430)--(4.132,4.425)--(4.249,4.421)--(4.366,4.416)%
  --(4.482,4.411)--(4.599,4.406)--(4.716,4.401)--(4.832,4.396)--(4.949,4.392)--(5.065,4.387)%
  --(5.182,4.382)--(5.299,4.377)--(5.415,4.372)--(5.532,4.367)--(5.649,4.363)--(5.765,4.358)%
  --(5.882,4.353)--(5.999,4.348)--(6.115,4.343)--(6.232,4.338)--(6.348,4.334)--(6.465,4.329)%
  --(6.582,4.324)--(6.698,4.319)--(6.815,4.314)--(6.932,4.309)--(7.048,4.305)--(7.165,4.300)%
  --(7.282,4.295)--(7.398,4.290)--(7.515,4.285)--(7.631,4.280)--(7.748,4.276)--(7.865,4.271)%
  --(7.981,4.266)--(8.098,4.261)--(8.215,4.256)--(8.331,4.251)--(8.448,4.247)--(8.565,4.242)%
  --(8.681,4.237)--(8.798,4.232)--(8.914,4.227)--(9.031,4.222)--(9.148,4.217)--(9.264,4.213)%
  --(9.381,4.208)--(9.498,4.203)--(9.614,4.198)--(9.731,4.193)--(9.848,4.188)--(9.964,4.184)%
  --(10.081,4.179)--(10.197,4.174)--(10.314,4.169)--(10.431,4.164)--(10.547,4.159)--(10.664,4.155)%
  --(10.781,4.150)--(10.897,4.145)--(11.014,4.140)--(11.131,4.135)--(11.247,4.130)--(11.364,4.126)%
  --(11.480,4.121)--(11.597,4.116)--(11.714,4.111)--(11.830,4.106)--(11.947,4.101);
\draw[gp path] (0.400,4.385)--(0.517,4.380)--(0.633,4.375)--(0.750,4.370)--(0.867,4.365)%
  --(0.983,4.360)--(1.100,4.356)--(1.216,4.351)--(1.333,4.346)--(1.450,4.341)--(1.566,4.336)%
  --(1.683,4.331)--(1.800,4.326)--(1.916,4.322)--(2.033,4.317)--(2.150,4.312)--(2.266,4.307)%
  --(2.383,4.302)--(2.499,4.297)--(2.616,4.293)--(2.733,4.288)--(2.849,4.283)--(2.966,4.278)%
  --(3.083,4.273)--(3.199,4.268)--(3.316,4.264)--(3.433,4.259)--(3.549,4.254)--(3.666,4.249)%
  --(3.782,4.244)--(3.899,4.239)--(4.016,4.235)--(4.132,4.230)--(4.249,4.225)--(4.366,4.220)%
  --(4.482,4.215)--(4.599,4.210)--(4.716,4.206)--(4.832,4.201)--(4.949,4.196)--(5.065,4.191)%
  --(5.182,4.186)--(5.299,4.181)--(5.415,4.177)--(5.532,4.172)--(5.649,4.167)--(5.765,4.162)%
  --(5.882,4.157)--(5.999,4.152)--(6.115,4.148)--(6.232,4.143)--(6.348,4.138)--(6.465,4.133)%
  --(6.582,4.128)--(6.698,4.123)--(6.815,4.118)--(6.932,4.114)--(7.048,4.109)--(7.165,4.104)%
  --(7.282,4.099)--(7.398,4.094)--(7.515,4.089)--(7.631,4.085)--(7.748,4.080)--(7.865,4.075)%
  --(7.981,4.070)--(8.098,4.065)--(8.215,4.060)--(8.331,4.056)--(8.448,4.051)--(8.565,4.046)%
  --(8.681,4.041)--(8.798,4.036)--(8.914,4.031)--(9.031,4.027)--(9.148,4.022)--(9.264,4.017)%
  --(9.381,4.012)--(9.498,4.007)--(9.614,4.002)--(9.731,3.998)--(9.848,3.993)--(9.964,3.988)%
  --(10.081,3.983)--(10.197,3.978)--(10.314,3.973)--(10.431,3.969)--(10.547,3.964)--(10.664,3.959)%
  --(10.781,3.954)--(10.897,3.949)--(11.014,3.944)--(11.131,3.940)--(11.247,3.935)--(11.364,3.930)%
  --(11.480,3.925)--(11.597,3.920)--(11.714,3.915)--(11.830,3.910)--(11.947,3.906);
\gpcolor{color=gp lt color border}
\node[gp node right] at (10.479,0.703) {Interpolazione da lontano};
\gpcolor{rgb color={0.102,0.000,1.000}}
\gpsetlinetype{gp lt plot 0}
\draw[gp path] (10.663,0.703)--(11.579,0.703);
\draw[gp path] (2.881,0.369)--(2.966,0.471)--(3.083,0.611)--(3.199,0.751)--(3.316,0.891)%
  --(3.433,1.031)--(3.549,1.172)--(3.666,1.312)--(3.782,1.452)--(3.899,1.592)--(4.016,1.732)%
  --(4.132,1.872)--(4.249,2.013)--(4.366,2.153)--(4.482,2.293)--(4.599,2.433)--(4.716,2.573)%
  --(4.832,2.713)--(4.949,2.854)--(5.065,2.994)--(5.182,3.134)--(5.299,3.274)--(5.415,3.414)%
  --(5.532,3.554)--(5.649,3.695)--(5.765,3.835)--(5.882,3.975)--(5.999,4.115)--(6.115,4.255)%
  --(6.232,4.395)--(6.348,4.535)--(6.465,4.676)--(6.582,4.816)--(6.698,4.956)--(6.815,5.096)%
  --(6.932,5.236)--(7.048,5.376)--(7.165,5.517)--(7.282,5.657)--(7.398,5.797)--(7.515,5.937)%
  --(7.631,6.077)--(7.748,6.217)--(7.865,6.358)--(7.981,6.498)--(8.098,6.638)--(8.215,6.778)%
  --(8.331,6.918)--(8.448,7.058)--(8.565,7.198)--(8.681,7.339)--(8.798,7.479)--(8.914,7.619)%
  --(9.031,7.759)--(9.148,7.899)--(9.264,8.039)--(9.381,8.180)--(9.498,8.320)--(9.549,8.381);
\gpcolor{rgb color={0.294,0.463,1.000}}
\gpsetlinetype{gp lt plot 1}
\draw[gp path] (2.849,0.369)--(2.966,0.517)--(3.083,0.665)--(3.199,0.813)--(3.316,0.961)%
  --(3.433,1.109)--(3.549,1.257)--(3.666,1.405)--(3.782,1.553)--(3.899,1.701)--(4.016,1.849)%
  --(4.132,1.996)--(4.249,2.144)--(4.366,2.292)--(4.482,2.440)--(4.599,2.588)--(4.716,2.736)%
  --(4.832,2.884)--(4.949,3.032)--(5.065,3.180)--(5.182,3.328)--(5.299,3.476)--(5.415,3.624)%
  --(5.532,3.772)--(5.649,3.920)--(5.765,4.068)--(5.882,4.216)--(5.999,4.364)--(6.115,4.512)%
  --(6.232,4.660)--(6.348,4.808)--(6.465,4.956)--(6.582,5.104)--(6.698,5.252)--(6.815,5.399)%
  --(6.932,5.547)--(7.048,5.695)--(7.165,5.843)--(7.282,5.991)--(7.398,6.139)--(7.515,6.287)%
  --(7.631,6.435)--(7.748,6.583)--(7.865,6.731)--(7.981,6.879)--(8.098,7.027)--(8.215,7.175)%
  --(8.331,7.323)--(8.448,7.471)--(8.565,7.619)--(8.681,7.767)--(8.798,7.915)--(8.914,8.063)%
  --(9.031,8.211)--(9.148,8.359)--(9.165,8.381);
\draw[gp path] (3.056,0.369)--(3.083,0.403)--(3.199,0.551)--(3.316,0.699)--(3.433,0.847)%
  --(3.549,0.995)--(3.666,1.143)--(3.782,1.291)--(3.899,1.439)--(4.016,1.587)--(4.132,1.735)%
  --(4.249,1.883)--(4.366,2.031)--(4.482,2.179)--(4.599,2.327)--(4.716,2.474)--(4.832,2.622)%
  --(4.949,2.770)--(5.065,2.918)--(5.182,3.066)--(5.299,3.214)--(5.415,3.362)--(5.532,3.510)%
  --(5.649,3.658)--(5.765,3.806)--(5.882,3.954)--(5.999,4.102)--(6.115,4.250)--(6.232,4.398)%
  --(6.348,4.546)--(6.465,4.694)--(6.582,4.842)--(6.698,4.990)--(6.815,5.138)--(6.932,5.286)%
  --(7.048,5.434)--(7.165,5.582)--(7.282,5.730)--(7.398,5.877)--(7.515,6.025)--(7.631,6.173)%
  --(7.748,6.321)--(7.865,6.469)--(7.981,6.617)--(8.098,6.765)--(8.215,6.913)--(8.331,7.061)%
  --(8.448,7.209)--(8.565,7.357)--(8.681,7.505)--(8.798,7.653)--(8.914,7.801)--(9.031,7.949)%
  --(9.148,8.097)--(9.264,8.245)--(9.372,8.381);
\draw[gp path] (2.686,0.369)--(2.733,0.422)--(2.849,0.554)--(2.966,0.687)--(3.083,0.819)%
  --(3.199,0.951)--(3.316,1.084)--(3.433,1.216)--(3.549,1.348)--(3.666,1.481)--(3.782,1.613)%
  --(3.899,1.746)--(4.016,1.878)--(4.132,2.010)--(4.249,2.143)--(4.366,2.275)--(4.482,2.407)%
  --(4.599,2.540)--(4.716,2.672)--(4.832,2.804)--(4.949,2.937)--(5.065,3.069)--(5.182,3.201)%
  --(5.299,3.334)--(5.415,3.466)--(5.532,3.599)--(5.649,3.731)--(5.765,3.863)--(5.882,3.996)%
  --(5.999,4.128)--(6.115,4.260)--(6.232,4.393)--(6.348,4.525)--(6.465,4.657)--(6.582,4.790)%
  --(6.698,4.922)--(6.815,5.055)--(6.932,5.187)--(7.048,5.319)--(7.165,5.452)--(7.282,5.584)%
  --(7.398,5.716)--(7.515,5.849)--(7.631,5.981)--(7.748,6.113)--(7.865,6.246)--(7.981,6.378)%
  --(8.098,6.510)--(8.215,6.643)--(8.331,6.775)--(8.448,6.908)--(8.565,7.040)--(8.681,7.172)%
  --(8.798,7.305)--(8.914,7.437)--(9.031,7.569)--(9.148,7.702)--(9.264,7.834)--(9.381,7.966)%
  --(9.498,8.099)--(9.614,8.231)--(9.731,8.364)--(9.746,8.381);
\draw[gp path] (2.917,0.369)--(2.966,0.425)--(3.083,0.557)--(3.199,0.690)--(3.316,0.822)%
  --(3.433,0.954)--(3.549,1.087)--(3.666,1.219)--(3.782,1.351)--(3.899,1.484)--(4.016,1.616)%
  --(4.132,1.748)--(4.249,1.881)--(4.366,2.013)--(4.482,2.146)--(4.599,2.278)--(4.716,2.410)%
  --(4.832,2.543)--(4.949,2.675)--(5.065,2.807)--(5.182,2.940)--(5.299,3.072)--(5.415,3.204)%
  --(5.532,3.337)--(5.649,3.469)--(5.765,3.601)--(5.882,3.734)--(5.999,3.866)--(6.115,3.999)%
  --(6.232,4.131)--(6.348,4.263)--(6.465,4.396)--(6.582,4.528)--(6.698,4.660)--(6.815,4.793)%
  --(6.932,4.925)--(7.048,5.057)--(7.165,5.190)--(7.282,5.322)--(7.398,5.455)--(7.515,5.587)%
  --(7.631,5.719)--(7.748,5.852)--(7.865,5.984)--(7.981,6.116)--(8.098,6.249)--(8.215,6.381)%
  --(8.331,6.513)--(8.448,6.646)--(8.565,6.778)--(8.681,6.910)--(8.798,7.043)--(8.914,7.175)%
  --(9.031,7.308)--(9.148,7.440)--(9.264,7.572)--(9.381,7.705)--(9.498,7.837)--(9.614,7.969)%
  --(9.731,8.102)--(9.848,8.234)--(9.964,8.366)--(9.977,8.381);
%% coordinates of the plot area
\gpdefrectangularnode{gp plot 1}{\pgfpoint{0.400cm}{0.369cm}}{\pgfpoint{11.947cm}{8.381cm}}
\end{tikzpicture}
%% gnuplot variables
 \caption{Incertezza sulle rette} \label{fig:01_graph_2.tex} \end{grafico}
%
Si trova per il fuoco il~valore 
\[ f^{\star}_1=\left(6.72 \pm 0.07\right) \cm , \] 
si \`e considerata trascurabile l'incertezza su $\mu^{\star}$ (la scelta di considerare l'incertezza strumentale non influisce quindi sugli altri risultati) rispetto a quella fornita dal laboratorio su $P_L$ e $P_O$, pari a $\sigma _P = 0.05 \cm$: ci\`o porta propagando a un'incertezza di $\sqrt{2}   \sigma_P = 0.07 \cm$. Tale stima va considerata a meno della correzione di aberrazione sferica (per la quale cfr. \autoref{subsec:aberrazione_sferica}).

