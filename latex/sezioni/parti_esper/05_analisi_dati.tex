I risultati delle misure sono riportati nella \autoref{tab:05_tab_1.tex}. 
Per questa esperienza si sono utilizzate le proprietà del doppietto 
acromatico "di Dollond", lente composta da una lente convergente e 
una divergente, le quali siano allineate sul medesimo asse ottico. 
Tale configurazione in condizioni particolari massimizza il numero 
di Abbe della lente stessa: infatti il doppietto acromatico fa 
sì che i raggi siano proiettati nel medesimo punto a prescindere 
dalla loro lunghezza d'onda. Per l'analisi si è considerato il 
doppietto acromatico come ideale, e non si è valutato l'errore 
sistematico legato al fatto che il raggio avrebbe potuto non essere 
effettivamente parallelo con uno dei due filtri. Tale proprietà 
risulta importante per la misurazione del numero di Abbe della lente 
presa in esame, in quanto è necessario che il fascio di partenza sia 
parallelo a prescindere dal filtro utilizzato.
\begin{tabella}
	\centering
	\begin{tabulary}{\textwidth}{CCCCC}
\toprule
$f_C$ & $f_F$ & $A$ & $V$ &$\sigma_V$
3.10 & 4.25 & 1.15 & 59 & 4
3.05 & 4.00 & 0.95 & 72 & 6
2.90 & 4.25 & 1.35 & 51 & 3
2.82 & 4.23 & 1.41 & 49 & 3
3.21 & 4.26 & 1.05 & 66 & 5
2.92 & 4.13 & 1.21 & 57 & 4
3.04 & 4.18 & 1.14 & 60 & 4
3.13 & 3.98 & 0.85 & 81 & 7
3.18 & 4.26 & 1.08 & 64 & 5
3.08 & 4.16 & 1.08 & 64 & 5
\bottomrule
\end{tabulary}



	\caption{Risultati aberrazione cromatica}
	\label{tab:05_tab_1.tex}
\end{tabella}

Per la stima del numero di Abbe $V$ si \`e utilizzata la formula
\[ V=\frac{f}{A} \]
dove \(A= f_C - f_F\) e $f_D$ \`e la miglior stima del fuoco nel 
giallo dalle precedenti esperienze. Per la stima di $A$, si è
utilizzata una media semplice sul campione ${A_i}$ creato a partire 
dai dati:
\[ A = (0.113 \pm 0.005) \cm .\]
Considerando invece il campione di \( V_i=\frac{f}{A_i} \), la formula di propagazione porta a 
\[  \sigma^2_{V_i} = \frac{
\frac{2f^2   s^2}{A_i^2}   \sigma^2 +  \sigma^2_{f}
}{A_i^2} \]
e la media pesata restituisce la nostra stima:
\[V= 56 \pm 1 .\]
