L'esperienza \`e suddivisa in 5 fasi diverse, ognuna con 
un'approccio sperimentale e un'analisi dati differente. Questa 
sezione ha lo scopo di descrivere la metodologia prescelta dagli 
sperimentatori in laboratorio, dunque non vi si discutono le scelte 
numeriche fatte o le analisi dati, un'accurata descrizione delle 
quali è fornita nelle sezioni successive.

Preliminarmente a ogni esperienza, è stato necessario preparare il 
banco ottico con la seguente procedura: in primo luogo tutti i 
micrometri sono portati allo zero della scala, dopodichè il 
cavaliere portalampada \`e fissato in un punto arbitrario della 
guida; tale punto è stato scelto il più marginalmente possibile 
sulla sinistra per avere spazio a disposizione per gli altri 
cavalieri. Bloccato con la vite il cavaliere portalampada, si 
utilizza il cavaliere con squadra per andare a leggere sulla scala 
millimetrata il valore che, dopo aver finito la preparazione 
dell'apparato sperimentale, corrisponder\`a alla posizione 
dell'oggetto relativa alla guida ottica. Il cavaliere con squadra va 
subito rimosso dalla guida ottica. Fatto ciò, viene montata la 
mascherina con foro singolo sull'apposito supporto presente sul 
cavaliere portalampada, mantenendo l'anello di fissaggio rivolto a 
destra. \newline Sistemato il cavaliere portalampada, il cavaliere 
portalente viene posizionato sulla guida ottica e vi si avvita la 
lente sul lato destro, poi il diaframma sul sinistro. \'E necessaria 
particolare attenzione nell'avvitare la lente fino alla fine del 
filo delle componenti meccaniche del sistema. Alla sinistra del 
portalente non ancora fissato va posto il cavaliere portaschermo con 
micrometro, in modo tale che lo schermo sia rivolto verso sinistra, 
e si inserisce nel cavaliere portalampada il filtro di colore giallo.

Per stimare la lunghezza focale attraverso il metodo 
dell'autocollimazione si è operato spostando il cavaliere portalente 
con micrometro sulla guida ottica manuale o attraverso il 
micrometro, e tramite lettura di valori sul metro posto sulla guida 
o sui micrometri. Il diaframma utilizzato per questa esperienza è 
quello con un singolo foro centrale. \newline Preparato il banco 
ottico, prima di iniziare la vera e propria presa dati si è cercato 
approssimativamente quale potesse essere la posizione del fuoco 
della lente. Si è quindi mosso il cavaliere portalente e 
successivamente confrontata grossolanamente la dimensione delle 
proiezioni del raggio sullo schermo quando esso veniva posto vicino 
o lontano dal cavaliere portalente. Eseguita una volta questa 
operazione, essa è stata ripetuta più volte muovendo il cavaliere 
portalente a sinistra o a destra a seconda che l'immagine nello 
schermo fosse più grande rispettivamente nel caso lo schermo fosse 
posto più vicino o più lontano dalla lente. Una reiterazione di 
queste azioni ha permesso dopo circa 4-5 ripetizioni di trovare un 
punto per il cavaliere portalente per il quale il cerchio proiettato 
sullo schermo fosse di grandezza approssimativamente costante al 
muoversi del cavaliere portaschermo. \newline Trovato il punto, ci si 
è spostati a destra di circa mezzo centimetro (iniziando con il 
valore zero della scala del micrometro sono possibili infatti solo 
spostamenti dello stesso che avvicinano la lente alla lampada, non 
c'è modo di allontanarle) e si è saldamente bloccato il cavaliere 
portalampada con l'apposita vite. Il valore letto dall'indice del 
cavaliere portalente è stato registrato e poi è stata iniziata la 
vera e propria presa dati. Per prima cosa è stato regolato il 
micrometro in modo tale che la lente fosse spostata di circa mezzo 
centimetro, in modo cioè che si trovasse sul punto nel quale a 
occhio era stata individuata la condizione di fascio parallelo. 
Successivamente, si è misurato il diametro a posizioni costanti 
dello schermo (da vicino, $20\cm$; da lontano, $130\cm$). Per fare 
ciò, è stato utilizzato il reticolo presente sullo schermo 
smerigliato: operando con il micrometro che permetteva allo schermo 
uno spostamento perpendicolare all'asse ottico, sono state misurate 
le posizioni che esso segnava quando una linea del reticolo 
diventava tangente al cerchio luminoso. Presi i dati, il diametro 
del cerchio \`e stato ottenuto sommando un centimetro alla 
differenza ottenuta, come risulta evidente dalla costruzione fisica 
del sistema. Confrontando i dati con la stessa logica dei passaggi 
precedenti, anche grazie a un programma che consentiva di 
visualizzarli istantaneamente su un grafico, si è spostata la lente 
avvicinandola o allontanandola dall'oggetto andando a operare sul 
micrometro presente nel cavaliere portalente. Trovato il punto in 
cui la differenza tra il diametro proiettato sullo schermo più 
distante e quello proiettato sullo schermo più prossimo cambiasse 
segno, si è indagato con maggiore precisione sull'area compresa tra 
le due misurazioni.

Per trovare il fuoco attraverso il metodo dei punti coniugati, il 
cavaliere portalampada \`e stato bloccato, ma, a differenza di 
quella precedente, non si è bloccato il cavaliere portalente. Si è 
operato senza micrometri, solamente spostando i cavalieri lungo la 
guida ottica, e come nell'esperienza precedente si è utilizzato il 
diaframma con singolo buco centrale. Preparato il banco ottico si è 
posto il cavaliere portalente inizialmente a una distanza dal 
cavaliere portalampada maggiore del fuoco che era stato stimato con 
il metodo dell'autocollimazione. Muovendo il cavaliere portaschermo 
si è cercato il punto in cui il raggio proiettato presentava 
diametro più piccolo possibile (per vederlo con più facilità si è 
fatto uso dell'oculare in dotazione), a quel punto si sono 
registrati su una tabella la posizione del cavaliere portalente e 
quella del cavaliere portaschermo. Il cavaliere portalente è stato 
poi spostato a destra e si è ripetuta l'operazione, aggiungendo una 
riga alla tabella precedentemente fatta. Il processo è stato 
ripetuto indagando il più uniformemente possibile le posizioni 
permesse dalla guida ottica.

Per trovare il fuoco attraverso il metodo di Bessel, sono stati 
tenuti fermi i cavalieri portalampada e quello portaschermo e si è 
mosso solamente il cavaliere portalente. Si è preparato il banco 
ottico e utilizzato il diaframma con buco singolo e il cavaliere 
portalente senza micrometro. Per prima cosa, data l'analisi teorica 
che si può fare a riguardo, si è bloccato il cavaliere protalampada 
il più a sinistra possibile (come nelle altre esperienze) e, 
utilizzando le stime fatte con le altre esperienze, si è bloccato il 
cavaliere portaschermo a più di 4f. Muovendo il cavaliere 
portalente, si sono cercate le due posizioni per cui l'immagine 
andava a fuoco. Nel caso non risultasse evidente distinguere un 
fuoco dall'altro si è allontanato ulteriormente il cavaliere 
portaschermo. Trovata una posizione dei due cavalieri fissi tale che 
i due fuochi fossero facilmente distinguibili e i cavalieri non 
fossero troppo lontani tra loro, si è iniziata la presa dati. 
Aiutandosi con l'oculare, è stata misurata la posizione della lente 
per la quale il fascio andava a fuoco sullo schermo. Spostando il 
cavaliere portalente è stato, successivamente, registrato il punto 
in cui l'immagine andava nuovamente a fuoco nonostante la posizione 
fosse diversa. Tali misurazioni di posizione sono stati ripetuti 10 
volte ciascuno.

Dopo le varie stime del fuoco, si sono cercate l'aberrazione sferica 
e l'aberrazione cromatica della lente. Per quanto riguarda 
l'aberrazione sferica, si è preparato il banco ottico andando, 
differentemente rispetto ai casi precedenti, ad utilizzare entrambi 
i cavalieri portalenti: quello con micrometro portante il doppietto 
di Dollond e quello senza micrometro portante la lente. Inizialmente 
non è stato montato alcun diaframma ed è stato posizionato sulla 
guida ottica solamente il cavaliere con il doppietto acromatico. 
Grazie ad una bacchettina lunga circa quanto la distanza focale del 
doppietto, si è bloccato ilcavaliere portalente in una posizione 
tale che l'oggetto occupasse uno dei suoi fuochi. Per fare ciò si è 
fatto in mod chetale barretta rimanessesospesa bloccata dagli 
attriti tra il doppietto e la mascherina montata sul cavaliere 
portalampada, a quel punto il cavaliere con il doppietto acromatico 
è stato bloccato con la vite. Successivamente, andando a spingere la 
lente verso sinistra mantenendo la vite del cavaliere portalente 
avvitata, grazie alla molla presente all'interno del micrometro, è 
stata rimossa la bacchettina. Analogamente a quanto fatto nel caso 
dell'autocollimazione (ma con il doppietto di Dollond e senza 
diaframma) è stato cercato velocemente il punto in cui il diametro 
della proiezione sullo schermo sembrava costante al variare della 
posizione del cavaliere portaschermo. Dopo aver aggiustato un paio 
di volte con il micrometro la posizione della lente, si è potuto 
utilizzare il raggio parallelo creato per la misurazione 
dell'aberrazione sferica. è stato messo tra il cavaliere con il 
doppietto acromatico e il cavaliere portaschermo il cavaliere con 
lente, al quale è stato avvitato il diaframma con 4 buchi, in modo 
tale che i due buchi marginali fossero su un piano orizzontale. Il 
cavaliere portalente senza micrometro è stato posizionato il più 
vicino possibile all'altro cavaliere portalente, in modo che non ci 
fossero dispersioni del fascio uniforme legate alla distanza e che 
tutti e 4 i buchi del diaframma fossero ben illuminati. Fatto ciò, si 
è andato a muovere il cavaliere portaschermo fino a trovare 
approssimativamente il punto in cui le 4 immagini si avvicinavano a 
formare quello che, a primo impatto, sembrava un punto unico. Si è 
bloccato il cavaliere portaschermo in quel punto e poi, andando a 
operare sul micrometro, si è verificato che il movimento dello 
schermo permettesse di raggiungere agevolmente la posizione della 
quale i 4 punti erano ancora separati e la posizione nella quale i 4 
punti se separavano di nuovo nettamente. Fatto ciò è stato possibile 
iniziare la vera e propria presa dati, per la quale è stato 
solamente utilizzato il micrometro che permetteva allo schermo uno 
spostamento parallelo all'asse ottico. Partendo dal punto per il 
quale i 4 punti erano ben distinti si è cercato il punto in cui i 
raggi marginali convergessero in un unico punto, facendo vedere 
l'immagine all'oculare come tre punti allineati verticalmente. 
Registrato il valore letto sul micrometro, si è portato avanti il 
micrometro fino a vedere i tre punti allienati, questa volta, 
orizzontalmente. Questo valore è stato registrato e poi si è 
registrata la posizione nella quale nello schermo erano ancora 
visibili tre punti orizzontali. Questi ultimi due valori sono stati 
tali che se la lettura del micrometro fosse stata inferiore al primo 
valore o superiore al secondo non si sarebbero visti i punti 
orizzontalmente allienati, mentre si sono visti per tutti i valori 
del micrometro tra loro comrpesi. Sono state prese 10 terne di 
valori diversi. Dopo aver stimato il fuoco prossimale andando a 
mediare per ogni terna gli ultimi due valori si è tornati in questa 
posizione per misurare la distanza tra i due punti orizzontali in 
quella posizione, ripetendo il procedimento fatto nell'esperienza 
dell'autocollimazione di stima di lunghezze sullo schermo 
smerigliato,  stando però attenti nel sommare o meno il passo del 
reticolo. Sfortunatamente a causa di un dissesto delle componenti 
meccaniche le ultime tre misure della distanza tra i raggi marginali 
nel fuoco prossimale sono risultate impossibili da prendere.

Per quanto riguarda la misurazione dell'aberrazione cromatica, 
ivnece, sono stati usati i filtri presenti sul cavaliere portalente 
differentemente da come fatto in tutte le altre esperienze. Il banco 
ottico è stato preparato, ed è stato creato un fascio di luce 
parallela analogamente a come era stato fatto nel caso della 
misurazione dell'aberrazione sferica. Anche in questo caso si è 
utilizzato il diaframma con i 4 buchi, andando a coprire però i 
buchi centrali, in modo da potersi concentrare sul fuoco marginale. 
Si è portato di nuovo lo schermo circa nella posizione in cui si 
vedevano convergere i fuochi marginali del colore giallo e si è 
bloccato tramite l'apposita vite. Fatto ciò, è stato cambiato il 
filtro, andando a creare un fascio di luce blu. Operando con il 
micrometro su schermo, si è cercato il punto in cui il fascio blu 
convergeva, proiettando sullo schermo smerigliato un solo punto blu. 
Registrato il valore letto sul micrometro, si è passati dal filtro 
blu a quello rosso e si è ripetuta la ricerca del fuoco, annotando 
il valore letto sul micrometro. L'esperienza è stata ripetuta 10 
volte.
