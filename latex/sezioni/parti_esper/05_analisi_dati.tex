%**05_tab_1**

Per questa esperienza si sono utilizzate le propriet\`a del doppietto acromatico "di Dollond", lente composita di una lente convergente e una divergente, le quali siano allineate sul medesimo asse ottico. Tale configurazione minimizza il numero di Abbe della lente stessa: infatti tale configurazione fa s\`i che i raggi siano proiettati nel medesimo punto a prescindere dalla loro lunghezza d'onda. Tale propriet\`a risulta importante per la misurazione del numero di Abbe della lente presa in esame, in quanto \`e necessario che il fascio di partenza sia parallelo a prescindere dal filtro utilizzato.

Per la stima del numero di Abbe $V$ si \`e utilizzata la formula
\[ V=\frac{f}{A} ,\]
dove \(A= f_C - f_F\) e $f$ \`e la miglior stima del fuoco nel giallo dalle precedenti esperienze. %Fuoco nel giallo --Davide
Per la stima di A, si \`e utilizzata una media semplice sul campione ${A_i}$ creato a partire dai dati:
\[ A = (0.113 \pm 0.005) \cm .\]
Considerando invece il campione di \( V_i=\frac{f}{A_i} \), la formula di propagazione porta a 
\[ Var(V_i) = \frac{
\frac{2f^2 \cdot s^2}{A_i^2} \cdot \sigma^2 + Var(f)
}{A_i^2} \]
e la media pesata restituisce la nostra stima:
\[V= 58 \pm 1 .\]


