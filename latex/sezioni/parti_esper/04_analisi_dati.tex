%**04_tab_1.tex**

I risultati delle misure sono riportati nella \autoref{tab:04_tab_1.tex}.
\begin{tabella}
	\centering
	\begin{tabulary}{\textwidth}{CCCCCCCCC}
\toprule
$f_m$ & $f_i$ & $f_f$ & $f_p$ & $t$ & $l$ & $\sigma_l$ & $c$ & $\sigma_c$ \\ \midrule
4.18 & 8.00 & 10.65 & 9.33 & 1.93 & 5.1 & 1.3 & 1.75 & 0.45\\ \midrule
4.10 & 8.18 & 10.60 & 9.39 & 2.02 & 5.3 & 1.2 & 1.80 & 0.41\\ \midrule
4.15 & 8.15 & 10.39 & 9.27 & 1.99 & 5.1 & 1.1 & 1.74 & 0.38\\ \midrule
4.08 & 7.50 & 9.90 & 9.20 & 1.94 & 4.6 & 2.2 & 1.57 & 0.41\\ \midrule
4.25 & 7.90 & 10.35 & 9.13 & 1.90 & 4.9 & 1.2 & 1.66 & 0.42\\ \midrule
3.90 & 8.75 & 9.80 & 9.28 & 1.95 & 5.4 & 0.5 & 1.83 & 0.18\\ \midrule
4.90 & 8.00 & 10.30 & 9.15 & 1.87 & 4.2 & 1.1 & 1.44 & 0.39\\ \midrule
4.10 & 8.25 & 10.30 & 9.28 & - & 5.2 & 1.0 & 1.76 & 0.35\\ \midrule
4.00 & 8.25 & 10.10 & 9.18 & - & 5.2 & 0.9 & 1.76 & 0.31\\ \midrule
4.07 & 7.31 & 10.50 & 8.91 & - & 4.8 & 1.6 & 1.64 & 0.54\\
\bottomrule
\end{tabulary}


	\caption{Aberrazione sferica, $[\mm\,]$ tranne le ultime due colonne adimensionali}
	\label{tab:04_tab_1.tex}
\end{tabella}

Per la stima del coefficiente di aberrazione sferica $c$, si \`e considerata la formula
\[ c=\frac{f   l}{R^2}, \] 
con \( l = \frac{f_i + f_f}{2} - f_m \) e $f$ miglior stima del fuoco nel giallo dalle precedenti esperienze.


Per trovare $l$ si \`e calcolato $l_i$ su ogni terna di valori, propagandone l'errore quadraticamente. Nel calcolo della propagazione si \`e notato che l'incertezza su $f_m$ è trascurabile rispetto a quella su $f_p$, infatti risulta da evidenze sperimentali che la profondità di campo dei raggi prossimali all'asse ottico \`e molto maggiore di quella dei raggi marginali: è stato più semplice comprendere la posizione del fuoco marginale rispetto a quella del fuoco prossimale, per identificare il quale è stato necessario addirittura trovare due valori all'interno del quale il fuoco fosse compreso. Al fuoco prossimale è stata associata, quindi, un'incertezza pari alla semiampiezza dell'intervallo di fuoco dei raggi prossimali, proprio in virt\`u delle difficolt\`a già espresse. Si \`e in seguito fornita una stima di $l$ come media pesata del campione cos\`i creato:
\[ l = (0.51 \pm 0.03) \cm. \]


Per la correzione di aberrazione sferica sulle stime $f^{\star}_j$ ottenute dalle precedenti esperienze, \`e necessario il valore $ l'$, definito come la distanza tra i punti di messa a fuoco rispettivamente dei raggi prossimali all'asse ottico e di quelli al margine del fascio, nel caso del diaframma a foro singolo (diametro $d_0=1.00 cm$). Per trovare tale valore si utilizza la definizione di aberrazione sferica longitudinale $l'=\frac{c(d/2)^2}{f}$: a partire dai dati dell'esperienza I, si \`e trovata l'ordinata dell'intersezione tra le due rette, corrispondente al diametro del fascio creato, $d= 1.182 \cm$ (non si \`e ricercata l'incertezza di tale valore, in quanto la correzione per aberrazione sferica prevede l'utilizzo esclusivamente del valore di $l'$, ma non dell'incertezza associata). Risolvendo il sistema tra la formula sopra detta e $l=\frac{cR^2}{f}$, 
\[l' = l(d/2R)^2 = 0.09\cm .\]


Con tale $l'$, si \`e compiuta una correzione di aberrazione sferica per ogni $f^{\star}_j$ ottenuto dalle precedenti esperienze, sommando $l'/2$ alla stima stessa e andando a stimare nuovamente la sua incertezza. Infatti i risultati presentati nel caso delle prime tre esperienze trovano il fuoco della lente a meno delle correzioni per aberrazione sferica: a causa della costruzione dell'apparato, l'oggetto non è effettivamente puntiforme, ma ha un diametro finito $d$ a causa del diaframma. Ciò vuol dire che, analogamente a quanto è stato notato nell'esecuzione dell'esperimento IV, i raggi che incidono la lente a distanza maggiore dal centro della stessa convergono prima degli altri. A causa di questo fenomeno irriducibile causato dalle approssimazioni considerate nella creazione del modello, quello che è stato trovato nelle esperienze precedenti non è realmente il fuoco (considerato come punto in cui convergono i punti prossimali all'asse ottico) ma è il centro dell'intervallo di messa a fuoco dell'immagine. Per questo, andando a correggere tale valore, risulta necessario aggiungere $\frac{l'}{2}$ per spostarsi dal punto in cui si \`e registrato il fuoco nelle prime tre esperienze al punto dove efffettivamente si ritiene più probabile sia il fuoco; all'incertezza va sommata quadraticamente la stessa quantità per considerare il fatto che non si sa esattamente dove sia il fuoco nell'intervallo che va dal fuoco registrato al fuoco pi\`u probabile. Consideriamo dunque f come media pesata di tali valori:
\[ f = (6.65 \pm 0.03) \cm. \] 


Per quanto riguarda $c$, si \`e proceduto creando un nuovo campione per ogni presa dati; la formula di propagazione fornisce 
\[  \sigma^2_{c_i} =\frac{f^2    \sigma^2_{l_i} + l^2    \sigma^2_{f}}{R^4}, \]
in cui si \`e considerata trascurabile l'incertezza su $R=1.4\cm$, fornito dal laboratorio.
Da questo campione si \`e ricavata la media pesata, con relativa incertezza:
\[ c = (1.7 \pm 0.1). \] 



Come ulteriore stima di $c$, si \`e utilizzata la formula
\[ c_t = \frac{t   f   (f-l)}{2R^3}, \]
dove $t$ \`e l'aberrazione sferica trasversale misurata, stimata con una media semplice, potendo considerare $\sigma$ costante:
\[ t = (0.194 \pm 0.002) \cm.\]
L'incertezza su $t$ si \`e trovata a posteriori con la formula $RMS$, per evitarne una difficile stima a partire dalla sensibilit\`a dello strumento. 
L'incertezza di $c_t$ si trova quindi con la propagazione: 
\[ \sigma_{c_t} = \sqrt{ \frac
{ f^   (f-1)^2    \sigma^2_{t}  +  t^2   (2f-1)^2    \sigma^2{f}  +  t^2   f^2    \sigma^2_{l}}
{4R^6}
}   \]
\[c_t = 1.45 \pm 0.04. \]


La stima di $c_t$ attraverso $t$ \`e meno affidabile della stima $c$ attraverso $l$, in quanto i valori di $t$ registrati sono influenzati da errori sistematici non trascurabili nel caso in cui la misura venga fatta non esattamente nel fuoco prossimale. Per questo motivo per la correzione del fuoco si \`e preferito utilizzare l'aberrazione sferica longitudinale rispetto a quella trasversale.
Anche la differenza tra le due stime $c$ e $c_t$ \`e imputabile ai medesimi errori sistematici su $t$. La difficolt\`a nella stima di tali errori rende inoltre impossibile una media tra le due stime, irrimediabilmente correlate.
