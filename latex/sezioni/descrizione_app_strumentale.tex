L’apparato strumentale consiste in un cilindro in plexiglass al cui interno è posto un peso in acciaio di massa: $(115.5 \pm 0.1) g$
 e diametro: $(22.7 \pm 0.1)mm$ e altezza: $(34.0 \pm 0.1)mm$ collegato ad un filo di acciaio armonico, materiale dotato discrete
 capacità elastiche, ed immerso in acqua. Il filo è inoltre collegato ad una piattaforma rotante azionata da un motore di
 diametro circa 8 cm che, una volta azionato, induce un’oscillazione sul corpo formato dal filo più il pesetto. Il range delle
 frequenze alla quale il motore può essere indotto ad oscillare risulta compreso tra 0.800 e 1.200 Hz. Di suddetta oscillazione è
 possibile modificare il periodo e l’ampiezza può essere impostata da 2 millesimi di giro a 16. 
 
 Il tutto viene controllato e
 registrato mediante l’interfaccia fornita da un computer. I dati vengono acquisiti in intervalli di 0.05 secondi 
 permettendo una frequenza di rilevamento di 20 dati per secondo.
 L’interfaccia permette di visualizzare valori della frequenza, dell’ampiezza. Inoltre sono presenti
 diversi grafici, il più intereante dei quali rappresenta l'angolo di cui è ruotao il pendolo in funzione del tempo. Infine la
 presenza del pulsante offset permette di tarare l’apparato dopo ogni misurazione, al fine di limitare errori sistematici.


