Le notazioni usate nella relazione indicano:
\begin{itemize}
\item $P_O =$ posizione oggetto;
\item $P_L =$ posizione lente;
\item $P_S =$ posizione schermo;
\item $dr = 0.230 \cm$ spessore dello spigolo della lente;
\item $\overline{VV'} = 0.800 \cm$ distanza tra i vertici della lente;
\item $\overline{PP'} \approx \nicefrac{\overline{VV'}}{3}$ distanza tra i piani principali;
\item Rosso = riga $C$ dello spettro di Fraunhofer, $\lambda = 656.3.1 \nm$;
\item Giallo = riga $D$ dello spettro di Fraunhofer, $\lambda = 589.3 \nm$;
\item Blu = riga $F$ dello spettro di Fraunhofer, $\lambda = 486.1 \nm$.
\end{itemize}
\FloatBarrier
\subsection{Autocollimazione}
\label{subsec:autocollimazione}
%**01_tab_1.tx**

Per la stima della posizione del fuoco e il calcolo dell'incertezza, si considera la seguente formula:
\begin{equation} 
f = P_L - P_0 + \left(\mu _0 - \mu ^{\star}\right) + \frac{\mathrm{d}r}{2} - \frac{\overline{PP'}}{2}.
\end{equation}
Per la stima di $ \mu ^{\star} $, valore del micrometro per cui il fascio \`e parallelo, si sono interpolati i diametri del fascio in ordinata con i valori indicati dal micrometro: $ \mu ^{\star}$ \`e individuato dall'intersezione delle due rette interpolanti:

\begin{grafico} \centering \begin{tikzpicture}[gnuplot]
%% generated with GNUPLOT 4.6p5 (Lua 5.1; terminal rev. 99, script rev. 100)
%% dom 21 dic 2014 11:00:22 CET
\path (0.000,0.000) rectangle (12.500,8.750);
\gpcolor{color=gp lt color border}
\gpsetlinetype{gp lt border}
\gpsetlinewidth{1.00}
\draw[gp path] (1.688,0.985)--(1.868,0.985);
\draw[gp path] (11.947,0.985)--(11.767,0.985);
\node[gp node right] at (1.504,0.985) { 10};
\draw[gp path] (1.688,1.840)--(1.868,1.840);
\draw[gp path] (11.947,1.840)--(11.767,1.840);
\node[gp node right] at (1.504,1.840) { 10.5};
\draw[gp path] (1.688,2.695)--(1.868,2.695);
\draw[gp path] (11.947,2.695)--(11.767,2.695);
\node[gp node right] at (1.504,2.695) { 11};
\draw[gp path] (1.688,3.550)--(1.868,3.550);
\draw[gp path] (11.947,3.550)--(11.767,3.550);
\node[gp node right] at (1.504,3.550) { 11.5};
\draw[gp path] (1.688,4.405)--(1.868,4.405);
\draw[gp path] (11.947,4.405)--(11.767,4.405);
\node[gp node right] at (1.504,4.405) { 12};
\draw[gp path] (1.688,5.260)--(1.868,5.260);
\draw[gp path] (11.947,5.260)--(11.767,5.260);
\node[gp node right] at (1.504,5.260) { 12.5};
\draw[gp path] (1.688,6.115)--(1.868,6.115);
\draw[gp path] (11.947,6.115)--(11.767,6.115);
\node[gp node right] at (1.504,6.115) { 13};
\draw[gp path] (1.688,6.970)--(1.868,6.970);
\draw[gp path] (11.947,6.970)--(11.767,6.970);
\node[gp node right] at (1.504,6.970) { 13.5};
\draw[gp path] (1.688,7.825)--(1.868,7.825);
\draw[gp path] (11.947,7.825)--(11.767,7.825);
\node[gp node right] at (1.504,7.825) { 14};
\draw[gp path] (1.688,0.985)--(1.688,1.165);
\draw[gp path] (1.688,7.825)--(1.688,7.645);
\node[gp node center] at (1.688,0.677) { 0.38};
\draw[gp path] (2.828,0.985)--(2.828,1.165);
\draw[gp path] (2.828,7.825)--(2.828,7.645);
\node[gp node center] at (2.828,0.677) { 0.4};
\draw[gp path] (3.968,0.985)--(3.968,1.165);
\draw[gp path] (3.968,7.825)--(3.968,7.645);
\node[gp node center] at (3.968,0.677) { 0.42};
\draw[gp path] (5.108,0.985)--(5.108,1.165);
\draw[gp path] (5.108,7.825)--(5.108,7.645);
\node[gp node center] at (5.108,0.677) { 0.44};
\draw[gp path] (6.248,0.985)--(6.248,1.165);
\draw[gp path] (6.248,7.825)--(6.248,7.645);
\node[gp node center] at (6.248,0.677) { 0.46};
\draw[gp path] (7.387,0.985)--(7.387,1.165);
\draw[gp path] (7.387,7.825)--(7.387,7.645);
\node[gp node center] at (7.387,0.677) { 0.48};
\draw[gp path] (8.527,0.985)--(8.527,1.165);
\draw[gp path] (8.527,7.825)--(8.527,7.645);
\node[gp node center] at (8.527,0.677) { 0.5};
\draw[gp path] (9.667,0.985)--(9.667,1.165);
\draw[gp path] (9.667,7.825)--(9.667,7.645);
\node[gp node center] at (9.667,0.677) { 0.52};
\draw[gp path] (10.807,0.985)--(10.807,1.165);
\draw[gp path] (10.807,7.825)--(10.807,7.645);
\node[gp node center] at (10.807,0.677) { 0.54};
\draw[gp path] (11.947,0.985)--(11.947,1.165);
\draw[gp path] (11.947,7.825)--(11.947,7.645);
\node[gp node center] at (11.947,0.677) { 0.56};
\draw[gp path] (1.688,7.825)--(1.688,0.985)--(11.947,0.985)--(11.947,7.825)--cycle;
\node[gp node center,rotate=-270] at (0.246,4.405) {d (mm)};
\node[gp node center] at (6.817,0.215) {p-m0 (mm)};
\node[gp node center] at (6.817,8.287) {Collimazione};
\node[gp node right] at (10.479,7.491) {Dati da vicino};
\gpcolor{color=gp lt color 0}
\gpsetpointsize{4.00}
\gppoint{gp mark 1}{(2.828,4.183)}
\gppoint{gp mark 1}{(5.678,4.285)}
\gppoint{gp mark 1}{(6.248,3.909)}
\gppoint{gp mark 1}{(7.102,3.858)}
\gppoint{gp mark 1}{(8.527,4.166)}
\gppoint{gp mark 1}{(11.377,4.234)}
\gppoint{gp mark 1}{(11.121,7.491)}
\gpcolor{color=gp lt color border}
\node[gp node right] at (10.479,7.183) {Interpolazione da vicino};
\gpcolor{color=gp lt color 1}
\gpsetlinetype{gp lt plot 1}
\draw[gp path] (10.663,7.183)--(11.579,7.183);
\draw[gp path] (2.828,4.084)--(2.914,4.084)--(3.001,4.085)--(3.087,4.085)--(3.173,4.086)%
  --(3.260,4.086)--(3.346,4.087)--(3.432,4.087)--(3.519,4.088)--(3.605,4.088)--(3.691,4.088)%
  --(3.778,4.089)--(3.864,4.089)--(3.951,4.090)--(4.037,4.090)--(4.123,4.091)--(4.210,4.091)%
  --(4.296,4.092)--(4.382,4.092)--(4.469,4.093)--(4.555,4.093)--(4.641,4.093)--(4.728,4.094)%
  --(4.814,4.094)--(4.900,4.095)--(4.987,4.095)--(5.073,4.096)--(5.159,4.096)--(5.246,4.097)%
  --(5.332,4.097)--(5.419,4.098)--(5.505,4.098)--(5.591,4.098)--(5.678,4.099)--(5.764,4.099)%
  --(5.850,4.100)--(5.937,4.100)--(6.023,4.101)--(6.109,4.101)--(6.196,4.102)--(6.282,4.102)%
  --(6.368,4.103)--(6.455,4.103)--(6.541,4.103)--(6.628,4.104)--(6.714,4.104)--(6.800,4.105)%
  --(6.887,4.105)--(6.973,4.106)--(7.059,4.106)--(7.146,4.107)--(7.232,4.107)--(7.318,4.108)%
  --(7.405,4.108)--(7.491,4.108)--(7.577,4.109)--(7.664,4.109)--(7.750,4.110)--(7.836,4.110)%
  --(7.923,4.111)--(8.009,4.111)--(8.096,4.112)--(8.182,4.112)--(8.268,4.113)--(8.355,4.113)%
  --(8.441,4.113)--(8.527,4.114)--(8.614,4.114)--(8.700,4.115)--(8.786,4.115)--(8.873,4.116)%
  --(8.959,4.116)--(9.045,4.117)--(9.132,4.117)--(9.218,4.118)--(9.305,4.118)--(9.391,4.118)%
  --(9.477,4.119)--(9.564,4.119)--(9.650,4.120)--(9.736,4.120)--(9.823,4.121)--(9.909,4.121)%
  --(9.995,4.122)--(10.082,4.122)--(10.168,4.123)--(10.254,4.123)--(10.341,4.123)--(10.427,4.124)%
  --(10.514,4.124)--(10.600,4.125)--(10.686,4.125)--(10.773,4.126)--(10.859,4.126)--(10.945,4.127)%
  --(11.032,4.127)--(11.118,4.128)--(11.204,4.128)--(11.291,4.128)--(11.377,4.129);
\gpcolor{color=gp lt color border}
\node[gp node right] at (10.479,6.875) {da lontano};
\gpcolor{color=gp lt color 2}
\gppoint{gp mark 3}{(2.828,0.985)}
\gppoint{gp mark 3}{(5.678,2.781)}
\gppoint{gp mark 3}{(6.248,3.926)}
\gppoint{gp mark 3}{(7.102,4.234)}
\gppoint{gp mark 3}{(8.527,5.328)}
\gppoint{gp mark 3}{(11.377,7.261)}
\gppoint{gp mark 3}{(11.121,6.875)}
\gpcolor{color=gp lt color border}
\node[gp node right] at (10.479,6.567) {Interpolazione da lontano};
\gpcolor{color=gp lt color 3}
\gpsetlinetype{gp lt plot 3}
\draw[gp path] (10.663,6.567)--(11.579,6.567);
\draw[gp path] (2.828,1.019)--(2.914,1.083)--(3.001,1.147)--(3.087,1.211)--(3.173,1.275)%
  --(3.260,1.339)--(3.346,1.403)--(3.432,1.467)--(3.519,1.531)--(3.605,1.595)--(3.691,1.660)%
  --(3.778,1.724)--(3.864,1.788)--(3.951,1.852)--(4.037,1.916)--(4.123,1.980)--(4.210,2.044)%
  --(4.296,2.108)--(4.382,2.172)--(4.469,2.236)--(4.555,2.301)--(4.641,2.365)--(4.728,2.429)%
  --(4.814,2.493)--(4.900,2.557)--(4.987,2.621)--(5.073,2.685)--(5.159,2.749)--(5.246,2.813)%
  --(5.332,2.877)--(5.419,2.942)--(5.505,3.006)--(5.591,3.070)--(5.678,3.134)--(5.764,3.198)%
  --(5.850,3.262)--(5.937,3.326)--(6.023,3.390)--(6.109,3.454)--(6.196,3.519)--(6.282,3.583)%
  --(6.368,3.647)--(6.455,3.711)--(6.541,3.775)--(6.628,3.839)--(6.714,3.903)--(6.800,3.967)%
  --(6.887,4.031)--(6.973,4.095)--(7.059,4.160)--(7.146,4.224)--(7.232,4.288)--(7.318,4.352)%
  --(7.405,4.416)--(7.491,4.480)--(7.577,4.544)--(7.664,4.608)--(7.750,4.672)--(7.836,4.736)%
  --(7.923,4.801)--(8.009,4.865)--(8.096,4.929)--(8.182,4.993)--(8.268,5.057)--(8.355,5.121)%
  --(8.441,5.185)--(8.527,5.249)--(8.614,5.313)--(8.700,5.377)--(8.786,5.442)--(8.873,5.506)%
  --(8.959,5.570)--(9.045,5.634)--(9.132,5.698)--(9.218,5.762)--(9.305,5.826)--(9.391,5.890)%
  --(9.477,5.954)--(9.564,6.018)--(9.650,6.083)--(9.736,6.147)--(9.823,6.211)--(9.909,6.275)%
  --(9.995,6.339)--(10.082,6.403)--(10.168,6.467)--(10.254,6.531)--(10.341,6.595)--(10.427,6.659)%
  --(10.514,6.724)--(10.600,6.788)--(10.686,6.852)--(10.773,6.916)--(10.859,6.980)--(10.945,7.044)%
  --(11.032,7.108)--(11.118,7.172)--(11.204,7.236)--(11.291,7.300)--(11.377,7.365);
\gpcolor{color=gp lt color border}
\gpsetlinetype{gp lt border}
\draw[gp path] (1.688,7.825)--(1.688,0.985)--(11.947,0.985)--(11.947,7.825)--cycle;
%% coordinates of the plot area
\gpdefrectangularnode{gp plot 1}{\pgfpoint{1.688cm}{0.985cm}}{\pgfpoint{11.947cm}{7.825cm}}
\end{tikzpicture}
%% gnuplot variables
 \caption{Interpolazione lineare} \label{gr:01_graph_1.tex} \end{grafico}

\[ \mu ^{\star} = x_{\mathrm{intersezione}} = \frac{a - a'}{b' - b}  = 4.73 \mm , \]
da cui, grazie alla formula di propagazione quadratica, si ottiene, considerando $\left(a, b\right)$, $\left(a', b'\right)$ rispettivamente correlati e le rette tra loro indipendenti,
% split va dentro a un'equazione
\begin{equation*}
\begin{split} % align numera tutte le righe e se devo toglierne una devo usare \nonumber alla fine della riga, align* nessuna. Meglio split che numera solo una volta
	\sigma^2_{\mu ^{\star}\left(a, a', b, b'\right)}  &= \left(\left.\frac{\partial F}{\partial a}\right|_{x_{\mathrm{int}}}\right)^2 \cdot \sigma^2_{a} + \left(\left.\frac{\partial F}{\partial a'}\right|_{x_{\mathrm{int}}}\right)^2 \cdot \sigma^2_{a'} + \left(\left.\frac{\partial F}{\partial b}\right|_{x_{\mathrm{int}}}\right)^2 \cdot \sigma^2_{b} +\\
								&+ \left(\left.\frac{\partial F}{\partial b'}\right|_{x_{\mathrm{int}}}\right)^2 \cdot \sigma^2_{b} + 2\left(\left.\frac{\partial F}{\partial a}\right|_{x_{\mathrm{int}}}\right)\left(\left.\frac{\partial F}{\partial b}\right|_{x_{\mathrm{int}}}\right) \cdot \cov\left(a, b\right) + \\
								&+ 2\left(\left.\frac{\partial F}{\partial a'}\right|_{x_{\mathrm{int}}}\right)\left(\left.\frac{\partial F}{\partial b'}\right|_{x_{\mathrm{int}}}\right)^2 \cdot \cov \left(a', b'\right)
\end{split}
\end{equation*}
che, sotto radice quadrata, d\`a l'incertezza per $ \mu ^{\star} $, considerandolo distribuito normalmente.
Svolgendo i calcoli, si trova
\begin{equation}
\begin{split}
\sigma^2_{\mu ^{\star}\left(a, a', b, b'\right)}  &= \left(\left.\frac{1}{b'- b}\right|_{x_{\mathrm{int}}}\right)^2 \cdot \left(\sigma^2_{a} + \sigma^2_{a'}\right) + \left(\left.\frac{a - a'}{\left(b' - b\right)^2}\right|_{x_{\mathrm{int}}}\right)^2 \left(\sigma^2_{b} + \sigma^2_{b'}\right) + \\
							&+ 2 \left(\left.\frac{1}{b'- b}\right|_{x_{\mathrm{int}}}\right) \left(\left.\frac{a - a'}{\left(b' - b\right)^2}\right|_{x_{\mathrm{int}}}\right) \big(\cov\left(a, b\right) + \cov(a', b') \big).
\end{split}
\end{equation}
Calcoliamo le covarianze:
\[ \cov\left(a, b\right) = -\frac{\sum_{i} x_i}{\Delta}\sigma_y^2 \] 
dove $\Delta$ \`e il parametro di interpolazione lineare; vale lo stesso per a' e b', con le adeguate (x, y).
\[ \cov(a, b) = -0.501 , \cov(a', b') = -0.895 . \]
L'incertezza cos\`i calcolata risulta di $0.0003\cm$, molto bassa a causa della precisione micrometrica, migliorata grazie al fit lineare. Tuttavia si ritiene pi\`u corretto considerarla non pi\`u bassa dell'incertezza strumentale. L'intersezione \`e quindi stimata come
\[ \mu ^{\star} =  \left(0.473 \pm 0.001\right) \cm. \] %errore un po' bassino %non piu' --Gab

Per rendere visivamente apprezzabile l'incertezza sulle rette, \`e stato creato un grafico (Grafico 2) contenente le rette tracciate per i valori estremali della quota e del coefficiente angolare.
\begin{grafico} \centering \begin{tikzpicture}[gnuplot]
%% generated with GNUPLOT 4.6p5 (Lua 5.1; terminal rev. 99, script rev. 100)
%% dom 21 dic 2014 11:00:22 CET
\path (0.000,0.000) rectangle (12.500,8.750);
\gpcolor{color=gp lt color border}
\gpsetlinetype{gp lt border}
\gpsetlinewidth{1.00}
\draw[gp path] (1.504,0.985)--(1.684,0.985);
\draw[gp path] (11.947,0.985)--(11.767,0.985);
\node[gp node right] at (1.320,0.985) {-300};
\draw[gp path] (1.504,2.125)--(1.684,2.125);
\draw[gp path] (11.947,2.125)--(11.767,2.125);
\node[gp node right] at (1.320,2.125) {-200};
\draw[gp path] (1.504,3.265)--(1.684,3.265);
\draw[gp path] (11.947,3.265)--(11.767,3.265);
\node[gp node right] at (1.320,3.265) {-100};
\draw[gp path] (1.504,4.405)--(1.684,4.405);
\draw[gp path] (11.947,4.405)--(11.767,4.405);
\node[gp node right] at (1.320,4.405) { 0};
\draw[gp path] (1.504,5.545)--(1.684,5.545);
\draw[gp path] (11.947,5.545)--(11.767,5.545);
\node[gp node right] at (1.320,5.545) { 100};
\draw[gp path] (1.504,6.685)--(1.684,6.685);
\draw[gp path] (11.947,6.685)--(11.767,6.685);
\node[gp node right] at (1.320,6.685) { 200};
\draw[gp path] (1.504,7.825)--(1.684,7.825);
\draw[gp path] (11.947,7.825)--(11.767,7.825);
\node[gp node right] at (1.320,7.825) { 300};
\draw[gp path] (1.504,0.985)--(1.504,1.165);
\draw[gp path] (1.504,7.825)--(1.504,7.645);
\node[gp node center] at (1.504,0.677) {-10};
\draw[gp path] (4.115,0.985)--(4.115,1.165);
\draw[gp path] (4.115,7.825)--(4.115,7.645);
\node[gp node center] at (4.115,0.677) {-5};
\draw[gp path] (6.726,0.985)--(6.726,1.165);
\draw[gp path] (6.726,7.825)--(6.726,7.645);
\node[gp node center] at (6.726,0.677) { 0};
\draw[gp path] (9.336,0.985)--(9.336,1.165);
\draw[gp path] (9.336,7.825)--(9.336,7.645);
\node[gp node center] at (9.336,0.677) { 5};
\draw[gp path] (11.947,0.985)--(11.947,1.165);
\draw[gp path] (11.947,7.825)--(11.947,7.645);
\node[gp node center] at (11.947,0.677) { 10};
\draw[gp path] (1.504,7.825)--(1.504,0.985)--(11.947,0.985)--(11.947,7.825)--cycle;
\node[gp node center,rotate=-270] at (0.246,4.405) {d (mm)};
\node[gp node center] at (6.725,0.215) {p-m0 (mm)};
\node[gp node center] at (6.725,8.287) {Collimazione};
\node[gp node right] at (10.479,7.491) {Interpolazione da vicino};
\gpcolor{rgb color={1.000,0.000,0.000}}
\gpsetlinetype{gp lt plot 0}
\draw[gp path] (10.663,7.491)--(11.579,7.491);
\draw[gp path] (1.504,4.519)--(1.609,4.519)--(1.715,4.520)--(1.820,4.520)--(1.926,4.520)%
  --(2.031,4.521)--(2.137,4.521)--(2.242,4.522)--(2.348,4.522)--(2.453,4.523)--(2.559,4.523)%
  --(2.664,4.523)--(2.770,4.524)--(2.875,4.524)--(2.981,4.525)--(3.086,4.525)--(3.192,4.525)%
  --(3.297,4.526)--(3.403,4.526)--(3.508,4.527)--(3.614,4.527)--(3.719,4.527)--(3.825,4.528)%
  --(3.930,4.528)--(4.036,4.529)--(4.141,4.529)--(4.247,4.529)--(4.352,4.530)--(4.458,4.530)%
  --(4.563,4.531)--(4.669,4.531)--(4.774,4.531)--(4.880,4.532)--(4.985,4.532)--(5.090,4.533)%
  --(5.196,4.533)--(5.301,4.533)--(5.407,4.534)--(5.512,4.534)--(5.618,4.535)--(5.723,4.535)%
  --(5.829,4.535)--(5.934,4.536)--(6.040,4.536)--(6.145,4.537)--(6.251,4.537)--(6.356,4.537)%
  --(6.462,4.538)--(6.567,4.538)--(6.673,4.539)--(6.778,4.539)--(6.884,4.539)--(6.989,4.540)%
  --(7.095,4.540)--(7.200,4.541)--(7.306,4.541)--(7.411,4.541)--(7.517,4.542)--(7.622,4.542)%
  --(7.728,4.543)--(7.833,4.543)--(7.939,4.544)--(8.044,4.544)--(8.150,4.544)--(8.255,4.545)%
  --(8.361,4.545)--(8.466,4.546)--(8.571,4.546)--(8.677,4.546)--(8.782,4.547)--(8.888,4.547)%
  --(8.993,4.548)--(9.099,4.548)--(9.204,4.548)--(9.310,4.549)--(9.415,4.549)--(9.521,4.550)%
  --(9.626,4.550)--(9.732,4.550)--(9.837,4.551)--(9.943,4.551)--(10.048,4.552)--(10.154,4.552)%
  --(10.259,4.552)--(10.365,4.553)--(10.470,4.553)--(10.576,4.554)--(10.681,4.554)--(10.787,4.554)%
  --(10.892,4.555)--(10.998,4.555)--(11.103,4.556)--(11.209,4.556)--(11.314,4.556)--(11.420,4.557)%
  --(11.525,4.557)--(11.631,4.558)--(11.736,4.558)--(11.842,4.558)--(11.947,4.559);
\gpcolor{color=gp lt color border}
\node[gp node right] at (10.479,7.183) { };
\gpcolor{rgb color={1.000,0.537,0.000}}
\gpsetlinetype{gp lt plot 1}
\draw[gp path] (10.663,7.183)--(11.579,7.183);
\draw[gp path] (1.504,4.407)--(1.609,4.410)--(1.715,4.413)--(1.820,4.415)--(1.926,4.418)%
  --(2.031,4.421)--(2.137,4.424)--(2.242,4.427)--(2.348,4.429)--(2.453,4.432)--(2.559,4.435)%
  --(2.664,4.438)--(2.770,4.440)--(2.875,4.443)--(2.981,4.446)--(3.086,4.449)--(3.192,4.451)%
  --(3.297,4.454)--(3.403,4.457)--(3.508,4.460)--(3.614,4.463)--(3.719,4.465)--(3.825,4.468)%
  --(3.930,4.471)--(4.036,4.474)--(4.141,4.476)--(4.247,4.479)--(4.352,4.482)--(4.458,4.485)%
  --(4.563,4.488)--(4.669,4.490)--(4.774,4.493)--(4.880,4.496)--(4.985,4.499)--(5.090,4.501)%
  --(5.196,4.504)--(5.301,4.507)--(5.407,4.510)--(5.512,4.513)--(5.618,4.515)--(5.723,4.518)%
  --(5.829,4.521)--(5.934,4.524)--(6.040,4.526)--(6.145,4.529)--(6.251,4.532)--(6.356,4.535)%
  --(6.462,4.537)--(6.567,4.540)--(6.673,4.543)--(6.778,4.546)--(6.884,4.549)--(6.989,4.551)%
  --(7.095,4.554)--(7.200,4.557)--(7.306,4.560)--(7.411,4.562)--(7.517,4.565)--(7.622,4.568)%
  --(7.728,4.571)--(7.833,4.574)--(7.939,4.576)--(8.044,4.579)--(8.150,4.582)--(8.255,4.585)%
  --(8.361,4.587)--(8.466,4.590)--(8.571,4.593)--(8.677,4.596)--(8.782,4.599)--(8.888,4.601)%
  --(8.993,4.604)--(9.099,4.607)--(9.204,4.610)--(9.310,4.612)--(9.415,4.615)--(9.521,4.618)%
  --(9.626,4.621)--(9.732,4.623)--(9.837,4.626)--(9.943,4.629)--(10.048,4.632)--(10.154,4.635)%
  --(10.259,4.637)--(10.365,4.640)--(10.470,4.643)--(10.576,4.646)--(10.681,4.648)--(10.787,4.651)%
  --(10.892,4.654)--(10.998,4.657)--(11.103,4.660)--(11.209,4.662)--(11.314,4.665)--(11.420,4.668)%
  --(11.525,4.671)--(11.631,4.673)--(11.736,4.676)--(11.842,4.679)--(11.947,4.682);
\gpcolor{color=gp lt color border}
\node[gp node right] at (10.479,6.875) { };
\gpcolor{rgb color={1.000,0.537,0.000}}
\gpsetlinetype{gp lt plot 2}
\draw[gp path] (10.663,6.875)--(11.579,6.875);
\draw[gp path] (1.504,4.396)--(1.609,4.399)--(1.715,4.402)--(1.820,4.404)--(1.926,4.407)%
  --(2.031,4.410)--(2.137,4.413)--(2.242,4.415)--(2.348,4.418)--(2.453,4.421)--(2.559,4.424)%
  --(2.664,4.426)--(2.770,4.429)--(2.875,4.432)--(2.981,4.435)--(3.086,4.438)--(3.192,4.440)%
  --(3.297,4.443)--(3.403,4.446)--(3.508,4.449)--(3.614,4.451)--(3.719,4.454)--(3.825,4.457)%
  --(3.930,4.460)--(4.036,4.463)--(4.141,4.465)--(4.247,4.468)--(4.352,4.471)--(4.458,4.474)%
  --(4.563,4.476)--(4.669,4.479)--(4.774,4.482)--(4.880,4.485)--(4.985,4.488)--(5.090,4.490)%
  --(5.196,4.493)--(5.301,4.496)--(5.407,4.499)--(5.512,4.501)--(5.618,4.504)--(5.723,4.507)%
  --(5.829,4.510)--(5.934,4.512)--(6.040,4.515)--(6.145,4.518)--(6.251,4.521)--(6.356,4.524)%
  --(6.462,4.526)--(6.567,4.529)--(6.673,4.532)--(6.778,4.535)--(6.884,4.537)--(6.989,4.540)%
  --(7.095,4.543)--(7.200,4.546)--(7.306,4.549)--(7.411,4.551)--(7.517,4.554)--(7.622,4.557)%
  --(7.728,4.560)--(7.833,4.562)--(7.939,4.565)--(8.044,4.568)--(8.150,4.571)--(8.255,4.574)%
  --(8.361,4.576)--(8.466,4.579)--(8.571,4.582)--(8.677,4.585)--(8.782,4.587)--(8.888,4.590)%
  --(8.993,4.593)--(9.099,4.596)--(9.204,4.598)--(9.310,4.601)--(9.415,4.604)--(9.521,4.607)%
  --(9.626,4.610)--(9.732,4.612)--(9.837,4.615)--(9.943,4.618)--(10.048,4.621)--(10.154,4.623)%
  --(10.259,4.626)--(10.365,4.629)--(10.470,4.632)--(10.576,4.635)--(10.681,4.637)--(10.787,4.640)%
  --(10.892,4.643)--(10.998,4.646)--(11.103,4.648)--(11.209,4.651)--(11.314,4.654)--(11.420,4.657)%
  --(11.525,4.660)--(11.631,4.662)--(11.736,4.665)--(11.842,4.668)--(11.947,4.671);
\gpcolor{color=gp lt color border}
\node[gp node right] at (10.479,6.567) { };
\gpcolor{rgb color={1.000,0.537,0.000}}
\gpsetlinetype{gp lt plot 3}
\draw[gp path] (10.663,6.567)--(11.579,6.567);
\draw[gp path] (1.504,4.642)--(1.609,4.640)--(1.715,4.638)--(1.820,4.636)--(1.926,4.634)%
  --(2.031,4.632)--(2.137,4.630)--(2.242,4.628)--(2.348,4.626)--(2.453,4.624)--(2.559,4.622)%
  --(2.664,4.620)--(2.770,4.618)--(2.875,4.616)--(2.981,4.614)--(3.086,4.612)--(3.192,4.610)%
  --(3.297,4.608)--(3.403,4.606)--(3.508,4.604)--(3.614,4.602)--(3.719,4.600)--(3.825,4.599)%
  --(3.930,4.597)--(4.036,4.595)--(4.141,4.593)--(4.247,4.591)--(4.352,4.589)--(4.458,4.587)%
  --(4.563,4.585)--(4.669,4.583)--(4.774,4.581)--(4.880,4.579)--(4.985,4.577)--(5.090,4.575)%
  --(5.196,4.573)--(5.301,4.571)--(5.407,4.569)--(5.512,4.567)--(5.618,4.565)--(5.723,4.563)%
  --(5.829,4.561)--(5.934,4.559)--(6.040,4.557)--(6.145,4.555)--(6.251,4.553)--(6.356,4.551)%
  --(6.462,4.549)--(6.567,4.547)--(6.673,4.545)--(6.778,4.543)--(6.884,4.541)--(6.989,4.540)%
  --(7.095,4.538)--(7.200,4.536)--(7.306,4.534)--(7.411,4.532)--(7.517,4.530)--(7.622,4.528)%
  --(7.728,4.526)--(7.833,4.524)--(7.939,4.522)--(8.044,4.520)--(8.150,4.518)--(8.255,4.516)%
  --(8.361,4.514)--(8.466,4.512)--(8.571,4.510)--(8.677,4.508)--(8.782,4.506)--(8.888,4.504)%
  --(8.993,4.502)--(9.099,4.500)--(9.204,4.498)--(9.310,4.496)--(9.415,4.494)--(9.521,4.492)%
  --(9.626,4.490)--(9.732,4.488)--(9.837,4.486)--(9.943,4.484)--(10.048,4.482)--(10.154,4.481)%
  --(10.259,4.479)--(10.365,4.477)--(10.470,4.475)--(10.576,4.473)--(10.681,4.471)--(10.787,4.469)%
  --(10.892,4.467)--(10.998,4.465)--(11.103,4.463)--(11.209,4.461)--(11.314,4.459)--(11.420,4.457)%
  --(11.525,4.455)--(11.631,4.453)--(11.736,4.451)--(11.842,4.449)--(11.947,4.447);
\gpcolor{color=gp lt color border}
\node[gp node right] at (10.479,6.259) { };
\gpcolor{rgb color={1.000,0.537,0.000}}
\gpsetlinetype{gp lt plot 4}
\draw[gp path] (10.663,6.259)--(11.579,6.259);
\draw[gp path] (1.504,4.631)--(1.609,4.629)--(1.715,4.627)--(1.820,4.625)--(1.926,4.623)%
  --(2.031,4.621)--(2.137,4.619)--(2.242,4.617)--(2.348,4.615)--(2.453,4.613)--(2.559,4.611)%
  --(2.664,4.609)--(2.770,4.607)--(2.875,4.605)--(2.981,4.603)--(3.086,4.601)--(3.192,4.599)%
  --(3.297,4.597)--(3.403,4.595)--(3.508,4.593)--(3.614,4.591)--(3.719,4.589)--(3.825,4.587)%
  --(3.930,4.585)--(4.036,4.583)--(4.141,4.581)--(4.247,4.580)--(4.352,4.578)--(4.458,4.576)%
  --(4.563,4.574)--(4.669,4.572)--(4.774,4.570)--(4.880,4.568)--(4.985,4.566)--(5.090,4.564)%
  --(5.196,4.562)--(5.301,4.560)--(5.407,4.558)--(5.512,4.556)--(5.618,4.554)--(5.723,4.552)%
  --(5.829,4.550)--(5.934,4.548)--(6.040,4.546)--(6.145,4.544)--(6.251,4.542)--(6.356,4.540)%
  --(6.462,4.538)--(6.567,4.536)--(6.673,4.534)--(6.778,4.532)--(6.884,4.530)--(6.989,4.528)%
  --(7.095,4.526)--(7.200,4.524)--(7.306,4.522)--(7.411,4.521)--(7.517,4.519)--(7.622,4.517)%
  --(7.728,4.515)--(7.833,4.513)--(7.939,4.511)--(8.044,4.509)--(8.150,4.507)--(8.255,4.505)%
  --(8.361,4.503)--(8.466,4.501)--(8.571,4.499)--(8.677,4.497)--(8.782,4.495)--(8.888,4.493)%
  --(8.993,4.491)--(9.099,4.489)--(9.204,4.487)--(9.310,4.485)--(9.415,4.483)--(9.521,4.481)%
  --(9.626,4.479)--(9.732,4.477)--(9.837,4.475)--(9.943,4.473)--(10.048,4.471)--(10.154,4.469)%
  --(10.259,4.467)--(10.365,4.465)--(10.470,4.463)--(10.576,4.462)--(10.681,4.460)--(10.787,4.458)%
  --(10.892,4.456)--(10.998,4.454)--(11.103,4.452)--(11.209,4.450)--(11.314,4.448)--(11.420,4.446)%
  --(11.525,4.444)--(11.631,4.442)--(11.736,4.440)--(11.842,4.438)--(11.947,4.436);
\gpcolor{color=gp lt color border}
\node[gp node right] at (10.479,5.951) {Interpolazione da lontano};
\gpcolor{rgb color={0.102,0.000,1.000}}
\gpsetlinetype{gp lt plot 5}
\draw[gp path] (10.663,5.951)--(11.579,5.951);
\draw[gp path] (1.504,1.586)--(1.609,1.643)--(1.715,1.700)--(1.820,1.757)--(1.926,1.814)%
  --(2.031,1.871)--(2.137,1.928)--(2.242,1.985)--(2.348,2.042)--(2.453,2.099)--(2.559,2.156)%
  --(2.664,2.213)--(2.770,2.270)--(2.875,2.327)--(2.981,2.384)--(3.086,2.441)--(3.192,2.498)%
  --(3.297,2.555)--(3.403,2.612)--(3.508,2.669)--(3.614,2.726)--(3.719,2.783)--(3.825,2.839)%
  --(3.930,2.896)--(4.036,2.953)--(4.141,3.010)--(4.247,3.067)--(4.352,3.124)--(4.458,3.181)%
  --(4.563,3.238)--(4.669,3.295)--(4.774,3.352)--(4.880,3.409)--(4.985,3.466)--(5.090,3.523)%
  --(5.196,3.580)--(5.301,3.637)--(5.407,3.694)--(5.512,3.751)--(5.618,3.808)--(5.723,3.865)%
  --(5.829,3.922)--(5.934,3.979)--(6.040,4.036)--(6.145,4.093)--(6.251,4.150)--(6.356,4.207)%
  --(6.462,4.264)--(6.567,4.321)--(6.673,4.378)--(6.778,4.435)--(6.884,4.492)--(6.989,4.549)%
  --(7.095,4.606)--(7.200,4.663)--(7.306,4.720)--(7.411,4.777)--(7.517,4.834)--(7.622,4.891)%
  --(7.728,4.948)--(7.833,5.005)--(7.939,5.062)--(8.044,5.119)--(8.150,5.176)--(8.255,5.233)%
  --(8.361,5.290)--(8.466,5.347)--(8.571,5.404)--(8.677,5.461)--(8.782,5.517)--(8.888,5.574)%
  --(8.993,5.631)--(9.099,5.688)--(9.204,5.745)--(9.310,5.802)--(9.415,5.859)--(9.521,5.916)%
  --(9.626,5.973)--(9.732,6.030)--(9.837,6.087)--(9.943,6.144)--(10.048,6.201)--(10.154,6.258)%
  --(10.259,6.315)--(10.365,6.372)--(10.470,6.429)--(10.576,6.486)--(10.681,6.543)--(10.787,6.600)%
  --(10.892,6.657)--(10.998,6.714)--(11.103,6.771)--(11.209,6.828)--(11.314,6.885)--(11.420,6.942)%
  --(11.525,6.999)--(11.631,7.056)--(11.736,7.113)--(11.842,7.170)--(11.947,7.227);
\gpcolor{color=gp lt color border}
\node[gp node right] at (10.479,5.643) { };
\gpcolor{rgb color={0.294,0.463,1.000}}
\gpsetlinetype{gp lt plot 6}
\draw[gp path] (10.663,5.643)--(11.579,5.643);
\draw[gp path] (1.504,1.436)--(1.609,1.497)--(1.715,1.557)--(1.820,1.617)--(1.926,1.677)%
  --(2.031,1.737)--(2.137,1.797)--(2.242,1.858)--(2.348,1.918)--(2.453,1.978)--(2.559,2.038)%
  --(2.664,2.098)--(2.770,2.158)--(2.875,2.218)--(2.981,2.279)--(3.086,2.339)--(3.192,2.399)%
  --(3.297,2.459)--(3.403,2.519)--(3.508,2.579)--(3.614,2.639)--(3.719,2.700)--(3.825,2.760)%
  --(3.930,2.820)--(4.036,2.880)--(4.141,2.940)--(4.247,3.000)--(4.352,3.060)--(4.458,3.121)%
  --(4.563,3.181)--(4.669,3.241)--(4.774,3.301)--(4.880,3.361)--(4.985,3.421)--(5.090,3.482)%
  --(5.196,3.542)--(5.301,3.602)--(5.407,3.662)--(5.512,3.722)--(5.618,3.782)--(5.723,3.842)%
  --(5.829,3.903)--(5.934,3.963)--(6.040,4.023)--(6.145,4.083)--(6.251,4.143)--(6.356,4.203)%
  --(6.462,4.263)--(6.567,4.324)--(6.673,4.384)--(6.778,4.444)--(6.884,4.504)--(6.989,4.564)%
  --(7.095,4.624)--(7.200,4.685)--(7.306,4.745)--(7.411,4.805)--(7.517,4.865)--(7.622,4.925)%
  --(7.728,4.985)--(7.833,5.045)--(7.939,5.106)--(8.044,5.166)--(8.150,5.226)--(8.255,5.286)%
  --(8.361,5.346)--(8.466,5.406)--(8.571,5.466)--(8.677,5.527)--(8.782,5.587)--(8.888,5.647)%
  --(8.993,5.707)--(9.099,5.767)--(9.204,5.827)--(9.310,5.888)--(9.415,5.948)--(9.521,6.008)%
  --(9.626,6.068)--(9.732,6.128)--(9.837,6.188)--(9.943,6.248)--(10.048,6.309)--(10.154,6.369)%
  --(10.259,6.429)--(10.365,6.489)--(10.470,6.549)--(10.576,6.609)--(10.681,6.669)--(10.787,6.730)%
  --(10.892,6.790)--(10.998,6.850)--(11.103,6.910)--(11.209,6.970)--(11.314,7.030)--(11.420,7.091)%
  --(11.525,7.151)--(11.631,7.211)--(11.736,7.271)--(11.842,7.331)--(11.947,7.391);
\gpcolor{color=gp lt color border}
\node[gp node right] at (10.479,5.335) { };
\gpcolor{rgb color={0.294,0.463,1.000}}
\gpsetlinetype{gp lt plot 7}
\draw[gp path] (10.663,5.335)--(11.579,5.335);
\draw[gp path] (1.504,1.422)--(1.609,1.482)--(1.715,1.542)--(1.820,1.602)--(1.926,1.662)%
  --(2.031,1.722)--(2.137,1.782)--(2.242,1.843)--(2.348,1.903)--(2.453,1.963)--(2.559,2.023)%
  --(2.664,2.083)--(2.770,2.143)--(2.875,2.204)--(2.981,2.264)--(3.086,2.324)--(3.192,2.384)%
  --(3.297,2.444)--(3.403,2.504)--(3.508,2.564)--(3.614,2.625)--(3.719,2.685)--(3.825,2.745)%
  --(3.930,2.805)--(4.036,2.865)--(4.141,2.925)--(4.247,2.985)--(4.352,3.046)--(4.458,3.106)%
  --(4.563,3.166)--(4.669,3.226)--(4.774,3.286)--(4.880,3.346)--(4.985,3.406)--(5.090,3.467)%
  --(5.196,3.527)--(5.301,3.587)--(5.407,3.647)--(5.512,3.707)--(5.618,3.767)--(5.723,3.828)%
  --(5.829,3.888)--(5.934,3.948)--(6.040,4.008)--(6.145,4.068)--(6.251,4.128)--(6.356,4.188)%
  --(6.462,4.249)--(6.567,4.309)--(6.673,4.369)--(6.778,4.429)--(6.884,4.489)--(6.989,4.549)%
  --(7.095,4.609)--(7.200,4.670)--(7.306,4.730)--(7.411,4.790)--(7.517,4.850)--(7.622,4.910)%
  --(7.728,4.970)--(7.833,5.031)--(7.939,5.091)--(8.044,5.151)--(8.150,5.211)--(8.255,5.271)%
  --(8.361,5.331)--(8.466,5.391)--(8.571,5.452)--(8.677,5.512)--(8.782,5.572)--(8.888,5.632)%
  --(8.993,5.692)--(9.099,5.752)--(9.204,5.812)--(9.310,5.873)--(9.415,5.933)--(9.521,5.993)%
  --(9.626,6.053)--(9.732,6.113)--(9.837,6.173)--(9.943,6.234)--(10.048,6.294)--(10.154,6.354)%
  --(10.259,6.414)--(10.365,6.474)--(10.470,6.534)--(10.576,6.594)--(10.681,6.655)--(10.787,6.715)%
  --(10.892,6.775)--(10.998,6.835)--(11.103,6.895)--(11.209,6.955)--(11.314,7.015)--(11.420,7.076)%
  --(11.525,7.136)--(11.631,7.196)--(11.736,7.256)--(11.842,7.316)--(11.947,7.376);
\gpcolor{color=gp lt color border}
\node[gp node right] at (10.479,5.027) { };
\gpcolor{rgb color={0.294,0.463,1.000}}
\gpsetlinetype{gp lt plot 0}
\draw[gp path] (10.663,5.027)--(11.579,5.027);
\draw[gp path] (1.504,1.750)--(1.609,1.804)--(1.715,1.858)--(1.820,1.912)--(1.926,1.966)%
  --(2.031,2.019)--(2.137,2.073)--(2.242,2.127)--(2.348,2.181)--(2.453,2.235)--(2.559,2.288)%
  --(2.664,2.342)--(2.770,2.396)--(2.875,2.450)--(2.981,2.504)--(3.086,2.557)--(3.192,2.611)%
  --(3.297,2.665)--(3.403,2.719)--(3.508,2.773)--(3.614,2.826)--(3.719,2.880)--(3.825,2.934)%
  --(3.930,2.988)--(4.036,3.042)--(4.141,3.096)--(4.247,3.149)--(4.352,3.203)--(4.458,3.257)%
  --(4.563,3.311)--(4.669,3.365)--(4.774,3.418)--(4.880,3.472)--(4.985,3.526)--(5.090,3.580)%
  --(5.196,3.634)--(5.301,3.687)--(5.407,3.741)--(5.512,3.795)--(5.618,3.849)--(5.723,3.903)%
  --(5.829,3.956)--(5.934,4.010)--(6.040,4.064)--(6.145,4.118)--(6.251,4.172)--(6.356,4.226)%
  --(6.462,4.279)--(6.567,4.333)--(6.673,4.387)--(6.778,4.441)--(6.884,4.495)--(6.989,4.548)%
  --(7.095,4.602)--(7.200,4.656)--(7.306,4.710)--(7.411,4.764)--(7.517,4.817)--(7.622,4.871)%
  --(7.728,4.925)--(7.833,4.979)--(7.939,5.033)--(8.044,5.086)--(8.150,5.140)--(8.255,5.194)%
  --(8.361,5.248)--(8.466,5.302)--(8.571,5.356)--(8.677,5.409)--(8.782,5.463)--(8.888,5.517)%
  --(8.993,5.571)--(9.099,5.625)--(9.204,5.678)--(9.310,5.732)--(9.415,5.786)--(9.521,5.840)%
  --(9.626,5.894)--(9.732,5.947)--(9.837,6.001)--(9.943,6.055)--(10.048,6.109)--(10.154,6.163)%
  --(10.259,6.216)--(10.365,6.270)--(10.470,6.324)--(10.576,6.378)--(10.681,6.432)--(10.787,6.485)%
  --(10.892,6.539)--(10.998,6.593)--(11.103,6.647)--(11.209,6.701)--(11.314,6.755)--(11.420,6.808)%
  --(11.525,6.862)--(11.631,6.916)--(11.736,6.970)--(11.842,7.024)--(11.947,7.077);
\gpcolor{color=gp lt color border}
\node[gp node right] at (10.479,4.719) { };
\gpcolor{rgb color={0.294,0.463,1.000}}
\gpsetlinetype{gp lt plot 1}
\draw[gp path] (10.663,4.719)--(11.579,4.719);
\draw[gp path] (1.504,1.735)--(1.609,1.789)--(1.715,1.843)--(1.820,1.897)--(1.926,1.951)%
  --(2.031,2.004)--(2.137,2.058)--(2.242,2.112)--(2.348,2.166)--(2.453,2.220)--(2.559,2.274)%
  --(2.664,2.327)--(2.770,2.381)--(2.875,2.435)--(2.981,2.489)--(3.086,2.543)--(3.192,2.596)%
  --(3.297,2.650)--(3.403,2.704)--(3.508,2.758)--(3.614,2.812)--(3.719,2.865)--(3.825,2.919)%
  --(3.930,2.973)--(4.036,3.027)--(4.141,3.081)--(4.247,3.134)--(4.352,3.188)--(4.458,3.242)%
  --(4.563,3.296)--(4.669,3.350)--(4.774,3.403)--(4.880,3.457)--(4.985,3.511)--(5.090,3.565)%
  --(5.196,3.619)--(5.301,3.673)--(5.407,3.726)--(5.512,3.780)--(5.618,3.834)--(5.723,3.888)%
  --(5.829,3.942)--(5.934,3.995)--(6.040,4.049)--(6.145,4.103)--(6.251,4.157)--(6.356,4.211)%
  --(6.462,4.264)--(6.567,4.318)--(6.673,4.372)--(6.778,4.426)--(6.884,4.480)--(6.989,4.533)%
  --(7.095,4.587)--(7.200,4.641)--(7.306,4.695)--(7.411,4.749)--(7.517,4.803)--(7.622,4.856)%
  --(7.728,4.910)--(7.833,4.964)--(7.939,5.018)--(8.044,5.072)--(8.150,5.125)--(8.255,5.179)%
  --(8.361,5.233)--(8.466,5.287)--(8.571,5.341)--(8.677,5.394)--(8.782,5.448)--(8.888,5.502)%
  --(8.993,5.556)--(9.099,5.610)--(9.204,5.663)--(9.310,5.717)--(9.415,5.771)--(9.521,5.825)%
  --(9.626,5.879)--(9.732,5.933)--(9.837,5.986)--(9.943,6.040)--(10.048,6.094)--(10.154,6.148)%
  --(10.259,6.202)--(10.365,6.255)--(10.470,6.309)--(10.576,6.363)--(10.681,6.417)--(10.787,6.471)%
  --(10.892,6.524)--(10.998,6.578)--(11.103,6.632)--(11.209,6.686)--(11.314,6.740)--(11.420,6.793)%
  --(11.525,6.847)--(11.631,6.901)--(11.736,6.955)--(11.842,7.009)--(11.947,7.062);
\gpcolor{color=gp lt color border}
\gpsetlinetype{gp lt border}
\draw[gp path] (1.504,7.825)--(1.504,0.985)--(11.947,0.985)--(11.947,7.825)--cycle;
%% coordinates of the plot area
\gpdefrectangularnode{gp plot 1}{\pgfpoint{1.504cm}{0.985cm}}{\pgfpoint{11.947cm}{7.825cm}}
\end{tikzpicture}
%% gnuplot variables
 \caption{Incertezza sulle rette} \label{gr:01_graph_2.tex} \end{grafico}

Si trova per il fuoco il valore 
\[ f=\left(6.72 \pm 0.07\right) \cm , \] 
si \`e considerata trascurabile l'incertezza su $\mu*$ (la scelta di considerare l'incertezza strumentale non influisce quindi sugli altri risultati) rispetto a quello fornito dal laboratorio su $P_L$ e $P_0$, pari a $\sigma _P = 0.05 \cm$: ci\`o porta propagando a un'incertezza di $\sqrt{2} \cdot \sigma_P = 0.07 \cm$. Tale stima va considerata a meno della correzione di aberrazione sferica, che verr\`a compiuta sulla miglior stima del valore di $f$).

% Io mi dilungherei personalmente un pochino di più sulla questione dell'errore, ho capito che è calato dall'alto, però ne parlerei con un po' più precisione: intendo, io direi qualcosa del tipo "L'incertezza è presa di 0.7 perché a causa dei ritardi delle componenti meccaniche e degli errori di lettura è stato precedentemente verificato che il valore della posizione di ogni cavaliere ha un'incertezza di 0.05 cm -tra l'altro Gabriele, c'è uno zero che mi manca qui-, per cui l'incertezza -Non usare errore, nella prima lezione che abbiamo fatto con lui ha fatto un discorso su come gli erorri siano una cosa, mentre quello che stima un fisico è un'incertezza- è data dalla somma quadratica di tutte le incertezze presenti nella formula, ma da un'occhiata rapida risulta evidente come tutte le incertezze sono effettivamente trascurabili rispetto a quelle su $P_L$ e $P_o$, per cui l'incertezza si riduce a $\radq{2} \cdot \sigma_{posizionamento} = 0.07 \cm$" ---Davide
% CORR1: Per quanto riguarda l'err... l'incertezza sono stato piu' preciso. Per la spiegazione di mustar si veda all'inizio del doc, per muzero effettivamente va inserito, lo metterei nella metodologia di misura. Inserisco anche l'errore su muzero e mustar.

\FloatBarrier
\subsection{Punti coniugati}
Un ulteriore metodo per la misurazione del fuoco della lente è quello che prevede l'utilizzo della legge dei punti coniugati.
 A partire dall'utilizzo dell'equazione di Gauss per le lenti sottili, apportando alcune correzioni legate al modello delle lenti
 spesse, si può trovare una serie di equazioni che descrivono il comportamento delle lenti attraverso semplici misure di
 lunghezze. Tali equazioni sono:
\[p=P_L - P_o + \frac {dr} {2} - \frac {\overline{PP'}}{2}\]
\[q=P_S - P_L -\frac {dr} {2} + \frac {\overline{PP'}} {2}\]
\begin{equation} \label{eq:punticoniugati}
\frac{1}{f} = \frac {1}{p} + \frac {1}{q}
\end{equation}

% con  p\textsubscript{o} distanza oggetto, p\textsubscript{L} distanza lente e p\textsubscript{S} distanza schermo.
 Sono stati prelevati due campioni diversi, dato che il primo campione possedeva delle imprecisioni. I campioni si possono trovare
 nelle due tabelle di cui sotto:
%**02_tab_1.fdat**	**02_tab_2.txt**
\begin{tabella}
	\centering
	\begin{tabulary}{\textwidth}{CCCC}
\toprule
%primo campione
\multicolumn{2}{c}{Campione I} & \multicolumn{2}{c}{Campione II} \\
\cmidrule(r){1-2} \cmidrule(l){3-4}
$p_L$ & $p_S$ & $p_L$ & $p_S$ \\ \midrule
23.70 & 30.00 & 21.00 & 62.25 \\ \midrule
24.40 & 32.00 & 22.00 & 47.50 \\ \midrule
24.30 & 34.50 & 23.00 & 43.30 \\ \midrule
24.70 & 37.00 & 24.00 & 41.10 \\ \midrule
27.90 & 40.00 & 25.00 & 40.15 \\ \midrule
33.00 & 43.50 & 26.00 & 39.90 \\ \midrule
37.95 & 47.50 & 27.00 & 39.70 \\ \midrule
43.35 & 52.50 & 28.00 & 40.10 \\ \midrule
57.00 & 65.50 & 29.00 & 40.75 \\ \midrule
68.90 & 77.00 & 30.00 & 41.10 \\ \midrule
83.00 & 91.00 & 35.00 & 44.70 \\ \midrule
103.15 & 111.00 & 40.00 & 49.10 \\ \midrule
127.35 & 135.00 & 50.00 & 58.30 \\ \midrule
- & - & 62.00 & 69.90 \\ \midrule
- & - & 75.00 & 82.70 \\ \midrule
- & - & 90.00 & 97.50 \\
\bottomrule%
% \\ \midrule
%secondo campione
\end{tabulary}


	\caption{Campioni (udm in $[\cm\,]$)}
	\label{tab:02tab1}
\end{tabella}

%\begin{tabella}
	%\centering
	%\input{./tabelle/02_tab_2.tex}
	%\caption{Campione II $[\cm\,]$}
	%\label{tab:02tab2}
%\end{tabella}
I valori mantenuti costanti durante tutta l'esperienza e i paramentri fisici della lente sono:
\[P_o = 13.15 \cm , \nicefrac{dr}{2} = 0.115 \cm ,\frac {\overline{PP'}}{2} = \frac{\overline{VV'}}{6} = 0.133 \mm\]
Tali campioni sono stati presi in momenti diversi, in particolare sono stati riposizionati tutti i cavalieri tra un campione e l'
 altro. Si faccia un grafico per cercare di comprendere come i dati si ditribuiscono in un piano cartesiano: nel grafico di cui
 sotto, è rappresentato $\usuq$ in funzione di $\usup$, dalle formule sopra evidenziate risulta che il grafico dovrebbe essere una retta con
 pendenza $-1$. Ecco come compiono i dati raccolti:
\begin{grafico} \centering \begin{tikzpicture}[gnuplot]
%% generated with GNUPLOT 4.6p4 (Lua 5.1; terminal rev. 99, script rev. 100)
%% lun 22 dic 2014 14:04:42 CET
\path (0.000,0.000) rectangle (12.500,8.750);
\gpcolor{color=gp lt color border}
\gpsetlinetype{gp lt border}
\gpsetlinewidth{1.00}
\draw[gp path] (1.688,0.985)--(1.868,0.985);
\draw[gp path] (11.947,0.985)--(11.767,0.985);
\node[gp node right] at (1.504,0.985) { 0.02};
\draw[gp path] (1.688,1.910)--(1.868,1.910);
\draw[gp path] (11.947,1.910)--(11.767,1.910);
\node[gp node right] at (1.504,1.910) { 0.04};
\draw[gp path] (1.688,2.834)--(1.868,2.834);
\draw[gp path] (11.947,2.834)--(11.767,2.834);
\node[gp node right] at (1.504,2.834) { 0.06};
\draw[gp path] (1.688,3.759)--(1.868,3.759);
\draw[gp path] (11.947,3.759)--(11.767,3.759);
\node[gp node right] at (1.504,3.759) { 0.08};
\draw[gp path] (1.688,4.683)--(1.868,4.683);
\draw[gp path] (11.947,4.683)--(11.767,4.683);
\node[gp node right] at (1.504,4.683) { 0.1};
\draw[gp path] (1.688,5.608)--(1.868,5.608);
\draw[gp path] (11.947,5.608)--(11.767,5.608);
\node[gp node right] at (1.504,5.608) { 0.12};
\draw[gp path] (1.688,6.532)--(1.868,6.532);
\draw[gp path] (11.947,6.532)--(11.767,6.532);
\node[gp node right] at (1.504,6.532) { 0.14};
\draw[gp path] (1.688,7.457)--(1.868,7.457);
\draw[gp path] (11.947,7.457)--(11.767,7.457);
\node[gp node right] at (1.504,7.457) { 0.16};
\draw[gp path] (1.688,8.381)--(1.868,8.381);
\draw[gp path] (11.947,8.381)--(11.767,8.381);
\node[gp node right] at (1.504,8.381) { 0.18};
\draw[gp path] (1.688,0.985)--(1.688,1.165);
\draw[gp path] (1.688,8.381)--(1.688,8.201);
\node[gp node center] at (1.688,0.677) { 0};
\draw[gp path] (3.154,0.985)--(3.154,1.165);
\draw[gp path] (3.154,8.381)--(3.154,8.201);
\node[gp node center] at (3.154,0.677) { 0.02};
\draw[gp path] (4.619,0.985)--(4.619,1.165);
\draw[gp path] (4.619,8.381)--(4.619,8.201);
\node[gp node center] at (4.619,0.677) { 0.04};
\draw[gp path] (6.085,0.985)--(6.085,1.165);
\draw[gp path] (6.085,8.381)--(6.085,8.201);
\node[gp node center] at (6.085,0.677) { 0.06};
\draw[gp path] (7.550,0.985)--(7.550,1.165);
\draw[gp path] (7.550,8.381)--(7.550,8.201);
\node[gp node center] at (7.550,0.677) { 0.08};
\draw[gp path] (9.016,0.985)--(9.016,1.165);
\draw[gp path] (9.016,8.381)--(9.016,8.201);
\node[gp node center] at (9.016,0.677) { 0.1};
\draw[gp path] (10.481,0.985)--(10.481,1.165);
\draw[gp path] (10.481,8.381)--(10.481,8.201);
\node[gp node center] at (10.481,0.677) { 0.12};
\draw[gp path] (11.947,0.985)--(11.947,1.165);
\draw[gp path] (11.947,8.381)--(11.947,8.201);
\node[gp node center] at (11.947,0.677) { 0.14};
\draw[gp path] (1.688,8.381)--(1.688,0.985)--(11.947,0.985)--(11.947,8.381)--cycle;
\node[gp node center,rotate=-270] at (0.246,4.683) {$\nicefrac{1}{q} [\cm^{-1}]$};
\node[gp node center] at (6.817,0.215) {$\nicefrac{1}{p} [\cm^{-1}]$};
\node[gp node right] at (10.479,8.047) {Campione I};
\gpcolor{color=gp lt color 0}
\gpsetpointsize{4.00}
\gppoint{gp mark 1}{(8.646,7.699)}
\gppoint{gp mark 1}{(8.212,6.348)}
\gppoint{gp mark 1}{(8.271,4.705)}
\gppoint{gp mark 1}{(8.043,3.896)}
\gppoint{gp mark 1}{(6.662,3.961)}
\gppoint{gp mark 1}{(5.383,4.570)}
\gppoint{gp mark 1}{(4.645,5.030)}
\gppoint{gp mark 1}{(4.116,5.253)}
\gppoint{gp mark 1}{(3.360,5.662)}
\gppoint{gp mark 1}{(3.003,5.948)}
\gppoint{gp mark 1}{(2.737,6.024)}
\gppoint{gp mark 1}{(2.502,6.141)}
\gppoint{gp mark 1}{(2.330,6.306)}
\gppoint{gp mark 1}{(11.121,8.047)}
\gpcolor{color=gp lt color border}
\node[gp node right] at (10.479,7.739) {Campione II};
\gpcolor{rgb color={0.000,0.502,0.000}}
\gppoint{gp mark 2}{(6.042,4.320)}
\gppoint{gp mark 2}{(6.317,4.079)}
\gppoint{gp mark 2}{(6.629,3.961)}
\gppoint{gp mark 2}{(6.986,3.773)}
\gppoint{gp mark 2}{(7.399,3.447)}
\gppoint{gp mark 2}{(7.881,3.163)}
\gppoint{gp mark 2}{(8.453,2.804)}
\gppoint{gp mark 2}{(9.141,2.366)}
\gppoint{gp mark 2}{(9.985,1.891)}
\gppoint{gp mark 2}{(11.045,1.188)}
\gppoint{gp mark 2}{(5.045,4.951)}
\gppoint{gp mark 2}{(4.419,5.283)}
\gppoint{gp mark 2}{(3.678,5.802)}
\gppoint{gp mark 2}{(3.189,6.102)}
\gppoint{gp mark 2}{(2.873,6.264)}
\gppoint{gp mark 2}{(2.642,6.435)}
\gppoint{gp mark 2}{(11.121,7.739)}
\gpcolor{color=gp lt color border}
\draw[gp path] (1.688,8.381)--(1.688,0.985)--(11.947,0.985)--(11.947,8.381)--cycle;
%% coordinates of the plot area
\gpdefrectangularnode{gp plot 1}{\pgfpoint{1.688cm}{0.985cm}}{\pgfpoint{11.947cm}{8.381cm}}
\end{tikzpicture}
%% gnuplot variables
 \caption{Punti coniugati} \label{gr:02graph1.tex} \end{grafico}

Dal grafico si evince facilmente come nel "Campione I" ci siano dei punti visibilmente fuori scala, probabilmente a causa di
 inesattezze da parte degli sperimentatori nel prendere quei dati (quando la lente era particolarmente vicina all'oggetto risultava
 meno semplice andare ad identificare in maniera precisa a quale distanza sarebbe dovuto essere lo schermo affinché l'immagine
 fosse a fuoco). Alcune accortezze, prese proprio in risposta al primo campione poco convincente, hanno permesso di prendere
 un altro set di misure che fosse più preciso del precedente. Ma i due campioni appartengono alla stessa popolazione? La risposta
 risulta evidentemente no, infatti anche nella zona in cui il "Campione I" non è evidentemente sbagliato le due rette non coincidono,
 come si può vedere dal seguente grafico in cui sono stati riportati i dati non evidentemente sbagliati assieme alla rette che li
 interpolano $y = ax + b$:
%**02_graph_2.txt**	**02_tab_3.txt**
\begin{grafico} \centering \begin{tikzpicture}[gnuplot]
%% generated with GNUPLOT 4.6p4 (Lua 5.1; terminal rev. 99, script rev. 100)
%% lun 22 dic 2014 14:04:42 CET
\path (0.000,0.000) rectangle (12.500,8.750);
\gpcolor{color=gp lt color border}
\gpsetlinetype{gp lt border}
\gpsetlinewidth{1.00}
\draw[gp path] (1.688,0.985)--(1.868,0.985);
\draw[gp path] (11.947,0.985)--(11.767,0.985);
\node[gp node right] at (1.504,0.985) { 0.02};
\draw[gp path] (1.688,2.042)--(1.868,2.042);
\draw[gp path] (11.947,2.042)--(11.767,2.042);
\node[gp node right] at (1.504,2.042) { 0.04};
\draw[gp path] (1.688,3.098)--(1.868,3.098);
\draw[gp path] (11.947,3.098)--(11.767,3.098);
\node[gp node right] at (1.504,3.098) { 0.06};
\draw[gp path] (1.688,4.155)--(1.868,4.155);
\draw[gp path] (11.947,4.155)--(11.767,4.155);
\node[gp node right] at (1.504,4.155) { 0.08};
\draw[gp path] (1.688,5.211)--(1.868,5.211);
\draw[gp path] (11.947,5.211)--(11.767,5.211);
\node[gp node right] at (1.504,5.211) { 0.1};
\draw[gp path] (1.688,6.268)--(1.868,6.268);
\draw[gp path] (11.947,6.268)--(11.767,6.268);
\node[gp node right] at (1.504,6.268) { 0.12};
\draw[gp path] (1.688,7.324)--(1.868,7.324);
\draw[gp path] (11.947,7.324)--(11.767,7.324);
\node[gp node right] at (1.504,7.324) { 0.14};
\draw[gp path] (1.688,8.381)--(1.868,8.381);
\draw[gp path] (11.947,8.381)--(11.767,8.381);
\node[gp node right] at (1.504,8.381) { 0.16};
\draw[gp path] (1.688,0.985)--(1.688,1.165);
\draw[gp path] (1.688,8.381)--(1.688,8.201);
\node[gp node center] at (1.688,0.677) { 0};
\draw[gp path] (3.154,0.985)--(3.154,1.165);
\draw[gp path] (3.154,8.381)--(3.154,8.201);
\node[gp node center] at (3.154,0.677) { 0.02};
\draw[gp path] (4.619,0.985)--(4.619,1.165);
\draw[gp path] (4.619,8.381)--(4.619,8.201);
\node[gp node center] at (4.619,0.677) { 0.04};
\draw[gp path] (6.085,0.985)--(6.085,1.165);
\draw[gp path] (6.085,8.381)--(6.085,8.201);
\node[gp node center] at (6.085,0.677) { 0.06};
\draw[gp path] (7.550,0.985)--(7.550,1.165);
\draw[gp path] (7.550,8.381)--(7.550,8.201);
\node[gp node center] at (7.550,0.677) { 0.08};
\draw[gp path] (9.016,0.985)--(9.016,1.165);
\draw[gp path] (9.016,8.381)--(9.016,8.201);
\node[gp node center] at (9.016,0.677) { 0.1};
\draw[gp path] (10.481,0.985)--(10.481,1.165);
\draw[gp path] (10.481,8.381)--(10.481,8.201);
\node[gp node center] at (10.481,0.677) { 0.12};
\draw[gp path] (11.947,0.985)--(11.947,1.165);
\draw[gp path] (11.947,8.381)--(11.947,8.201);
\node[gp node center] at (11.947,0.677) { 0.14};
\draw[gp path] (1.688,8.381)--(1.688,0.985)--(11.947,0.985)--(11.947,8.381)--cycle;
\node[gp node center,rotate=-270] at (0.246,4.683) {$\nicefrac{1}{q} [\cm^{-1}]$};
\node[gp node center] at (6.817,0.215) {$\nicefrac{1}{p} [\cm^{-1}]$};
\node[gp node right] at (10.479,8.047) {Campione I};
\gpcolor{rgb color={0.698,0.000,0.000}}
\gpsetpointsize{4.00}
\gppoint{gp mark 1}{(6.042,4.797)}
\gppoint{gp mark 1}{(6.317,4.522)}
\gppoint{gp mark 1}{(6.629,4.386)}
\gppoint{gp mark 1}{(6.986,4.171)}
\gppoint{gp mark 1}{(7.399,3.798)}
\gppoint{gp mark 1}{(7.881,3.474)}
\gppoint{gp mark 1}{(8.453,3.063)}
\gppoint{gp mark 1}{(9.141,2.563)}
\gppoint{gp mark 1}{(9.985,2.021)}
\gppoint{gp mark 1}{(11.045,1.217)}
\gppoint{gp mark 1}{(5.045,5.518)}
\gppoint{gp mark 1}{(4.419,5.897)}
\gppoint{gp mark 1}{(3.678,6.490)}
\gppoint{gp mark 1}{(3.189,6.833)}
\gppoint{gp mark 1}{(2.873,7.018)}
\gppoint{gp mark 1}{(2.642,7.213)}
\gppoint{gp mark 1}{(11.121,8.047)}
\gpcolor{color=gp lt color border}
\node[gp node right] at (10.479,7.739) {Campione II};
\gpcolor{rgb color={0.000,0.000,0.800}}
\gppoint{gp mark 2}{(5.383,5.082)}
\gppoint{gp mark 2}{(4.645,5.608)}
\gppoint{gp mark 2}{(4.116,5.863)}
\gppoint{gp mark 2}{(3.360,6.331)}
\gppoint{gp mark 2}{(3.003,6.657)}
\gppoint{gp mark 2}{(2.737,6.744)}
\gppoint{gp mark 2}{(2.502,6.878)}
\gppoint{gp mark 2}{(2.330,7.066)}
\gppoint{gp mark 2}{(11.121,7.739)}
\gpcolor{color=gp lt color border}
\node[gp node right] at (10.479,7.431) {f(x)};
\gpcolor{rgb color={1.000,0.000,0.000}}
\gpsetlinetype{gp lt plot 2}
\draw[gp path] (10.663,7.431)--(11.579,7.431);
\draw[gp path] (2.330,7.425)--(2.418,7.363)--(2.506,7.300)--(2.594,7.237)--(2.682,7.175)%
  --(2.770,7.112)--(2.858,7.050)--(2.946,6.987)--(3.034,6.925)--(3.122,6.862)--(3.210,6.799)%
  --(3.298,6.737)--(3.386,6.674)--(3.474,6.612)--(3.562,6.549)--(3.650,6.486)--(3.738,6.424)%
  --(3.826,6.361)--(3.914,6.299)--(4.002,6.236)--(4.090,6.173)--(4.178,6.111)--(4.266,6.048)%
  --(4.354,5.986)--(4.442,5.923)--(4.531,5.860)--(4.619,5.798)--(4.707,5.735)--(4.795,5.673)%
  --(4.883,5.610)--(4.971,5.547)--(5.059,5.485)--(5.147,5.422)--(5.235,5.360)--(5.323,5.297)%
  --(5.411,5.234)--(5.499,5.172)--(5.587,5.109)--(5.675,5.047)--(5.763,4.984)--(5.851,4.921)%
  --(5.939,4.859)--(6.027,4.796)--(6.115,4.734)--(6.203,4.671)--(6.291,4.608)--(6.379,4.546)%
  --(6.467,4.483)--(6.555,4.421)--(6.643,4.358)--(6.731,4.296)--(6.819,4.233)--(6.907,4.170)%
  --(6.995,4.108)--(7.083,4.045)--(7.171,3.983)--(7.259,3.920)--(7.347,3.857)--(7.435,3.795)%
  --(7.524,3.732)--(7.612,3.670)--(7.700,3.607)--(7.788,3.544)--(7.876,3.482)--(7.964,3.419)%
  --(8.052,3.357)--(8.140,3.294)--(8.228,3.231)--(8.316,3.169)--(8.404,3.106)--(8.492,3.044)%
  --(8.580,2.981)--(8.668,2.918)--(8.756,2.856)--(8.844,2.793)--(8.932,2.731)--(9.020,2.668)%
  --(9.108,2.605)--(9.196,2.543)--(9.284,2.480)--(9.372,2.418)--(9.460,2.355)--(9.548,2.292)%
  --(9.636,2.230)--(9.724,2.167)--(9.812,2.105)--(9.900,2.042)--(9.988,1.979)--(10.076,1.917)%
  --(10.164,1.854)--(10.252,1.792)--(10.340,1.729)--(10.428,1.666)--(10.517,1.604)--(10.605,1.541)%
  --(10.693,1.479)--(10.781,1.416)--(10.869,1.354)--(10.957,1.291)--(11.045,1.228);
\gpcolor{color=gp lt color border}
\node[gp node right] at (10.479,7.123) {g(x)};
\gpcolor{rgb color={0.102,0.000,1.000}}
\gpsetlinetype{gp lt plot 3}
\draw[gp path] (10.663,7.123)--(11.579,7.123);
\draw[gp path] (2.330,7.025)--(2.418,6.969)--(2.506,6.914)--(2.594,6.858)--(2.682,6.802)%
  --(2.770,6.746)--(2.858,6.691)--(2.946,6.635)--(3.034,6.579)--(3.122,6.524)--(3.210,6.468)%
  --(3.298,6.412)--(3.386,6.357)--(3.474,6.301)--(3.562,6.245)--(3.650,6.189)--(3.738,6.134)%
  --(3.826,6.078)--(3.914,6.022)--(4.002,5.967)--(4.090,5.911)--(4.178,5.855)--(4.266,5.799)%
  --(4.354,5.744)--(4.442,5.688)--(4.531,5.632)--(4.619,5.577)--(4.707,5.521)--(4.795,5.465)%
  --(4.883,5.409)--(4.971,5.354)--(5.059,5.298)--(5.147,5.242)--(5.235,5.187)--(5.323,5.131)%
  --(5.411,5.075)--(5.499,5.019)--(5.587,4.964)--(5.675,4.908)--(5.763,4.852)--(5.851,4.797)%
  --(5.939,4.741)--(6.027,4.685)--(6.115,4.629)--(6.203,4.574)--(6.291,4.518)--(6.379,4.462)%
  --(6.467,4.407)--(6.555,4.351)--(6.643,4.295)--(6.731,4.239)--(6.819,4.184)--(6.907,4.128)%
  --(6.995,4.072)--(7.083,4.017)--(7.171,3.961)--(7.259,3.905)--(7.347,3.849)--(7.435,3.794)%
  --(7.524,3.738)--(7.612,3.682)--(7.700,3.627)--(7.788,3.571)--(7.876,3.515)--(7.964,3.459)%
  --(8.052,3.404)--(8.140,3.348)--(8.228,3.292)--(8.316,3.237)--(8.404,3.181)--(8.492,3.125)%
  --(8.580,3.069)--(8.668,3.014)--(8.756,2.958)--(8.844,2.902)--(8.932,2.847)--(9.020,2.791)%
  --(9.108,2.735)--(9.196,2.679)--(9.284,2.624)--(9.372,2.568)--(9.460,2.512)--(9.548,2.457)%
  --(9.636,2.401)--(9.724,2.345)--(9.812,2.289)--(9.900,2.234)--(9.988,2.178)--(10.076,2.122)%
  --(10.164,2.067)--(10.252,2.011)--(10.340,1.955)--(10.428,1.900)--(10.517,1.844)--(10.605,1.788)%
  --(10.693,1.732)--(10.781,1.677)--(10.869,1.621)--(10.957,1.565)--(11.045,1.510);
\gpcolor{color=gp lt color border}
\gpsetlinetype{gp lt border}
\draw[gp path] (1.688,8.381)--(1.688,0.985)--(11.947,0.985)--(11.947,8.381)--cycle;
%% coordinates of the plot area
\gpdefrectangularnode{gp plot 1}{\pgfpoint{1.688cm}{0.985cm}}{\pgfpoint{11.947cm}{8.381cm}}
\end{tikzpicture}
%% gnuplot variables
 \caption{Le interpolazioni dei due campioni} \label{gr:02graph2.tex} \end{grafico}
\begin{tabella}
	\centering
	\begin{tabulary}{\textwidth}{CCCCC}
\toprule
Campione & $a$  & $\sigma_a$ & $b [\cm^{-1}]$ & $\sigma_b [\cm^{-1}]$\\ \midrule
I & -0.88 & 0.02 & 0.1420 & 0.0006\\ %\midrule
II & -0.986 & 0.004 & 0.1505 & 0.0003\\
\bottomrule
\end{tabulary}


	\caption{Rette interpolanti}
	\label{tab:02tab3}
\end{tabella}

Come è stato già detto, la teoria ci dice che la retta che rappresentiamo sul piano cartesiano ha pendenza $-1$. Il nostro approccio
 sperimentale, però, inserisce errori di tipo statistico e sisetmatico che fanno in modo che la retta che effettivamente si trova
 non ha coefficiente angolare esattamente $-1$ ma ci si avvicina. A questo punto sorge spontaneo chiedersi se convenga di più fare
 un'analisi dati basata su un'interpolazione a doppio parametro (pendenza, intercetta), o fissare la pendenza a $-1$ come vuole la
 teoria andando a cercare quale sia l'intercetta in qusto caso. Le due diverse interpolazioni danno origine a un grafico simile:
%**02_graph_3.txt**	**02_tab_4.txt**
\begin{grafico} \centering \begin{tikzpicture}[gnuplot]
%% generated with GNUPLOT 4.6p4 (Lua 5.1; terminal rev. 99, script rev. 100)
%% dom 21 dic 2014 17:09:25 CET
\path (0.000,0.000) rectangle (12.500,8.750);
\gpcolor{color=gp lt color border}
\gpsetlinetype{gp lt border}
\gpsetlinewidth{1.00}
\draw[gp path] (1.688,0.985)--(1.868,0.985);
\draw[gp path] (11.947,0.985)--(11.767,0.985);
\node[gp node right] at (1.504,0.985) { 0.02};
\draw[gp path] (1.688,2.218)--(1.868,2.218);
\draw[gp path] (11.947,2.218)--(11.767,2.218);
\node[gp node right] at (1.504,2.218) { 0.04};
\draw[gp path] (1.688,3.450)--(1.868,3.450);
\draw[gp path] (11.947,3.450)--(11.767,3.450);
\node[gp node right] at (1.504,3.450) { 0.06};
\draw[gp path] (1.688,4.683)--(1.868,4.683);
\draw[gp path] (11.947,4.683)--(11.767,4.683);
\node[gp node right] at (1.504,4.683) { 0.08};
\draw[gp path] (1.688,5.916)--(1.868,5.916);
\draw[gp path] (11.947,5.916)--(11.767,5.916);
\node[gp node right] at (1.504,5.916) { 0.1};
\draw[gp path] (1.688,7.148)--(1.868,7.148);
\draw[gp path] (11.947,7.148)--(11.767,7.148);
\node[gp node right] at (1.504,7.148) { 0.12};
\draw[gp path] (1.688,8.381)--(1.868,8.381);
\draw[gp path] (11.947,8.381)--(11.767,8.381);
\node[gp node right] at (1.504,8.381) { 0.14};
\draw[gp path] (1.688,0.985)--(1.688,1.165);
\draw[gp path] (1.688,8.381)--(1.688,8.201);
\node[gp node center] at (1.688,0.677) { 0};
\draw[gp path] (3.154,0.985)--(3.154,1.165);
\draw[gp path] (3.154,8.381)--(3.154,8.201);
\node[gp node center] at (3.154,0.677) { 0.02};
\draw[gp path] (4.619,0.985)--(4.619,1.165);
\draw[gp path] (4.619,8.381)--(4.619,8.201);
\node[gp node center] at (4.619,0.677) { 0.04};
\draw[gp path] (6.085,0.985)--(6.085,1.165);
\draw[gp path] (6.085,8.381)--(6.085,8.201);
\node[gp node center] at (6.085,0.677) { 0.06};
\draw[gp path] (7.550,0.985)--(7.550,1.165);
\draw[gp path] (7.550,8.381)--(7.550,8.201);
\node[gp node center] at (7.550,0.677) { 0.08};
\draw[gp path] (9.016,0.985)--(9.016,1.165);
\draw[gp path] (9.016,8.381)--(9.016,8.201);
\node[gp node center] at (9.016,0.677) { 0.1};
\draw[gp path] (10.481,0.985)--(10.481,1.165);
\draw[gp path] (10.481,8.381)--(10.481,8.201);
\node[gp node center] at (10.481,0.677) { 0.12};
\draw[gp path] (11.947,0.985)--(11.947,1.165);
\draw[gp path] (11.947,8.381)--(11.947,8.201);
\node[gp node center] at (11.947,0.677) { 0.14};
\draw[gp path] (1.688,8.381)--(1.688,0.985)--(11.947,0.985)--(11.947,8.381)--cycle;
\node[gp node center,rotate=-270] at (0.246,4.683) {$\nicefrac{1}{q} [\cm^{-1}]$};
\node[gp node center] at (6.817,0.215) {$\nicefrac{1}{p} [\cm^{-1}]$};
\node[gp node right] at (10.479,8.047) {Dati Sperimentali};
\gpcolor{color=gp lt color 0}
\gpsetpointsize{4.00}
\gppoint{gp mark 1}{(6.042,5.432)}
\gppoint{gp mark 1}{(6.317,5.111)}
\gppoint{gp mark 1}{(6.629,4.953)}
\gppoint{gp mark 1}{(6.986,4.702)}
\gppoint{gp mark 1}{(7.399,4.267)}
\gppoint{gp mark 1}{(7.881,3.888)}
\gppoint{gp mark 1}{(8.453,3.410)}
\gppoint{gp mark 1}{(9.141,2.826)}
\gppoint{gp mark 1}{(9.985,2.193)}
\gppoint{gp mark 1}{(11.045,1.256)}
\gppoint{gp mark 1}{(5.045,6.273)}
\gppoint{gp mark 1}{(4.419,6.715)}
\gppoint{gp mark 1}{(3.678,7.407)}
\gppoint{gp mark 1}{(3.189,7.807)}
\gppoint{gp mark 1}{(2.873,8.023)}
\gppoint{gp mark 1}{(2.642,8.252)}
\gppoint{gp mark 1}{(11.121,8.047)}
\gpcolor{color=gp lt color border}
\node[gp node right] at (10.479,7.739) {Interpolazione con due parametri};
\gpcolor{color=gp lt color 1}
\gpsetlinetype{gp lt plot 1}
\draw[gp path] (10.663,7.739)--(11.579,7.739);
\draw[gp path] (2.642,8.240)--(2.727,8.169)--(2.812,8.099)--(2.896,8.029)--(2.981,7.958)%
  --(3.066,7.888)--(3.151,7.817)--(3.236,7.747)--(3.321,7.677)--(3.406,7.606)--(3.491,7.536)%
  --(3.575,7.465)--(3.660,7.395)--(3.745,7.324)--(3.830,7.254)--(3.915,7.184)--(4.000,7.113)%
  --(4.085,7.043)--(4.170,6.972)--(4.254,6.902)--(4.339,6.832)--(4.424,6.761)--(4.509,6.691)%
  --(4.594,6.620)--(4.679,6.550)--(4.764,6.479)--(4.849,6.409)--(4.933,6.339)--(5.018,6.268)%
  --(5.103,6.198)--(5.188,6.127)--(5.273,6.057)--(5.358,5.987)--(5.443,5.916)--(5.528,5.846)%
  --(5.612,5.775)--(5.697,5.705)--(5.782,5.635)--(5.867,5.564)--(5.952,5.494)--(6.037,5.423)%
  --(6.122,5.353)--(6.207,5.282)--(6.292,5.212)--(6.376,5.142)--(6.461,5.071)--(6.546,5.001)%
  --(6.631,4.930)--(6.716,4.860)--(6.801,4.790)--(6.886,4.719)--(6.971,4.649)--(7.055,4.578)%
  --(7.140,4.508)--(7.225,4.438)--(7.310,4.367)--(7.395,4.297)--(7.480,4.226)--(7.565,4.156)%
  --(7.650,4.085)--(7.734,4.015)--(7.819,3.945)--(7.904,3.874)--(7.989,3.804)--(8.074,3.733)%
  --(8.159,3.663)--(8.244,3.593)--(8.329,3.522)--(8.413,3.452)--(8.498,3.381)--(8.583,3.311)%
  --(8.668,3.240)--(8.753,3.170)--(8.838,3.100)--(8.923,3.029)--(9.008,2.959)--(9.093,2.888)%
  --(9.177,2.818)--(9.262,2.748)--(9.347,2.677)--(9.432,2.607)--(9.517,2.536)--(9.602,2.466)%
  --(9.687,2.396)--(9.772,2.325)--(9.856,2.255)--(9.941,2.184)--(10.026,2.114)--(10.111,2.043)%
  --(10.196,1.973)--(10.281,1.903)--(10.366,1.832)--(10.451,1.762)--(10.535,1.691)--(10.620,1.621)%
  --(10.705,1.551)--(10.790,1.480)--(10.875,1.410)--(10.960,1.339)--(11.045,1.269);
\gpcolor{color=gp lt color border}
\node[gp node right] at (10.479,7.431) {Interpolazione con un parametro};
\gpcolor{color=gp lt color 2}
\gpsetlinetype{gp lt plot 2}
\draw[gp path] (10.663,7.431)--(11.579,7.431);
\draw[gp path] (2.642,8.283)--(2.727,8.211)--(2.812,8.140)--(2.896,8.068)--(2.981,7.997)%
  --(3.066,7.926)--(3.151,7.854)--(3.236,7.783)--(3.321,7.711)--(3.406,7.640)--(3.491,7.569)%
  --(3.575,7.497)--(3.660,7.426)--(3.745,7.355)--(3.830,7.283)--(3.915,7.212)--(4.000,7.140)%
  --(4.085,7.069)--(4.170,6.998)--(4.254,6.926)--(4.339,6.855)--(4.424,6.783)--(4.509,6.712)%
  --(4.594,6.641)--(4.679,6.569)--(4.764,6.498)--(4.849,6.426)--(4.933,6.355)--(5.018,6.284)%
  --(5.103,6.212)--(5.188,6.141)--(5.273,6.069)--(5.358,5.998)--(5.443,5.927)--(5.528,5.855)%
  --(5.612,5.784)--(5.697,5.713)--(5.782,5.641)--(5.867,5.570)--(5.952,5.498)--(6.037,5.427)%
  --(6.122,5.356)--(6.207,5.284)--(6.292,5.213)--(6.376,5.141)--(6.461,5.070)--(6.546,4.999)%
  --(6.631,4.927)--(6.716,4.856)--(6.801,4.784)--(6.886,4.713)--(6.971,4.642)--(7.055,4.570)%
  --(7.140,4.499)--(7.225,4.428)--(7.310,4.356)--(7.395,4.285)--(7.480,4.213)--(7.565,4.142)%
  --(7.650,4.071)--(7.734,3.999)--(7.819,3.928)--(7.904,3.856)--(7.989,3.785)--(8.074,3.714)%
  --(8.159,3.642)--(8.244,3.571)--(8.329,3.499)--(8.413,3.428)--(8.498,3.357)--(8.583,3.285)%
  --(8.668,3.214)--(8.753,3.143)--(8.838,3.071)--(8.923,3.000)--(9.008,2.928)--(9.093,2.857)%
  --(9.177,2.786)--(9.262,2.714)--(9.347,2.643)--(9.432,2.571)--(9.517,2.500)--(9.602,2.429)%
  --(9.687,2.357)--(9.772,2.286)--(9.856,2.214)--(9.941,2.143)--(10.026,2.072)--(10.111,2.000)%
  --(10.196,1.929)--(10.281,1.857)--(10.366,1.786)--(10.451,1.715)--(10.535,1.643)--(10.620,1.572)%
  --(10.705,1.501)--(10.790,1.429)--(10.875,1.358)--(10.960,1.286)--(11.045,1.215);
\gpcolor{color=gp lt color border}
\gpsetlinetype{gp lt border}
\draw[gp path] (1.688,8.381)--(1.688,0.985)--(11.947,0.985)--(11.947,8.381)--cycle;
%% coordinates of the plot area
\gpdefrectangularnode{gp plot 1}{\pgfpoint{1.688cm}{0.985cm}}{\pgfpoint{11.947cm}{8.381cm}}
\end{tikzpicture}
%% gnuplot variables
 \caption{Le due diverse interpolazioni} \label{gr:02_graph_3.tex} \end{grafico}
\begin{tabella}
	\centering
	\begin{tabulary}{\textwidth}{CCCCC}
\toprule
Numero parametri & $a$  & $\sigma_a$ & $b [\cm^{-1}]$ & $\sigma_b [\cm^{-1}]$\\ \midrule
1 & -1 & - & 0.1514 & 0.0002\\
2 & -0.986 & 0.004 & 0.1505 & 0.0003\\
\bottomrule
\end{tabulary}


	\caption{Numero parametri d'interpolazione}
	\label{tab:02tab4}
\end{tabella}

Per comprendere quale dei due modi di interpretare il fenomeno sia più opportuno usare si utilizzi il metodo dell'$\mathcal{F}$-test:
 sono stati presi gli scarti quadratici medi delle due interpolazione diverse e sono stati normalizzati ai gradi di libertà
 (sono state raccolte 16 coppie di dati, ciò vuol dire che la retta con un solo parametro libero ha 15 gradi di libertà,
 siano $n_1$ nelle formule, e la retta con due parametri liberi ha 14 gradi di libertà, $n_2$).
 La formula utilizzata è:
\[\mathcal{F}=\frac{\dfrac{{\sum_i \big( y_i-f_1 (c_i) \big) ^2} - {\sum_i \big( y_i-f_2 (c_i) \big) ^2}}{n_1-n_2}}{\dfrac{\sum_i \big( y_i-f_2 
(c_i) \big) ^2}{n_2}}\]

Andando a calcolare la $\mathcal{F}$ da questa formula risulta $\mathcal{F} = 0.1740$. Ora, per capire quale delle due interpolazioni meglio si addice alla
 casistica trovata, va cercato nelle tabelle di riferimento relative alla funzione di Von Mises-Fisher, andando a leggere il valore
 in corrispondenza dei gradi di liberà del numeratore (in questo caso 1) per trovare la colonna e i gradi di libertà del denominatore
 (in questo caso 14) per trovare la riga, si legge che vale la pena introdurre un nuovo parametro alla teoria 
 se \footnote{Si prenda una significanza del $90\%$, riferimento tabella: \url{http://www.socr.ucla.edu/applets.dir/f_table.html}} $\mathcal{F} > 3.10$, 
 da cui evinciamo come non sia necessaria la stima del secondo parametro d'interpolazione della retta, che può essere fissato senza perdere
 troppe informazioni a $-1$. Per limitare le perdite di dati si effettuino le analisi sia considerando la pendenza della retta fissa,
 sia considerandola come una variabile casuale dipendente dal campione.

Considerando la pendenza dipendente dal campione, abbiamo una retta che interseca gli assi in due punti di coordinate diverse,
 coordinate che indicano proprio due stime diverse del reciproco del fuoco della lente, che chiameremo rispettivamente
$f_x$ se l'intersezione è con l'asse delle $x$ e $f_y$ se l'intersezione è con l'asse delle $y$.
 Per quanto riguarda $f_y$, la sua stima è piuttosto semplice: prendendo l'equazione della retta interpolante $y=ax + b$
 con $a$ e $b$ letti dalla tabella di cui sopra, basta cercare il punto con la $x$ nulla per trovare l'intersezione con gli assi.
 Con una banale sostituzione otteniamo $f_y= b$ il che ci dà immediatamente un valore reciproco del fuoco che sia anche
 accompagnato dalla sua incertezza: con una semplice propagazione si trova che l'incertezza su $f_y$ è uguale a quella su $b$,
 stimata a partire dalle formule di interpolazione che massimizzano la verosimiglianza (anche questo riportato nella tabella).
 Il primo valore del reciproco del fuoco quindi risulta:
\[f_y = (0.1505 \pm 0.0003) \cm^{-1}\]

Per quanto riguarda l'intersezione con l'asse delle $x$, stavolta il calcolo è leggermente più complesso, infatti sostituendo $y = 0$
 alla formula della retta interpolata si ottiene $f_x = -\frac{q}{m}$.  Per trovare l'incertezza a questo punto risulta necessario
 propagare gli errori su $m$ e su $q$. Applicando la solita formula di propagazione risulta
\[\sigma_{f_x} = \sqrt {\frac{1} {m^2} \sigma_q^2 + \frac{q}{m^2} \sigma_m^2 + 2 \cov (q,m) }.\] Sostituendo,
 il reciproco del fuoco ha un valore di:
\[f_x = (0.153 \pm 0.002) \cm^{-1}\]

Per trovare il fuoco è necessario calcolare i reciproci di $f_x$ ed $f_y$, andando a propagare gli errori secondo la formula di 
propagazione: $f'_x = \frac{1}{f_x}$ e $\sigma_{f'x} = \frac{\sigma_{fx}}{f_{x}^2}$ ed analogo per $f_y$. Da cui i risultati:
\[f'_x = (6.55 \pm 0.09) \cm \]
\[f'_y = (6.64 \pm 0.01) \cm \]

Per dare un risultato finale, si possono includere tutti gli errori nel fatto che questi due numeri sono diversi: si può andare a
 stimare effettivamente il fuoco $f^{\star}$ facendo una media aritmetica dei valori ottenuti e si può stimare l'incertezza
 andando a vedere la
 semidifferenza tra i due valori casuali ottenuti. Il risultato finale è:
\[f = (6.60 \pm 0.04) \cm .\]

Si veda ora come si sarebbero potuti analizzare i dati considerando la retta con l'unico parametro libero l'intercetta. In questo
 caso il risultato del fuoco, come del resto un'indicazione del suo errore statistico, ci vengono dati dai parametri interpolati e
 riportati nella tabella 4, per cui la lunghezza focale, calcolata come nel caso precedente trovando il reciproco
 dell'intercetta e propagando gli errori, sembrerebbe valere:
\[f' = (6.604 \pm 0.009) \cm \]
Ricordando però che i campioni raccolti per la stima della lunghezza focale attraverso il metodo dei punti coniugati erano due ed
 erano visibilmente poco compatibili tra loro, sorge spontaneo chiedersi quali siano gli errori sistematici collegati a tale metodo.
 Dalle formule della legge dei punti coniugati risulta che p dipende dalla posizione iniziale dell'oggetto, che tra
 l'altro rimane fisso per tutta la durata dell'esperimento (probabilmente è da ricondurre ad una non perfetta lettura della posizione
 dell'oggetto il fatto che i due campioni non siano sottoinsiemi della stessa popolazione), il che vuol dire che un'errata lettura
 nella posizione iniziale dell'oggetto può sistematicamente creare un bias nel campione, spostando la retta interpolante dalla
 posizione che realmente dovrebbe occupare. Per stimare tale errore sistematico sono stati creati dei campioni fittizi di $\usup$ e
 di $\usuq$ considerando l'oggetto non nella posizione registrata al momento dell'esperimento, ma spostata di una sigma (0.5 mm) da
 uno dei due lati. I valori ottenuti sono riassunti nella seguene tabella:
%**02_tab_5.txt**
\begin{tabella}
	\centering
	\begin{tabulary}{\textwidth}{CCCC}
\toprule
$\usup$ con $p_o$ aumentato & $\usup$ campione & $\usup$ con $p_o$ diminuito & $\usuq$\\ \midrule
0.0596 & 0.0594 & 0.0592 & 0.0922\\ \midrule
0.0634 & 0.0632 & 0.0630 & 0.0869\\ \midrule
0.0677 & 0.0674 & 0.0672 & 0.0844\\ \midrule
0.0726 & 0.0723 & 0.0720 & 0.0803\\ \midrule
0.0782 & 0.0779 & 0.0776 & 0.0733\\ \midrule
0.0849 & 0.0845 & 0.0842 & 0.0671\\ \midrule
0.0927 & 0.0923 & 0.0919 & 0.0593\\ \midrule
0.1022 & 0.1017 & 0.1012 & 0.0499\\ \midrule
0.1139 & 0.1132 & 0.1126 & 0.0396\\ \midrule
0.1285 & 0.1277 & 0.1269 & 0.0244\\ \midrule
0.0459 & 0.0458 & 0.0457 & 0.1058\\ \midrule
0.0373 & 0.0373 & 0.0372 & 0.1130\\ \midrule
0.0272 & 0.0272 & 0.0271 & 0.1242\\ \midrule
0.0205 & 0.0205 & 0.0205 & 0.1307\\ \midrule
0.0162 & 0.0162 & 0.0162 & 0.1342\\ \midrule
0.0130 & 0.0130 & 0.0130 & 0.1379\\
\bottomrule
\end{tabulary}

	\caption{Campioni con errori sistematici $[\cm^{-1}]$}
	\label{tab:02tab5}
\end{tabella}

Andando a rappresentare sia il campione reale che i campioni fittizi con le relative rette interpolate risulta:
%**02_graf_4.txt**	**02_tab_6.txt**
\begin{grafico} \centering \begin{tikzpicture}[gnuplot]
%% generated with GNUPLOT 4.6p4 (Lua 5.1; terminal rev. 99, script rev. 100)
%% dom 21 dic 2014 17:09:25 CET
\path (0.000,0.000) rectangle (12.500,8.750);
\gpcolor{color=gp lt color border}
\gpsetlinetype{gp lt border}
\gpsetlinewidth{1.00}
\draw[gp path] (1.688,0.985)--(1.868,0.985);
\draw[gp path] (11.947,0.985)--(11.767,0.985);
\node[gp node right] at (1.504,0.985) { 0.02};
\draw[gp path] (1.688,2.218)--(1.868,2.218);
\draw[gp path] (11.947,2.218)--(11.767,2.218);
\node[gp node right] at (1.504,2.218) { 0.04};
\draw[gp path] (1.688,3.450)--(1.868,3.450);
\draw[gp path] (11.947,3.450)--(11.767,3.450);
\node[gp node right] at (1.504,3.450) { 0.06};
\draw[gp path] (1.688,4.683)--(1.868,4.683);
\draw[gp path] (11.947,4.683)--(11.767,4.683);
\node[gp node right] at (1.504,4.683) { 0.08};
\draw[gp path] (1.688,5.916)--(1.868,5.916);
\draw[gp path] (11.947,5.916)--(11.767,5.916);
\node[gp node right] at (1.504,5.916) { 0.1};
\draw[gp path] (1.688,7.148)--(1.868,7.148);
\draw[gp path] (11.947,7.148)--(11.767,7.148);
\node[gp node right] at (1.504,7.148) { 0.12};
\draw[gp path] (1.688,8.381)--(1.868,8.381);
\draw[gp path] (11.947,8.381)--(11.767,8.381);
\node[gp node right] at (1.504,8.381) { 0.14};
\draw[gp path] (1.688,0.985)--(1.688,1.165);
\draw[gp path] (1.688,8.381)--(1.688,8.201);
\node[gp node center] at (1.688,0.677) { 0};
\draw[gp path] (3.154,0.985)--(3.154,1.165);
\draw[gp path] (3.154,8.381)--(3.154,8.201);
\node[gp node center] at (3.154,0.677) { 0.02};
\draw[gp path] (4.619,0.985)--(4.619,1.165);
\draw[gp path] (4.619,8.381)--(4.619,8.201);
\node[gp node center] at (4.619,0.677) { 0.04};
\draw[gp path] (6.085,0.985)--(6.085,1.165);
\draw[gp path] (6.085,8.381)--(6.085,8.201);
\node[gp node center] at (6.085,0.677) { 0.06};
\draw[gp path] (7.550,0.985)--(7.550,1.165);
\draw[gp path] (7.550,8.381)--(7.550,8.201);
\node[gp node center] at (7.550,0.677) { 0.08};
\draw[gp path] (9.016,0.985)--(9.016,1.165);
\draw[gp path] (9.016,8.381)--(9.016,8.201);
\node[gp node center] at (9.016,0.677) { 0.1};
\draw[gp path] (10.481,0.985)--(10.481,1.165);
\draw[gp path] (10.481,8.381)--(10.481,8.201);
\node[gp node center] at (10.481,0.677) { 0.12};
\draw[gp path] (11.947,0.985)--(11.947,1.165);
\draw[gp path] (11.947,8.381)--(11.947,8.201);
\node[gp node center] at (11.947,0.677) { 0.14};
\draw[gp path] (1.688,8.381)--(1.688,0.985)--(11.947,0.985)--(11.947,8.381)--cycle;
\node[gp node center,rotate=-270] at (0.246,4.683) {$\nicefrac{1}{q} [\cm^{-1}]$};
\node[gp node center] at (6.817,0.215) {$\nicefrac{1}{p} [\cm^{-1}]$};
\node[gp node right] at (10.479,8.047) {Campione originario};
\gpcolor{color=gp lt color 0}
\gpsetpointsize{4.00}
\gppoint{gp mark 1}{(6.042,5.432)}
\gppoint{gp mark 1}{(6.317,5.111)}
\gppoint{gp mark 1}{(6.629,4.953)}
\gppoint{gp mark 1}{(6.986,4.702)}
\gppoint{gp mark 1}{(7.399,4.267)}
\gppoint{gp mark 1}{(7.881,3.888)}
\gppoint{gp mark 1}{(8.453,3.410)}
\gppoint{gp mark 1}{(9.141,2.826)}
\gppoint{gp mark 1}{(9.985,2.193)}
\gppoint{gp mark 1}{(11.045,1.256)}
\gppoint{gp mark 1}{(5.045,6.273)}
\gppoint{gp mark 1}{(4.419,6.715)}
\gppoint{gp mark 1}{(3.678,7.407)}
\gppoint{gp mark 1}{(3.189,7.807)}
\gppoint{gp mark 1}{(2.873,8.023)}
\gppoint{gp mark 1}{(2.642,8.252)}
\gppoint{gp mark 1}{(11.121,8.047)}
\gpcolor{color=gp lt color border}
\node[gp node right] at (10.479,7.739) {Campione per difetto};
\gpcolor{color=gp lt color 1}
\gppoint{gp mark 2}{(6.029,5.432)}
\gppoint{gp mark 2}{(6.302,5.111)}
\gppoint{gp mark 2}{(6.612,4.953)}
\gppoint{gp mark 2}{(6.967,4.702)}
\gppoint{gp mark 2}{(7.377,4.267)}
\gppoint{gp mark 2}{(7.855,3.888)}
\gppoint{gp mark 2}{(8.422,3.410)}
\gppoint{gp mark 2}{(9.104,2.826)}
\gppoint{gp mark 2}{(9.939,2.193)}
\gppoint{gp mark 2}{(10.985,1.256)}
\gppoint{gp mark 2}{(5.037,6.273)}
\gppoint{gp mark 2}{(4.414,6.715)}
\gppoint{gp mark 2}{(3.675,7.407)}
\gppoint{gp mark 2}{(3.187,7.807)}
\gppoint{gp mark 2}{(2.872,8.023)}
\gppoint{gp mark 2}{(2.641,8.252)}
\gppoint{gp mark 2}{(11.121,7.739)}
\gpcolor{color=gp lt color border}
\node[gp node right] at (10.479,7.431) {Campione per eccesso};
\gpcolor{color=gp lt color 2}
\gppoint{gp mark 3}{(6.055,5.432)}
\gppoint{gp mark 3}{(6.334,5.111)}
\gppoint{gp mark 3}{(6.649,4.953)}
\gppoint{gp mark 3}{(7.008,4.702)}
\gppoint{gp mark 3}{(7.418,4.267)}
\gppoint{gp mark 3}{(7.909,3.888)}
\gppoint{gp mark 3}{(8.481,3.410)}
\gppoint{gp mark 3}{(9.177,2.826)}
\gppoint{gp mark 3}{(10.034,2.193)}
\gppoint{gp mark 3}{(11.104,1.256)}
\gppoint{gp mark 3}{(5.051,6.273)}
\gppoint{gp mark 3}{(4.421,6.715)}
\gppoint{gp mark 3}{(3.681,7.407)}
\gppoint{gp mark 3}{(3.190,7.807)}
\gppoint{gp mark 3}{(2.875,8.023)}
\gppoint{gp mark 3}{(2.641,8.252)}
\gppoint{gp mark 3}{(11.121,7.431)}
\gpcolor{color=gp lt color border}
\node[gp node right] at (10.479,7.123) {Fit};
\gpcolor{color=gp lt color 3}
\gpsetlinetype{gp lt plot 3}
\draw[gp path] (10.663,7.123)--(11.579,7.123);
\draw[gp path] (2.641,8.284)--(2.726,8.212)--(2.812,8.140)--(2.897,8.068)--(2.983,7.996)%
  --(3.068,7.924)--(3.154,7.852)--(3.239,7.780)--(3.325,7.708)--(3.410,7.636)--(3.496,7.564)%
  --(3.581,7.493)--(3.667,7.421)--(3.752,7.349)--(3.838,7.277)--(3.923,7.205)--(4.008,7.133)%
  --(4.094,7.061)--(4.179,6.989)--(4.265,6.917)--(4.350,6.845)--(4.436,6.773)--(4.521,6.702)%
  --(4.607,6.630)--(4.692,6.558)--(4.778,6.486)--(4.863,6.414)--(4.949,6.342)--(5.034,6.270)%
  --(5.120,6.198)--(5.205,6.126)--(5.291,6.054)--(5.376,5.983)--(5.462,5.911)--(5.547,5.839)%
  --(5.633,5.767)--(5.718,5.695)--(5.804,5.623)--(5.889,5.551)--(5.975,5.479)--(6.060,5.407)%
  --(6.146,5.335)--(6.231,5.263)--(6.317,5.192)--(6.402,5.120)--(6.488,5.048)--(6.573,4.976)%
  --(6.659,4.904)--(6.744,4.832)--(6.830,4.760)--(6.915,4.688)--(7.001,4.616)--(7.086,4.544)%
  --(7.172,4.473)--(7.257,4.401)--(7.343,4.329)--(7.428,4.257)--(7.514,4.185)--(7.599,4.113)%
  --(7.685,4.041)--(7.770,3.969)--(7.856,3.897)--(7.941,3.825)--(8.027,3.753)--(8.112,3.682)%
  --(8.198,3.610)--(8.283,3.538)--(8.369,3.466)--(8.454,3.394)--(8.540,3.322)--(8.625,3.250)%
  --(8.711,3.178)--(8.796,3.106)--(8.882,3.034)--(8.967,2.963)--(9.052,2.891)--(9.138,2.819)%
  --(9.223,2.747)--(9.309,2.675)--(9.394,2.603)--(9.480,2.531)--(9.565,2.459)--(9.651,2.387)%
  --(9.736,2.315)--(9.822,2.243)--(9.907,2.172)--(9.993,2.100)--(10.078,2.028)--(10.164,1.956)%
  --(10.249,1.884)--(10.335,1.812)--(10.420,1.740)--(10.506,1.668)--(10.591,1.596)--(10.677,1.524)%
  --(10.762,1.452)--(10.848,1.381)--(10.933,1.309)--(11.019,1.237)--(11.104,1.165);
\gpcolor{color=gp lt color border}
\node[gp node right] at (10.479,6.815) {Fit per difetto};
\gpcolor{color=gp lt color 4}
\gpsetlinetype{gp lt plot 4}
\draw[gp path] (10.663,6.815)--(11.579,6.815);
\draw[gp path] (2.641,8.267)--(2.726,8.196)--(2.812,8.124)--(2.897,8.052)--(2.983,7.980)%
  --(3.068,7.908)--(3.154,7.836)--(3.239,7.764)--(3.325,7.692)--(3.410,7.620)--(3.496,7.548)%
  --(3.581,7.477)--(3.667,7.405)--(3.752,7.333)--(3.838,7.261)--(3.923,7.189)--(4.008,7.117)%
  --(4.094,7.045)--(4.179,6.973)--(4.265,6.901)--(4.350,6.829)--(4.436,6.757)--(4.521,6.686)%
  --(4.607,6.614)--(4.692,6.542)--(4.778,6.470)--(4.863,6.398)--(4.949,6.326)--(5.034,6.254)%
  --(5.120,6.182)--(5.205,6.110)--(5.291,6.038)--(5.376,5.967)--(5.462,5.895)--(5.547,5.823)%
  --(5.633,5.751)--(5.718,5.679)--(5.804,5.607)--(5.889,5.535)--(5.975,5.463)--(6.060,5.391)%
  --(6.146,5.319)--(6.231,5.247)--(6.317,5.176)--(6.402,5.104)--(6.488,5.032)--(6.573,4.960)%
  --(6.659,4.888)--(6.744,4.816)--(6.830,4.744)--(6.915,4.672)--(7.001,4.600)--(7.086,4.528)%
  --(7.172,4.456)--(7.257,4.385)--(7.343,4.313)--(7.428,4.241)--(7.514,4.169)--(7.599,4.097)%
  --(7.685,4.025)--(7.770,3.953)--(7.856,3.881)--(7.941,3.809)--(8.027,3.737)--(8.112,3.666)%
  --(8.198,3.594)--(8.283,3.522)--(8.369,3.450)--(8.454,3.378)--(8.540,3.306)--(8.625,3.234)%
  --(8.711,3.162)--(8.796,3.090)--(8.882,3.018)--(8.967,2.946)--(9.052,2.875)--(9.138,2.803)%
  --(9.223,2.731)--(9.309,2.659)--(9.394,2.587)--(9.480,2.515)--(9.565,2.443)--(9.651,2.371)%
  --(9.736,2.299)--(9.822,2.227)--(9.907,2.156)--(9.993,2.084)--(10.078,2.012)--(10.164,1.940)%
  --(10.249,1.868)--(10.335,1.796)--(10.420,1.724)--(10.506,1.652)--(10.591,1.580)--(10.677,1.508)%
  --(10.762,1.436)--(10.848,1.365)--(10.933,1.293)--(11.019,1.221)--(11.104,1.149);
\gpcolor{color=gp lt color border}
\node[gp node right] at (10.479,6.507) {Fit per eccesso};
\gpcolor{color=gp lt color 5}
\gpsetlinetype{gp lt plot 5}
\draw[gp path] (10.663,6.507)--(11.579,6.507);
\draw[gp path] (2.641,8.300)--(2.726,8.228)--(2.812,8.156)--(2.897,8.084)--(2.983,8.012)%
  --(3.068,7.940)--(3.154,7.868)--(3.239,7.796)--(3.325,7.724)--(3.410,7.653)--(3.496,7.581)%
  --(3.581,7.509)--(3.667,7.437)--(3.752,7.365)--(3.838,7.293)--(3.923,7.221)--(4.008,7.149)%
  --(4.094,7.077)--(4.179,7.005)--(4.265,6.934)--(4.350,6.862)--(4.436,6.790)--(4.521,6.718)%
  --(4.607,6.646)--(4.692,6.574)--(4.778,6.502)--(4.863,6.430)--(4.949,6.358)--(5.034,6.286)%
  --(5.120,6.214)--(5.205,6.143)--(5.291,6.071)--(5.376,5.999)--(5.462,5.927)--(5.547,5.855)%
  --(5.633,5.783)--(5.718,5.711)--(5.804,5.639)--(5.889,5.567)--(5.975,5.495)--(6.060,5.424)%
  --(6.146,5.352)--(6.231,5.280)--(6.317,5.208)--(6.402,5.136)--(6.488,5.064)--(6.573,4.992)%
  --(6.659,4.920)--(6.744,4.848)--(6.830,4.776)--(6.915,4.704)--(7.001,4.633)--(7.086,4.561)%
  --(7.172,4.489)--(7.257,4.417)--(7.343,4.345)--(7.428,4.273)--(7.514,4.201)--(7.599,4.129)%
  --(7.685,4.057)--(7.770,3.985)--(7.856,3.913)--(7.941,3.842)--(8.027,3.770)--(8.112,3.698)%
  --(8.198,3.626)--(8.283,3.554)--(8.369,3.482)--(8.454,3.410)--(8.540,3.338)--(8.625,3.266)%
  --(8.711,3.194)--(8.796,3.123)--(8.882,3.051)--(8.967,2.979)--(9.052,2.907)--(9.138,2.835)%
  --(9.223,2.763)--(9.309,2.691)--(9.394,2.619)--(9.480,2.547)--(9.565,2.475)--(9.651,2.403)%
  --(9.736,2.332)--(9.822,2.260)--(9.907,2.188)--(9.993,2.116)--(10.078,2.044)--(10.164,1.972)%
  --(10.249,1.900)--(10.335,1.828)--(10.420,1.756)--(10.506,1.684)--(10.591,1.613)--(10.677,1.541)%
  --(10.762,1.469)--(10.848,1.397)--(10.933,1.325)--(11.019,1.253)--(11.104,1.181);
\gpcolor{color=gp lt color border}
\gpsetlinetype{gp lt border}
\draw[gp path] (1.688,8.381)--(1.688,0.985)--(11.947,0.985)--(11.947,8.381)--cycle;
%% coordinates of the plot area
\gpdefrectangularnode{gp plot 1}{\pgfpoint{1.688cm}{0.985cm}}{\pgfpoint{11.947cm}{8.381cm}}
\end{tikzpicture}
%% gnuplot variables
 \caption{Errori su Po} \label{gr:02_graph_4.tex} \end{grafico}
\begin{tabella}
	\centering
	\begin{tabulary}{\textwidth}{LCC}
\toprule
Campione & Intercetta $[\cm^{-1}]$  & $\sigma [\cm^{-1}]$\\ \midrule
Campione originario & 0.1514 & 0.0002\\
Con $p_{o}'=(p_o+0.5)\mm$ & 0.1517 & 0.0008\\
Con $p_{o}'=(p_o-0.5)\mm$ & 0.1512 & 0.0001\\
\bottomrule
\end{tabulary}

	\caption{Rette interpolanti errori sistematici}
	\label{tab:02tab6}
\end{tabella}
Da un confronto delle rette interpolanti si può stimare l'incertezza che può essere collegato all'imprecisione nella lettura della
 posizione del cavaliere portalampada. In particolare si può leggere l'effettivo valore del fuoco come quello collegato al
 campione realmente raccolto, e ad esso si può associare un errore sistematico legato alla massima distanza tra i fuochi fittizi
 ricavati dalle interpolazioni dei campioni fittizi, che risulta di 0.01~cm. Il valore finale risulta, quindi:
\[f^{\star}_2 = (6.60 \pm 0.01) \cm\]

Dove è stata considerata la sommma tra gli errori statistici e gli errori sistematici su p\textsubscript{o}, rispetto ai quali gli
 errori statistici precedentemente considerati risultano trascurabili.
 Il valore più giusto tra i due, alla luce dell'$\mathcal{F}$-test effettuato e delle considerazioni sugli errori sistematici, è il secondo:
 infatti la semplicità della retta interpolante a signolo parametro libero permette un più semplice studio degli errori collegati
 al posizionamento dell'oggetto. Un approccio simile per la ricerca dell'errore sistematico nel caso del doppio parametro libero
 renderebbe impossibile l'approssimazione dell'errore attraverso le formule sopra citate, rendendo difficoltosa una pratica stima
 dello stesso. Il valore ottenuto va comunque corretto tramite i coefficienti di aberrazione sferica, come verrà discusso più avanti.

\FloatBarrier
\subsection{Bessel}
L'ultimo metodo per la misura della lunghezza focale è il cosiddetto 
metodo di Bessel: questo metodo, usando una differenza delle 
posizioni, consente di ridurre l'errore sistematico dovuto alle 
imprecisioni nel posizionamento dei cavalieri. Il metodo consiste 
nel fissare lo schermo e l'oggetto, e muovere la lente fino a 
trovare i fuochi. 

Ci saranno due posizioni della lente, come spiegato sotto. Chiamiamo 
$L$ la distanza tra l'oggetto e lo schermo (sarà quindi fissata, e 
dovrà essere $L \geq 4f$), e $S=p_2-p_1$ la differenza di posizioni $p_1$ e $p_2$ della 
lente.  Dalla formula \eqref{eq:punticoniugati} si ricava, 
sostituendo $L$ e $S$,
\begin{equation}
f = \frac{L^2-S^2}{4L}
\end{equation}
e per trovare l'errore $\sigma_f$ deriviamo
\begin{equation} \label{eq:dfdl}
\frac{\partial f}{\partial L} = \frac{L^2+S^2}{4L^2} \approx \frac{1}{4}
\end{equation}
\begin{equation}
\frac{\partial f}{\partial S} = -\frac{S}{2L}
\end{equation}

Questi errori sono minimizzati per $L \approx 4f$, d'altra parte la 
profondità di campo rende difficile una stima precisa della 
posizione di coincidenza. Quindi meglio usare valori di $L$ di poco 
superiori a $4f$ (calcolato con le stime precedenti). Abbiamo dunque 
trovato una decina di misure di $p_1$ e $p_2$, e abbiamo calcolato per 
ciascuna il fuoco col suo errore (tutto ciò è illustrato nella 
\autoref{tab:bessel}) e infine abbiamo fatto una media pesata per 
ottenere la stima del fuoco col metodo di Bessel $f^{\star}_3$ e il 
suo errore $\sigma_{f^{\star}_3}$

\begin{tabella}
	\centering
	\begin{tabulary}{\textwidth}{CCCCC}
\toprule
$p_1$ & $p_2$ & $S$ & $f$ & $\sigma_f$\\ \midrule
21.35 & 46.55 & 25.20 & 6.58 & 0.27 \\ \midrule
21.25 & 46.45 & 25.20 & 6.58 & 0.27 \\ \midrule
21.35 & 46.60 & 25.25 & 6.56 & 0.28 \\ \midrule
21.30 & 46.45 & 25.15 & 6.59 & 0.27 \\ \midrule
21.35 & 46.45 & 25.10 & 6.61 & 0.27 \\ \midrule
21.35 & 46.45 & 25.10 & 6.61 & 0.27 \\ \midrule
21.35 & 46.50 & 25.15 & 6.59 & 0.27 \\ \midrule
21.30 & 46.45 & 25.15 & 6.59 & 0.27 \\ \midrule
21.35 & 46.55 & 25.20 & 6.58 & 0.27 \\ \midrule
21.30 & 46.45 & 25.15 & 6.59 & 0.27 \\
\bottomrule
\end{tabulary}

	\caption{Metodo di Bessel $[\cm\,]$}
	\label{tab:bessel}
\end{tabella}
% 03_tab_1.tex

\FloatBarrier
\subsection{Aberrazione sferica}
\label{subsec:aberrazione_sferica}
%**04_tab_1.tex**

I risultati delle misure sono riportati nelle prime colonne della \autoref{tab:04_tab_1.tex}.
\begin{tabella}
	\centering
	\begin{tabulary}{\textwidth}{CCCCCCCCC}
\toprule
$f_m$ & $f_i$ & $f_f$ & $f_p$ & $t$ & $l$ & $\sigma_l$ & $c$ & $\sigma_c$ \\ \midrule
4.18 & 8.00 & 10.65 & 9.33 & 1.93 & 5.1 & 1.3 & 1.75 & 0.45\\ \midrule
4.10 & 8.18 & 10.60 & 9.39 & 2.02 & 5.3 & 1.2 & 1.80 & 0.41\\ \midrule
4.15 & 8.15 & 10.39 & 9.27 & 1.99 & 5.1 & 1.1 & 1.74 & 0.38\\ \midrule
4.08 & 7.50 & 9.90 & 9.20 & 1.94 & 4.6 & 2.2 & 1.57 & 0.41\\ \midrule
4.25 & 7.90 & 10.35 & 9.13 & 1.90 & 4.9 & 1.2 & 1.66 & 0.42\\ \midrule
3.90 & 8.75 & 9.80 & 9.28 & 1.95 & 5.4 & 0.5 & 1.83 & 0.18\\ \midrule
4.90 & 8.00 & 10.30 & 9.15 & 1.87 & 4.2 & 1.1 & 1.44 & 0.39\\ \midrule
4.10 & 8.25 & 10.30 & 9.28 & - & 5.2 & 1.0 & 1.76 & 0.35\\ \midrule
4.00 & 8.25 & 10.10 & 9.18 & - & 5.2 & 0.9 & 1.76 & 0.31\\ \midrule
4.07 & 7.31 & 10.50 & 8.91 & - & 4.8 & 1.6 & 1.64 & 0.54\\
\bottomrule
\end{tabulary}


	\caption{Aberrazione sferica, $[\mm\,]$ tranne le ultime due colonne adimensionali}
	\label{tab:04_tab_1.tex}
\end{tabella}

Per la stima del coefficiente di aberrazione sferica $c$, si \`e considerata la formula
\[ c=\frac{f   l}{R^2}, \] 
con \( l = f_p - f_m = \frac{f_i + f_f}{2} - f_m \) e $f$ miglior stima del fuoco nel giallo dalle precedenti esperienze.


Per trovare $l$ si \`e calcolato $l_i$ su ogni terna di valori, propagandone l'errore quadraticamente. Nel calcolo della propagazione si \`e notato che l'incertezza su $f_m$ è trascurabile rispetto a quella su $f_p$: infatti risulta da evidenze sperimentali che la profondità di campo dei raggi prossimali all'asse ottico \`e molto maggiore di quella dei raggi marginali; è stato più semplice comprendere la posizione del fuoco marginale rispetto a quella del fuoco prossimale, per identificare il quale è stato necessario trovare i due valori all'interno del quale il fuoco fosse compreso. Al fuoco prossimale è stata associata, quindi, un'incertezza pari alla semiampiezza dell'intervallo di fuoco dei raggi prossimali, proprio in conseguenza delle difficolt\`a già espresse. Si \`e in seguito fornita una stima di $l$ come media pesata del campione cos\`i creato:
\[ l = (0.51 \pm 0.03) \cm. \]


Per la correzione di aberrazione sferica sulle stime $f^{\star}_j$ ottenute dalle precedenti esperienze, \`e necessario il valore $ l'$, definito come la distanza tra i punti di messa a fuoco rispettivamente dei raggi prossimali all'asse ottico e di quelli al margine del fascio, nel caso del diaframma a foro singolo (diametro $d_0=1.00 cm$). Per trovare tale valore si utilizza la definizione di aberrazione sferica longitudinale $l'=\frac{c(d/2)^2}{f}$: a partire dai dati dell'esperienza illustrata nella \autoref{subsec:autocollimazione}, si \`e trovata l'ordinata dell'intersezione tra le due rette, corrispondente al diametro del fascio creato, $d= 1.182 \cm$ (non si \`e ricercata l'incertezza di tale valore, in quanto la correzione per aberrazione sferica prevede l'utilizzo esclusivamente del valore di $l'$, ma non dell'incertezza associata). Risolvendo il sistema tra la formula sopra detta e $l=\frac{cR^2}{f}$, 
\[l' = l\left(\frac{d}{2R}\right)^2 = 0.09\cm .\]


Con tale $l'$, si \`e compiuta una correzione di aberrazione sferica per ogni $f^{\star}_j$ ottenuto dalle precedenti esperienze, sommando $l'/2$ alla stima stessa e andando a stimare nuovamente la sua incertezza. Infatti i risultati presentati nel caso delle prime tre esperienze trovano il fuoco della lente a meno delle correzioni per aberrazione sferica: a causa della costruzione dell'apparato, l'oggetto non è effettivamente puntiforme, ma ha un diametro finito $d$ a causa del diaframma. Ciò vuol dire che, analogamente a quanto è stato notato nell'esecuzione dell'esperimento IV, i raggi che incidono la lente a distanza maggiore dal centro della stessa convergono prima degli altri. A causa di questo fenomeno irriducibile, quello che è stato trovato nelle esperienze precedenti non è realmente il fuoco (considerato come punto in cui convergono i raggi prossimali all'asse ottico), ma è il centro dell'intervallo di messa a fuoco dell'immagine. Per questo, andando a correggere tale valore, risulta necessario aggiungere $\frac{l'}{2}$ per spostarsi dal punto in cui si \`e registrato il fuoco nelle prime tre esperienze al punto dove efffettivamente si ritiene più probabile sia il fuoco; all'incertezza va sommata quadraticamente la stessa quantità per considerare il fatto che non si sa esattamente dove sia il fuoco nell'intervallo che va da quello registrato a quello pi\`u probabile. Si considera dunque f come media pesata di tali valori:
\[ f = (6.65 \pm 0.03) \cm. \] 


Per quanto riguarda $c$, si \`e proceduto creando un nuovo valore per ogni set di dati; la formula di propagazione fornisce 
\[  \sigma^2_{c_i} =\frac{f^2    \sigma^2_{l_i} + l^2    \sigma^2_{f}}{R^4} \]
in cui si \`e considerata trascurabile l'incertezza su $R=1.40\cm$, fornito dal laboratorio.
Da questo campione si \`e ricavata la media pesata, con relativa incertezza:
\[ c = 1.7 \pm 0.1. \] 



Come ulteriore stima di $c$, si \`e utilizzata la formula
\[ c_t = \frac{t   f   (f-l)}{2R^3}, \]
dove $t$ \`e l'aberrazione sferica trasversale misurata, stimata con una media semplice, potendo considerare $\sigma$ costante:
\[ t = (0.194 \pm 0.002) \cm.\]
L'incertezza su $t$ si \`e trovata a posteriori con la formula RMS, per evitarne una difficile stima a partire dalla sensibilit\`a dello strumento. 
L'incertezza di $c_t$ si trova quindi con la propagazione: 
\[ \sigma_{c_t} = \sqrt{ \frac
{ f^2   (f-1)^2    \sigma^2_{t}  +  t^2   (2f-1)^2    \sigma^2{f}  +  t^2   f^2    \sigma^2_{l}}
{4R^6}
}   \]
\[c_t = 1.45 \pm 0.04. \]


La stima di $c_t$ attraverso $t$ \`e meno affidabile della stima di $c$ attraverso $l$, in quanto i valori di $t$ registrati sono influenzati da errori sistematici non trascurabili nel caso in cui la misura venga fatta non esattamente nel fuoco prossimale. Per questo motivo per la correzione del fuoco si \`e preferito utilizzare l'aberrazione sferica longitudinale rispetto a quella trasversale.
Anche la differenza tra le due stime $c$ e $c_t$ \`e imputabile ai medesimi errori sistematici su $t$. La difficolt\`a nella stima di tali errori rende inoltre impossibile una media tra le due stime, irrimediabilmente correlate.

\FloatBarrier
\subsection{Aberrazione cromatica e numero di Abbe}
I risultati delle misure sono riportati nella \autoref{tab:05_tab_1.tex}. 
Per questa esperienza si sono utilizzate le proprietà del doppietto 
acromatico "di Dollond", lente composta da una lente convergente e 
una divergente, le quali siano allineate sul medesimo asse ottico. 
Tale configurazione in condizioni particolari massimizza il numero 
di Abbe della lente stessa: infatti il doppietto acromatico fa 
sì che i raggi siano proiettati nel medesimo punto a prescindere 
dalla loro lunghezza d'onda. Per l'analisi si è considerato il 
doppietto acromatico come ideale, e non si è valutato l'errore 
sistematico legato al fatto che il raggio avrebbe potuto non essere 
effettivamente parallelo con uno dei due filtri. Tale proprietà 
risulta importante per la misurazione del numero di Abbe della lente 
presa in esame, in quanto è necessario che il fascio di partenza sia 
parallelo a prescindere dal filtro utilizzato.
\begin{tabella}
	\centering
	\begin{tabulary}{\textwidth}{CCCCC}
\toprule
$f_C$ & $f_F$ & $A$ & $V$ &$\sigma_V$
3.10 & 4.25 & 1.15 & 59 & 4
3.05 & 4.00 & 0.95 & 72 & 6
2.90 & 4.25 & 1.35 & 51 & 3
2.82 & 4.23 & 1.41 & 49 & 3
3.21 & 4.26 & 1.05 & 66 & 5
2.92 & 4.13 & 1.21 & 57 & 4
3.04 & 4.18 & 1.14 & 60 & 4
3.13 & 3.98 & 0.85 & 81 & 7
3.18 & 4.26 & 1.08 & 64 & 5
3.08 & 4.16 & 1.08 & 64 & 5
\bottomrule
\end{tabulary}



	\caption{Risultati aberrazione cromatica}
	\label{tab:05_tab_1.tex}
\end{tabella}

Per la stima del numero di Abbe $V$ si \`e utilizzata la formula
\[ V=\frac{f}{A} \]
dove \(A= f_C - f_F\) e $f_D$ \`e la miglior stima del fuoco nel 
giallo dalle precedenti esperienze. Per la stima di $A$, si è
utilizzata una media semplice sul campione ${A_i}$ creato a partire 
dai dati:
\[ A = (0.113 \pm 0.005) \cm .\]
Considerando invece il campione di \( V_i=\frac{f}{A_i} \), la formula di propagazione porta a 
\[  \sigma^2_{V_i} = \frac{
\frac{2f^2   s^2}{A_i^2}   \sigma^2 +  \sigma^2_{f}
}{A_i^2} \]
e la media pesata restituisce la nostra stima:
\[V= 56 \pm 1 .\]

